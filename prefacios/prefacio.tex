\thispagestyle{empty}

\chapter*{Resumen}

Gracias al rápido avance tecnológico y la evolución de la unidad de procesamiento de gráficos o \textit{GPU}, podemos experimentar con técnicas propuestas durante el siglo pasado, que eran poco eficientes debido al hardware del momento. El objetivo de este trabajo es hacer uso de este avance para probar nuevas técnicas de creación de escenas, utilizando una clase de funciones, conocidas como \textit{Funciones de distancia con signo} o \textit{FDS}\ref{ch:fds}. Desarrollaremos el proyecto en el lenguaje \textit{GLSL}\ref{ch:glsl}, donde veremos alguno de los nuevos tipos y operadores ya que es su sintaxis es similar a C. Las escenas creadas a partir de las \textit{Funciones de distancia con signo} son completamente analíticas, es decir, exactas, que a diferencia de otras técnicas que utilizan vértices. Por ejemplo, estas otras técnicas generan una esfera aproximada por un poliedro geodésico\footnote{Se trata de un poliedro convexo hecho de triángulos, el algoritmo: \url{https://stackoverflow.com/a/17795311}}. Para el renderizado de la escena tridimensional, en el cuarto capítulo, presentaremos un algoritimo con el nombre de \enquote{\textit{Spheremarcher}}\ref{sec:spheremarching}.

\vspace{0.7cm}

\noindent{
	\spacedlowsmallcaps{Palabras clave}
	\textit{gráficos}, \textit{marcher}, \textit{GLSL}, \textit{modelo de iluminación}
}

\cleardoublepage

\chapter*{Summary}

\begin{otherlanguage}{english}
In this project we are going to present a new graphic technique in thanks to the advance in technology and the continuously increasing of the efficiency of the GPU. We are going to divide it into five important chapters.

\paragraph{Chapter 1} In this first chapter we are going to present the mathematical foundation used to create an analytical scene for our project. We will give an introduction to the \textit{signed distance functions} and dive into them in later chapters. Because we are working with analytical scenes, we will present a theorem to calculate surface normal that will be in the Light Model. Finally, we will give two mathematical concepts: \textit{homeomorphism} and \textit{homotopies}, commonly used in textures.

\paragraph{Chapter 2} In this second chapter, we will present the language we will use through the whole project, called \textit{GLSL}. The syntax of this language is similar to C, we will be presenting some new mathematical types and primitive functions. Each section will also have tips and code examples.

\paragraph{Chapter 3} In this chapter we will analyze the analytical algorithm used to approximate a sceene that is created using \textit{signed distance functions}. We will be presenting the online enviroment used to create this project, called \textit{Shadertoy}, designed by Iñígo Quilez and Pol Jeremias in 2013.

\paragraph{Chapter 4} In this chapter, we will be presenting the basic operators to create an Illumination Model. We are going to implement an Illumination Model designed in the 70's, called Phong's Model. This model splits the light intensity into three different ones: ambient intensity, that affects with a constant value at the surface everywhere; diffuse intensity, dependent to the surface curvature and the light direction and specular intensity is calculated by the reflection of the light from the surface to the eye. Finally, we will be presenting the easiest method to create a shadow used to give a notion of depth.

\paragraph{Chapter 5} In this chapter we will be covering the main theme of this project, that are indispensable for the creating of a scene. \textit{Signed distance functions} are a special type of functions that returns a signed distance to a surface or perimetre, when the distance is positive, we would say that we are outside the scene, when it is negative, we will be inside an object. The zeros will represent the surface, but because we will be using a numerical algorithm to march it, we will define the surface as an interval. We will be proving some of the most important primitives for 2-dimensional and 3-dimensional spaces and two types of operators, one that will not affect the marcher and the other that will create artefacts and aggravate the efficiency of the marcher.

\paragraph{Chapter 6} In this penultimate chapter we will be presenting techniques to draw exact scenes, independently to the type of operator, we will talk about two problems, overestimation and underestimation. Overestimation happens when we are working with nonexact \textit{signed distance fields} and the marcher goes inside the surface or break through it. The underestimation happens in both cases, it considers background surfaces that, with more iterations, will be traced.

\paragraph{Chapter 7} Finally, in this last chapter, we will show how to use different colors and materials, we will give an ID to each object from the scene, that, will make some a lot of changes in code to work. We will present projection of a shape into texture coordinate space.

\vspace{0.7cm}
\noindent{
	\spacedlowsmallcaps{Keywords}
	\textit{graphics}, \textit{signed distance functions}, 
	\textit{illumination model}, \textit{texturing}
}

\end{otherlanguage}

\cleardoublepage
%\thispagestyle{empty}
%\noindent\rule[-1ex]{\textwidth}{2pt}\\[4.5ex]
%D. \textbf{Tutora/e(s)}, Profesor(a) del ...
%\vspace{0.5cm}
%\textbf{Informo:}
%\vspace{0.5cm}
%Que el presente trabajo, titulado \textit{\textbf{Chief}},
%ha sido realizado bajo mi supervisión por \textbf{Estudiante}, y autorizo la defensa de dicho trabajo ante el tribunal
%que corresponda.
%\vspace{0.5cm}
%Y para que conste, expiden y firman el presente informe en Granada a Junio de 2018.
%\vspace{1cm}
%\textbf{El/la director(a)/es: }
%\vspace{5cm}
%\noindent \textbf{(nombre completo tutor/a/es)}

%\chapter*{Agradecimientos}




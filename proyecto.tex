%%%%%%%%%%%%%%%%%%%%%%%%%%%%%%%%%%%%%%%%%
% Short Sectioned Assignment LaTeX Template Version 1.0 (5/5/12)
% This template has been downloaded from: http://www.LaTeXTemplates.com
% Original author:  Frits Wenneker (http://www.howtotex.com)
% License: CC BY-NC-SA 3.0 (http://creativecommons.org/licenses/by-nc-sa/3.0/)
%%%%%%%%%%%%%%%%%%%%%%%%%%%%%%%%%%%%%%%%%

% \documentclass[paper=a4, fontsize=11pt]{scrartcl} % A4 paper and 11pt font size
\documentclass[
	12pt,
	a4paper,
	english,
	spanish
]{scrbook}

% Images
\usepackage{eso-pic}

% Idioma
\usepackage[es-noindentfirst, es-lcroman, es-tabla]{babel}
%\spanishdashitems

% Titulo
\usepackage[
drafting=false,
dottedtoc=true,
parts,
pdfspacing,
beramono=false,
palatino=false,
%
floatperchapter=false,
linedheaders=true,
listings=false
]{notsoclassicthesis}

%\setsansfont[Scale=0.8]{Open Sans} 
%\renewfontface\chapterNumber[Scale=7, Color=000000]{EBGaramond}
%\setmonofont[Scale=0.75]{Go Mono}

% Matemáticas
\usepackage{amsmath, amsthm, amssymb}
\usepackage{mathtools}
\usepackage{commath}

% Teorema
\newtheoremstyle{theorem-style}  % Nombre del estilo
{\topsep}                                  % Espacio por encima
{\topsep}                                  % Espacio por debajo
{\itshape}                                  % Fuente del cuerpo
{0pt}                                  % Identación
{\scshape}                      % Fuente para la cabecera
{}                                 % Puntuación tras la cabecera
{5pt plus 1pt minus 1pt}                              % Espacio tras la cabecera
{\lsc{{\thmname{#1}\thmnumber{ #2}}.\thmnote{ (#3.)}}}  % Especificación de la cabecera
\theoremstyle{theorem-style}
\newtheorem{theorem}{Teorema}[section]
\newtheorem{corollary}[theorem]{Corolario}
\newtheorem{lemma}[theorem]{Lema}
\newtheorem{proposition}[theorem]{Proposición}
\newtheorem{question}{Pregunta}
\newtheorem{conjecture}[theorem]{Conjetura}
\newtheorem{remark}[theorem]{Nota}
\newtheoremstyle{definition-style}  % Nombre del estilo
{\topsep}                                  % Espacio por encima
{\topsep}                                  % Espacio por debajo
{}                                  % Fuente del cuerpo
{0pt}                                  % Identación
{}                      % Fuente para la cabecera
{.}                                 % Puntuación tras la cabecera
{5pt plus 1pt minus 1pt}                              % Espacio tras la cabecera
{\lsc{{\thmname{#1}\thmnumber{ #2}}\thmnote{ (#3)}}}  % Especificación de la cabecera
\theoremstyle{definition-style}
\newtheorem{definition}[theorem]{Definición}
\newtheorem{example}[theorem]{Ejemplo}
\newtheorem{notation}[theorem]{Notación}
\newtheorem{exercise}[theorem]{Ejercicio}


% Listas
\usepackage[inline]{enumitem}
\setlist[itemize]{ noitemsep, leftmargin=*}
\setlist[enumerate]{noitemsep, leftmargin=*}

% Posicionamiento
\usepackage{float}

% Código
\usepackage{listings}
\lstset{
	basicstyle=\ttfamily,%
	breaklines=true,%
	captionpos=b,                    % sets the caption-position to bottom
	tabsize=2,	                   % sets default tabsize to 2 spaces
	frame=none,
	numbers=left,
	xleftmargin=18pt,
	stepnumber=1,
	aboveskip=12pt,
	showstringspaces=false,
	keywordstyle=\bfseries,
	commentstyle=\itshape,
	numberstyle=\scriptsize\bfseries,
	morekeywords={sage},
}
\renewcommand{\lstlistingname}{Listado}

\usepackage{cite} %para incluir citas del archivo <nombre>.bib

% Lorem
\usepackage{blindtext}


\begin{document}

	% Plantilla portada UGR
	\begin{titlepage}
\newlength{\centeroffset}
\setlength{\centeroffset}{-0.5\oddsidemargin}
\addtolength{\centeroffset}{0.5\evensidemargin}
\thispagestyle{empty}

\noindent\hspace*{\centeroffset}

	\includegraphics[width=0.9\textwidth]{logos/logo_ugr.jpg}\\[1.4cm]

	\begin{addmargin}[2.56cm]{0cm}
		\begin{minipage}{\textwidth}
			\textsc{\bfseries E.T.S de Ingenierías Informática y de Telecomunicación}\\
			
			\textsc{GRADO EN INGENIERÍA INFORMÁTICA}
			
			\vspace{3.0cm}
			
			\spacedlowsmallcaps{TRABAJO FIN DE GRADO}\\[0.5cm]
			\begingroup
				\LARGE{\bfseries Funciones de distancia con signo}\\\bigskip
			\endgroup
	
			\vspace{3.0cm}
			
			\large{Autor.\\ Lukas Häring García}\\[0.4cm]
			\large{Director.\\ Juan Carlos Torres Cantero}\\[2cm]
			%\includegraphics[width=0.3\textwidth]{logos/etsiit_logo.png}\\[0.1cm]
			\textsc{---}\\
			%Granada, Junio de 201x
			Granada\\
			Curso académico 2019-2020
		\end{minipage}
	\end{addmargin}

\end{titlepage}


	% Plantilla prefacio UGR
	\thispagestyle{empty}

\chapter*{Resumen}

Gracias al rápido avance tecnológico y la evolución de la unidad de procesamiento de gráficos o \textit{GPU}, podemos experimentar con técnicas propuestas durante el siglo pasado, que eran poco eficientes debido al hardware del momento. El objetivo de este trabajo es hacer uso de este avance para probar nuevas técnicas de creación de escenas, utilizando una clase de funciones, conocidas como \textit{funciones de distancia con signo}(\textit{FDS}). Desarrollaremos el proyecto en el lenguaje \textit{GLSL}, donde veremos alguno de los nuevos tipos y operadores, ya que su sintaxis es similar a C. Las escenas creadas a partir de las \textit{funciones de distancia con signo} son completamente analíticas, es decir, exactas, a diferencia de las técnicas que utilizan vértices. Para el renderizado de la escena tridimensional, en el cuarto capítulo, presentaremos un algoritimo con el nombre de \enquote{\textit{Spheremarcher}}, un algoritmo numérico que aproxima la escena. Analizaremos y demostraremos algunas de las \textit{funciones de distancia con signo}, que denotaremos primitiva, comenzando con las primitivas de \(\mathbb{R}^2\) y algunos operadores que las trasforman en otras funciones, presentando el término d exactitud. Continuaremos con las de una dimensión superior, veremos algunas primitivas y operadores generalizados de \(\mathbb{R}^2\) y presentaremos operadores que junto a las \textit{funciones de distancia con signo 2D}, podremos generar otras \textit{3D} de manera exacta. Cuando trabajamos con funciones inexactas, la aproximación a una superficie también lo será, creando \enquote{artefactos}, en el sexto capítulo veremos como solucionarlos y qué consecuencias trae. Finalmente, en el último capítulo, veremos como implementar un sistema de materiales, asignaremos un identificador para cada figura haremos uso de este para devolver el material, en un primer ejemplo, devolveremos distintos colores y en otro ejemplo, una textura.

\vspace{0.7cm}

\noindent{
	\spacedlowsmallcaps{Palabras clave}
	\textit{gráficos}, \textit{marcher}, \textit{GLSL}, \textit{modelo de iluminación}
}

\cleardoublepage

\chapter*{Summary}

\begin{otherlanguage}{english}
A new graphic technique will be presented regarding the advance in technology and the continuously increasing of the efficiency of the GPU. We are going to divide it into seven important chapters.

\paragraph{Chapter 1} In this first chapter we are going to present the mathematical foundation used to create an analytical scene for our project. We will give an introduction to the \textit{signed distance functions} and dive into them later. Because we are working with analytical scenes, we will present a theorem to calculate surface normal, that are important to define the Light Model. Finally, we will give two mathematical concepts: \textit{homeomorphism} and \textit{homotopies}, commonly used in textures.

\paragraph{Chapter 2} In this second chapter, we will be presenting the language used through the project, called \textit{GLSL}. The syntax of this language is similar to C, we will be presenting some new mathematical types and primitive functions. Each section will also have tips and code examples.

\paragraph{Chapter 3} In this chapter we will analyze the analytical algorithm used to approximate a sceene that is created using \textit{signed distance functions}. We will be presenting the online enviroment used to create this project, called \textit{Shadertoy}, designed by Iñígo Quilez and Pol Jeremias in 2013.

\paragraph{Chapter 4} In this chapter, we will be presenting the basic operators to create an Illumination Model. In particular, the model we will be using was defined in the 70's, called \textit{Phong's Model}. This model splits the light intensity into three different ones: ambient intensity which affects over a surface with a constant value; diffuse intensity which depends on the surface curvature and light direction; finally, specular intensity is calculated by the reflection of the light from the surface to the eye. Shadows are important to give the sense of depth in an image, we will give a notion how to implement the umbra of a shadow.

\paragraph{Chapter 5} In this chapter we will be covering the main theme of this project: \textit{Signed Distance Functions}. These functions return a signed distance to a surface or perimetre, when the distance is positive, we define it as outside of an object; if it is negative, we define it as the interior of an object. The zeros will represent the surface, but in numerical algorithms, we will define the surface as an interval near to zero. We will be proving some of the most important primitives for 2-dimensional and 3-dimensional spaces. Finally, different operators will be presented that could affect the \textit{Marcher}, creating artefacts (errors in the resulting image), aggravating the efficiency of the marcher.

\paragraph{Chapter 6} In this penultimate chapter we will be presenting techniques to draw nearly exact scenes, independently to the type of operator, we will talk about two problems, overestimation and underestimation. Overestimation occurs when we are working with nonexact \textit{signed distance fields} and the marcher goes inside the surface or break through it. The underestimation happens in both cases, this happens when the algorithm stops earlier than it should.

\paragraph{Chapter 7} This last chapter, will present how to use different materials for each object, giving an ID to each point in the surface traced, this ID will be used to return a color and will be affected by the Illumination model, an example of texture mapping is given to prove that texturing is also possible, using the coordinate of the point approximate.

\vspace{0.7cm}
\noindent{
	\spacedlowsmallcaps{Keywords}
	\textit{graphics}, \textit{signed distance functions}, 
	\textit{illumination model}, \textit{texturing}
}

\end{otherlanguage}

\cleardoublepage
%\thispagestyle{empty}
%\noindent\rule[-1ex]{\textwidth}{2pt}\\[4.5ex]
%D. \textbf{Tutora/e(s)}, Profesor(a) del ...
%\vspace{0.5cm}
%\textbf{Informo:}
%\vspace{0.5cm}
%Que el presente trabajo, titulado \textit{\textbf{Chief}},
%ha sido realizado bajo mi supervisión por \textbf{Estudiante}, y autorizo la defensa de dicho trabajo ante el tribunal
%que corresponda.
%\vspace{0.5cm}
%Y para que conste, expiden y firman el presente informe en Granada a Junio de 2018.
%\vspace{1cm}
%\textbf{El/la director(a)/es: }
%\vspace{5cm}
%\noindent \textbf{(nombre completo tutor/a/es)}

%\chapter*{Agradecimientos}





	% Índice de contenidos
	%\newpage
	%\tableofcontents
	{\hypersetup{hidelinks}
		\tableofcontents
	}

	% Índice de imágenes y tablas
	%\newpage
	%\listoffigures

	% Si hay suficientes se incluirá dicho índice
	%\listoftables 
	%\newpage

	% Introducción 
	\chapter{Preliminares}


% https://it.wikipedia.org/wiki/Isosuperficie#cite_note-:2-3
\section{Funciones de distancia con signo}

Como su propio nombre indica, se trata de una función multi-dimensional 
\[f(x_1,\cdots,x_n)=d \mid d \in \mathbb{R}\]
Dado un punto \((z_1,\cdots,z_n)\), el valor que tomará \(f(z_1,\cdots,z_n)\) será la distancia signada $\pm d$ desde el punto a la superficie más cercana.\\\\
El signo de esta distancia contiene información sobre la escena. Si la distancia es positiva, nos encontramos en el exterior de una figura. Cuando esta es cero, nos encontraremos en la superficie (corteza). Por último, si la distancia es negativa, estaremos dentro de la figura.\\\\
La distancia es negativa, cuando tratamos de una \textit{superficie cerrada}. La distancia en positivo representa la profundidad del punto respecto de la superficie (corteza). 

\begin{definition}
	Sea \(f(x_1, \cdots, x_n)\) una función de distancia con signo, definimos como \textit{isosuperficie} \(S\) al conjunto de puntos tales que \(f(x_1, \cdots, x_n) = 0\)
\end{definition}



\section{Normal de un campo escalar}

Dada una superficie sobre una función de distancia \(f(x, y, z)=d=0\)

% https://tutorial.math.lamar.edu/classes/calciii/gradientvectortangentplane.aspx
\begin{theorem}
	El vector gradiente \(\nabla f(x_0, y_0, z_0)\)  es ortogonal a las superficies de nivel \(f(x, y, z)=k\) en el punto \((x_0, y_0, z_0)\).
\end{theorem}

\begin{proof}
	Dada una superficie de nivel \(f(x, y, z)=k\), el plano tangente en un punto \((x_0, y_0, z_0)\) está definido por,
	$$f_x(x_0, y_0, z_0)(x - x_0)+f_y(x_0, y_0, z_0)(y - y_0)+f_z(x_0, y_0, z_0)(z - z_0)$$
\end{proof}


Esta propiedad es fundamental para el desarrollo de la técnica. Ya que, ofrecerá las normales de los objetos (superficies) que serán utilizadas para una gran variedad de casos, por ejemplo, iluminación.



	% Descripción del problema y hasta donde se llega
	\input{secciones/02_descripcion}

	% Estado del arte
	% 	1. Crítica al estado del arte
	% 	2. Propuesta
	\input{secciones/03_estado_del_arte}
	
	\input{secciones/04_planificacion}

	% Análisis del problema
	% 1. Análisis de requisitos
	% 2. Análisis de las soluciones
	% 3. Solucion propuesta
	% 4. Análisis de seguridad
	\input{secciones/05_analisis}

	% Desarrollo bajo sprints: 
	% 	1. Permitir registros y login de usuarios
	% 	2. Desarrollo del sistema de incidencias
	% 	3. Desarrollo del sistema de denuncias administrativas y accidentes
	% 	4. Desarrollo del sistema de croquis
	%   5. Instalación de la aplicación de manera automática
	\input{secciones/06_implementacion}

	% Presupuesto

	% Conclusiones
	\input{secciones/07_conclusiones}

	% Trabajos futuros


	
	\newpage
	\bibliography{bibliografia}
	\bibliographystyle{plain}
	
\end{document}


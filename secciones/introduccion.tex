\pdfbookmark[1]{Introducción}{Introducción}
\addcontentsline{toc}{chapter}{\tocEntry{Introducción}}
\chapter*{Introducción}
Para la confección de este trabajo, ha sido indispensable la ayuda de mi tutor del proyecto, Prof. Dr. Juan Carlos Torres Cantero y al departamento de Lenguajes y Sistemas Informáticos.

\section*{Motivación}
El continuo interés personal sobre las matemáticas y la informática gráfica, empujada por la divulgación de Iñigo Quilez en su blog personal, ha sido el detonante del desarrollo de este proyecto, que agrupa los fundamentos matemáticos y algoritmos. Además, mencionar y agradecer a los autores a \textit{Tuong Phong} y \textit{Jhon C. Hart}, investigadores y padres de muchas de las técnicas que vamos a presentar, indispensables para el desarollo de este trabajo.

\section*{Objetivos}
El objetivo de este trabajo es la experimentación de nuevas técnicas de renderizado y  dar visibilidad al fundamento matemático que lo sustenta. Presentaremos un nuevo lenguaje de programación no visto durante la carrera, que será utilizado durante todo el desarrollo del trabajo. Desarrollaremos técnicas para formar nuestras propias escenas, que son creadas mediante funciones matemáticas, demostrando cada una de las propiedades presentadas.

\section*{Estructura}
Nuestro trabajo está estructurado en siete grandes apartados. Un primer apartado, en el que introducimos los principales conceptos que vamos a emplear a lo largo de nuestro trabajo, para demostrar el dominio de los conceptos trabajados durante estos cuatro años en la Universidad de Granada. Un segundo apartado, en el que presentamos el lenguaje de programación utilizado, \textit{GLSL}. Un tercer apartado, en el que desarrollamos el concepto de \textit{Marcher}, presentando el algoritmo principal de nuestro proyecto, \textit{Spheremarcher} y que será esencial para el trazado de una escena. En el cuarto capítulo, presentaremos algunos operadores importantes para la creación de un modelo de iluminación, junto a dos tipos de luces: radiales y direccionales. Haremos uso de un modelo de iluminación conocido con el nombre de \textit{Modelo Phong}, el cual divide la intensidad lumínica en tres, ambiental, difusa y especular. Un quinto capítulo, donde veremos en profundidad las \textit{funciones de distancia con signo}, en particular, veremos algunas funciones primitivas y sus respectivas demostraciones, que nos dará una intuición general para desarrollar nuevas. Además, vamos a presentar operadores para transformar estas funciones, donde encontramos: la traslación, rotación, escalado y simetría, presentando el concepto de \textit{isometría}. Un operador de deformación, que hará que tengamos que distinguir este tipo de funciones como exactas o inexactas y sus diferencias. Presentaremos también operadores entre dos \textit{funciones de distancia con signo}, el de adición, intersección y substracción. Para las \textit{funciones de distancia con signo} tridimensionales, definiremos dos operadores en este espacio que hacen uso de las funciones definidas en el espacio \textit{bidimensionales} para generar nuevas de manera exacta a partir de una extrusión o una revolución.
Una de las consecuencias de no trabajar con funciones exacta es la aparición de \enquote{artefactos}, en este penúltimo apartado, veremos como solucionarlos y que consecuencias tienen. Finalmente, en este último capítulo, veremos como pasar de una escena trazada en escala de grises a una escena con materiales para cada elemento de nuestra escena, realizaremos dos ejemplos, uno utilizando colores uniformes para cada figura y otro, utilizando una proyección de textura.\\\\
\pdfbookmark[1]{Introducción}{Introducción}
\addcontentsline{toc}{chapter}{\tocEntry{Introducción}}
\chapter*{Introducción}
Para la confección de este trabajo ha sido indispensable la ayuda de mi tutor del proyecto, Prof. Dr. Juan Carlos Torres Cantero y al departamento de Lenguajes y Sistemas Informáticos.\\\\
Nuestro trabajo está estructurado en siete grandes apartados. Un primer apartado en el que introducimos los principales conceptos que vamos a emplear a lo largo de nuestro trabajo, para demostrar el dominio de los conceptos trabajados durante estos cuatro años en la Universidad de Granada. Un segundo apartado en el que presentamos el lenguaje de programación utilizado, \textit{GLSL}\ref{ch:glsl}. Un tercer apartado en el que desarrollamos el algoritmo \textit{Spheremarcher}, esencial para la confección de este trabajo. En el cuarto capítulo, presentaremos algunos operadores importantes para la creación de un modelo de iluminación. Presentaremos dos tipos de luces: radiales y direccionales. Haremos uso de un modelo de iluminación conocido con el nombre de \textit{Modelo Phong}.Un quinto capítulo, donde veremos en profundidad las \textit{funciones de distancia con signo}\ref{ch:fds},  
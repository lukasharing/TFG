%%%%%%%%%%%%%%%%%%%%%%%%%%%%%%%%%%%%%%%%%
% Short Sectioned Assignment LaTeX Template Version 1.0 (5/5/12)
% This template has been downloaded from: http://www.LaTeXTemplates.com
% Original author:  Frits Wenneker (http://www.howtotex.com)
% License: CC BY-NC-SA 3.0 (http://creativecommons.org/licenses/by-nc-sa/3.0/)
%%%%%%%%%%%%%%%%%%%%%%%%%%%%%%%%%%%%%%%%%
% \documentclass[paper=a4, fontsize=11pt]{scrartcl} % A4 paper and 11pt font size
\documentclass[
	12pt,
	a4paper,
	english,
	spanish,
	dvipsnames,
    footinclude,
    headinclude,
]{scrbook}

% Images
\usepackage{subfig}
\usepackage{eso-pic}
\newcommand*{\fullref}[1]{\hyperref[{#1}]{\autoref*{#1} \nameref*{#1}}}

% Tablas
\usepackage{tabularx}

% Comillas
\usepackage{csquotes}

% Idioma
\usepackage[spanish, es-nodecimaldot]{babel}
%\usepackage{listingsutf8} % Codigo
%\spanishdashitems

% Enlaces: TODO Fix Colors
\usepackage[pdfencoding=auto]{hyperref}
\usepackage{xcolor}
\definecolor{CTurl}{named}{Maroon}
\definecolor{CTlink}{named}{RoyalBlue}
\hypersetup{
  colorlinks=true,
  citecolor=,
}

% Comillas
\let\oldquote\quote
\let\endoldquote\endquote
\renewenvironment{quote}[2][]
  {\if\relax\detokenize{#1}\relax
     \def\quoteauthor{#2}%
   \else
     \def\quoteauthor{#2~---~#1}%
   \fi
   \oldquote}
  {\par\nobreak\smallskip\hfill(\quoteauthor)%
   \endoldquote\addvspace{\bigskipamount}}

% Titulo
\usepackage[
drafting=false,
dottedtoc=true,
parts,
pdfspacing=true,
beramono=false,
palatino=false,
floatperchapter=true,
linedheaders=true,
listings=false
]{notsoclassicthesis}

% Gráficos
\usepackage{tikz}
\usepackage{pgfplots}

% Matemáticas
\usepackage{amsmath, amsthm, amssymb}
\usepackage{mathtools}
\usepackage{commath}
\usepackage{scalerel}

% Teorema
\newtheoremstyle{theorem-style}  % Nombre del estilo
{\topsep}                                  % Espacio por encima
{\topsep}                                  % Espacio por debajo
{\itshape}                                  % Fuente del cuerpo
{0pt}                                  % Identación
{\scshape}                      % Fuente para la cabecera
{}                                 % Puntuación tras la cabecera
{5pt plus 1pt minus 1pt}                              % Espacio tras la cabecera
{\lsc{{\thmname{#1}\thmnumber{ #2}}.\thmnote{ (#3.)}}}  % Especificación de la cabecera
\theoremstyle{theorem-style}
\newtheorem{theorem}{Teorema}[section]
\newtheorem{corollary}[theorem]{Corolario}
\newtheorem{lemma}[theorem]{Lema}
\newtheorem{proposition}[theorem]{Proposición}
\newtheorem{question}{Pregunta}
\newtheorem{conjecture}[theorem]{Conjetura}
\newtheorem{remark}[theorem]{Nota}
\newtheoremstyle{definition-style}  % Nombre del estilo
{\topsep}                                  % Espacio por encima
{\topsep}                                  % Espacio por debajo
{}                                  % Fuente del cuerpo
{0pt}                                  % Identación
{}                      % Fuente para la cabecera
{.}                                 % Puntuación tras la cabecera
{5pt plus 1pt minus 1pt}                              % Espacio tras la cabecera
{\lsc{{\thmname{#1}\thmnumber{ #2}}\thmnote{ (#3)}}}  % Especificación de la cabecera
\theoremstyle{definition-style}
\newtheorem{definition}[theorem]{Definición}
\newtheorem{example}[theorem]{Ejemplo}
\newtheorem{notation}[theorem]{Notación}
\newtheorem{exercise}[theorem]{Ejercicio}

% Listas
\usepackage[inline]{enumitem}
\setlist[itemize]{ noitemsep, leftmargin=*}
\setlist[enumerate]{noitemsep, leftmargin=*}

% Posicionamiento
\usepackage{float}
\newcommand\tab[1][1cm]{\hspace*{#1}}
\usepackage{pbox}

% Código
\usepackage{listings}
\lstset{
    basicstyle=\ttfamily,
    extendedchars=true,
    inputencoding=utf8,
	breaklines=true,%
	captionpos=b,
	tabsize=1
	showtabs=false,
	showstringspaces=false,
	language=C,
	numbers=left,
	xleftmargin=0pt,
	stepnumber=1,
	aboveskip=5pt,
	keywordstyle=\bfseries,
	commentstyle=\color{OliveGreen}\itshape,
	numberstyle=\scriptsize\bfseries,
	morekeywords={sage},
	literate=
        {á}{{\'a}}1
        {é}{{\'e}}1
        {í}{{\'i}}1
        {ó}{{\'o}}1
        {ú}{{\'u}}1
        {ñ}{{\~n}}1
}
\renewcommand{\lstlistingname}{Listado}

\usepackage{cite} %para incluir citas del archivo <nombre>.bib

% Lorem
\usepackage{blindtext}

\begin{document}

	% Plantilla portada UGR
	\begin{titlepage}
\newlength{\centeroffset}
\setlength{\centeroffset}{-0.5\oddsidemargin}
\addtolength{\centeroffset}{0.5\evensidemargin}
\thispagestyle{empty}

\noindent\hspace*{\centeroffset}

	\includegraphics[width=0.9\textwidth]{logos/logo_ugr.jpg}\\[1.4cm]

	\begin{addmargin}[2.56cm]{0cm}
		\begin{minipage}{\textwidth}
			\textsc{\bfseries E.T.S de Ingenierías Informática y de Telecomunicación}\\
			
			\textsc{GRADO EN INGENIERÍA INFORMÁTICA}
			
			\vspace{3.0cm}
			
			\spacedlowsmallcaps{TRABAJO FIN DE GRADO}\\[0.5cm]
			\begingroup
				\LARGE{\bfseries Funciones de distancia con signo}\\\bigskip
			\endgroup
	
			\vspace{3.0cm}
			
			\large{Autor.\\ Lukas Häring García}\\[0.4cm]
			\large{Director.\\ Juan Carlos Torres Cantero}\\[2cm]
			%\includegraphics[width=0.3\textwidth]{logos/etsiit_logo.png}\\[0.1cm]
			\textsc{---}\\
			%Granada, Junio de 201x
			Granada\\
			Curso académico 2019-2020
		\end{minipage}
	\end{addmargin}

\end{titlepage}


	% Plantilla prefacio UGR
	\thispagestyle{empty}

\chapter*{Resumen}

Gracias al rápido avance tecnológico y la evolución de la unidad de procesamiento de gráficos o \textit{GPU}, podemos experimentar con técnicas propuestas durante el siglo pasado, que eran poco eficientes debido al hardware del momento. El objetivo de este trabajo es hacer uso de este avance para probar nuevas técnicas de creación de escenas, utilizando una clase de funciones, conocidas como \textit{funciones de distancia con signo}(\textit{FDS}). Desarrollaremos el proyecto en el lenguaje \textit{GLSL}, donde veremos alguno de los nuevos tipos y operadores, ya que su sintaxis es similar a C. Las escenas creadas a partir de las \textit{funciones de distancia con signo} son completamente analíticas, es decir, exactas, a diferencia de las técnicas que utilizan vértices. Para el renderizado de la escena tridimensional, en el cuarto capítulo, presentaremos un algoritimo con el nombre de \enquote{\textit{Spheremarcher}}, un algoritmo numérico que aproxima la escena. Analizaremos y demostraremos algunas de las \textit{funciones de distancia con signo}, que denotaremos primitiva, comenzando con las primitivas de \(\mathbb{R}^2\) y algunos operadores que las trasforman en otras funciones, presentando el término d exactitud. Continuaremos con las de una dimensión superior, veremos algunas primitivas y operadores generalizados de \(\mathbb{R}^2\) y presentaremos operadores que junto a las \textit{funciones de distancia con signo 2D}, podremos generar otras \textit{3D} de manera exacta. Cuando trabajamos con funciones inexactas, la aproximación a una superficie también lo será, creando \enquote{artefactos}, en el sexto capítulo veremos como solucionarlos y qué consecuencias trae. Finalmente, en el último capítulo, veremos como implementar un sistema de materiales, asignaremos un identificador para cada figura haremos uso de este para devolver el material, en un primer ejemplo, devolveremos distintos colores y en otro ejemplo, una textura.

\vspace{0.7cm}

\noindent{
	\spacedlowsmallcaps{Palabras clave}
	\textit{gráficos}, \textit{marcher}, \textit{GLSL}, \textit{modelo de iluminación}
}

\cleardoublepage

\chapter*{Summary}

\begin{otherlanguage}{english}
A new graphic technique will be presented regarding the advance in technology and the continuously increasing of the efficiency of the GPU. We are going to divide it into seven important chapters.

\paragraph{Chapter 1} In this first chapter we are going to present the mathematical foundation used to create an analytical scene for our project. We will give an introduction to the \textit{signed distance functions} and dive into them later. Because we are working with analytical scenes, we will present a theorem to calculate surface normal, that are important to define the Light Model. Finally, we will give two mathematical concepts: \textit{homeomorphism} and \textit{homotopies}, commonly used in textures.

\paragraph{Chapter 2} In this second chapter, we will be presenting the language used through the project, called \textit{GLSL}. The syntax of this language is similar to C, we will be presenting some new mathematical types and primitive functions. Each section will also have tips and code examples.

\paragraph{Chapter 3} In this chapter we will analyze the analytical algorithm used to approximate a sceene that is created using \textit{signed distance functions}. We will be presenting the online enviroment used to create this project, called \textit{Shadertoy}, designed by Iñígo Quilez and Pol Jeremias in 2013.

\paragraph{Chapter 4} In this chapter, we will be presenting the basic operators to create an Illumination Model. In particular, the model we will be using was defined in the 70's, called \textit{Phong's Model}. This model splits the light intensity into three different ones: ambient intensity which affects over a surface with a constant value; diffuse intensity which depends on the surface curvature and light direction; finally, specular intensity is calculated by the reflection of the light from the surface to the eye. Shadows are important to give the sense of depth in an image, we will give a notion how to implement the umbra of a shadow.

\paragraph{Chapter 5} In this chapter we will be covering the main theme of this project: \textit{Signed Distance Functions}. These functions return a signed distance to a surface or perimetre, when the distance is positive, we define it as outside of an object; if it is negative, we define it as the interior of an object. The zeros will represent the surface, but in numerical algorithms, we will define the surface as an interval near to zero. We will be proving some of the most important primitives for 2-dimensional and 3-dimensional spaces. Finally, different operators will be presented that could affect the \textit{Marcher}, creating artefacts (errors in the resulting image), aggravating the efficiency of the marcher.

\paragraph{Chapter 6} In this penultimate chapter we will be presenting techniques to draw nearly exact scenes, independently to the type of operator, we will talk about two problems, overestimation and underestimation. Overestimation occurs when we are working with nonexact \textit{signed distance fields} and the marcher goes inside the surface or break through it. The underestimation happens in both cases, this happens when the algorithm stops earlier than it should.

\paragraph{Chapter 7} This last chapter, will present how to use different materials for each object, giving an ID to each point in the surface traced, this ID will be used to return a color and will be affected by the Illumination model, an example of texture mapping is given to prove that texturing is also possible, using the coordinate of the point approximate.

\vspace{0.7cm}
\noindent{
	\spacedlowsmallcaps{Keywords}
	\textit{graphics}, \textit{signed distance functions}, 
	\textit{illumination model}, \textit{texturing}
}

\end{otherlanguage}

\cleardoublepage
%\thispagestyle{empty}
%\noindent\rule[-1ex]{\textwidth}{2pt}\\[4.5ex]
%D. \textbf{Tutora/e(s)}, Profesor(a) del ...
%\vspace{0.5cm}
%\textbf{Informo:}
%\vspace{0.5cm}
%Que el presente trabajo, titulado \textit{\textbf{Chief}},
%ha sido realizado bajo mi supervisión por \textbf{Estudiante}, y autorizo la defensa de dicho trabajo ante el tribunal
%que corresponda.
%\vspace{0.5cm}
%Y para que conste, expiden y firman el presente informe en Granada a Junio de 2018.
%\vspace{1cm}
%\textbf{El/la director(a)/es: }
%\vspace{5cm}
%\noindent \textbf{(nombre completo tutor/a/es)}

%\chapter*{Agradecimientos}





	% TOC
	%\newpage
	%\tableofcontents
	{\hypersetup{hidelinks}
		\tableofcontents
	}

	% TOC Imagenes
	\newpage
	\listoffigures

	% TOC Tablas
	\listoftables 
	\newpage

	% Introduccion
	\pdfbookmark[1]{Introducción}{Introducción}
\addcontentsline{toc}{chapter}{\tocEntry{Introducción}}
\chapter*{Introducción}
Para la confección de este trabajo, ha sido indispensable la ayuda de mi tutor del proyecto, Prof. Dr. Juan Carlos Torres Cantero y al departamento de Lenguajes y Sistemas Informáticos.

\section*{Motivación}
El continuo interés personal sobre las matemáticas y la informática gráfica, empujada por la divulgación de Iñigo Quilez en su blog personal, ha sido el detonante del desarrollo de este proyecto, que agrupa los fundamentos matemáticos y algoritmos. Además, mencionar y agradecer a los autores a \textit{Tuong Phong} y \textit{Jhon C. Hart}, investigadores y padres de muchas de las técnicas que vamos a presentar, indispensables para el desarollo de este trabajo.

\section*{Objetivos}
El objetivo de este trabajo es la experimentación de nuevas técnicas de renderizado y  dar visibilidad al fundamento matemático que lo sustenta. Presentaremos un nuevo lenguaje de programación no visto durante la carrera, que será utilizado durante todo el desarrollo del trabajo. Desarrollaremos técnicas para formar nuestras propias escenas, que son creadas mediante funciones matemáticas, demostrando cada una de las propiedades presentadas.

\section*{Estructura}
Nuestro trabajo está estructurado en siete grandes apartados. Un primer apartado, en el que introducimos los principales conceptos que vamos a emplear a lo largo de nuestro trabajo, para demostrar el dominio de los conceptos trabajados durante estos cuatro años en la Universidad de Granada. Un segundo apartado, en el que presentamos el lenguaje de programación utilizado, \textit{GLSL}. Un tercer apartado, en el que desarrollamos el concepto de \textit{Marcher}, presentando el algoritmo principal de nuestro proyecto, \textit{Spheremarcher} y que será esencial para el trazado de una escena. En el cuarto capítulo, presentaremos algunos operadores importantes para la creación de un modelo de iluminación, junto a dos tipos de luces: radiales y direccionales. Haremos uso de un modelo de iluminación conocido con el nombre de \textit{Modelo Phong}, el cual divide la intensidad lumínica en tres, ambiental, difusa y especular. Un quinto capítulo, donde veremos en profundidad las \textit{funciones de distancia con signo}, en particular, veremos algunas funciones primitivas y sus respectivas demostraciones, que nos dará una intuición general para desarrollar nuevas. Además, vamos a presentar operadores para transformar estas funciones, donde encontramos: la traslación, rotación, escalado y simetría, presentando el concepto de \textit{isometría}. Un operador de deformación, que hará que tengamos que distinguir este tipo de funciones como exactas o inexactas y sus diferencias. Presentaremos también operadores entre dos \textit{funciones de distancia con signo}, el de adición, intersección y substracción. Para las \textit{funciones de distancia con signo} tridimensionales, definiremos dos operadores en este espacio que hacen uso de las funciones definidas en el espacio \textit{bidimensionales} para generar nuevas de manera exacta a partir de una extrusión o una revolución.
Una de las consecuencias de no trabajar con funciones exacta es la aparición de \enquote{artefactos}, en este penúltimo apartado, veremos como solucionarlos y que consecuencias tienen. Finalmente, en este último capítulo, veremos como pasar de una escena trazada en escala de grises a una escena con materiales para cada elemento de nuestra escena, realizaremos dos ejemplos, uno utilizando colores uniformes para cada figura y otro, utilizando una proyección de textura.\\\\
	% Preliminares 
	\chapter{Preliminares}


% https://it.wikipedia.org/wiki/Isosuperficie#cite_note-:2-3
\section{Funciones de distancia con signo}

Como su propio nombre indica, se trata de una función multi-dimensional 
\[f(x_1,\cdots,x_n)=d \mid d \in \mathbb{R}\]
Dado un punto \((z_1,\cdots,z_n)\), el valor que tomará \(f(z_1,\cdots,z_n)\) será la distancia signada $\pm d$ desde el punto a la superficie más cercana.\\\\
El signo de esta distancia contiene información sobre la escena. Si la distancia es positiva, nos encontramos en el exterior de una figura. Cuando esta es cero, nos encontraremos en la superficie (corteza). Por último, si la distancia es negativa, estaremos dentro de la figura.\\\\
La distancia es negativa, cuando tratamos de una \textit{superficie cerrada}. La distancia en positivo representa la profundidad del punto respecto de la superficie (corteza). 

\begin{definition}
	Sea \(f(x_1, \cdots, x_n)\) una función de distancia con signo, definimos como \textit{isosuperficie} \(S\) al conjunto de puntos tales que \(f(x_1, \cdots, x_n) = 0\)
\end{definition}



\section{Normal de un campo escalar}

Dada una superficie sobre una función de distancia \(f(x, y, z)=d=0\)

% https://tutorial.math.lamar.edu/classes/calciii/gradientvectortangentplane.aspx
\begin{theorem}
	El vector gradiente \(\nabla f(x_0, y_0, z_0)\)  es ortogonal a las superficies de nivel \(f(x, y, z)=k\) en el punto \((x_0, y_0, z_0)\).
\end{theorem}

\begin{proof}
	Dada una superficie de nivel \(f(x, y, z)=k\), el plano tangente en un punto \((x_0, y_0, z_0)\) está definido por,
	$$f_x(x_0, y_0, z_0)(x - x_0)+f_y(x_0, y_0, z_0)(y - y_0)+f_z(x_0, y_0, z_0)(z - z_0)$$
\end{proof}


Esta propiedad es fundamental para el desarrollo de la técnica. Ya que, ofrecerá las normales de los objetos (superficies) que serán utilizadas para una gran variedad de casos, por ejemplo, iluminación.


	% Lenguaje
	\chapter{Lenguaje GLSL}
% https://www.khronos.org/registry/OpenGL/specs/gl/GLSLangSpec.4.40.pdf
Vamos a presentar el lenguaje que se ha elegido. El lenguaje \textit{OpenGL Shading Language} (\textit{GLSL}) tiene una sintaxis similar a C\footnote{En su definición, se utilizó como referencia C ya que ....}. Se ha escogido debido a que es y ha sido uno de los lenguajes más utilizados la última década % Poner Referencia
. Aún así, han aparecido nuevos lenguajes de programación de shaders, como por ejemplo, Vulkan.\\\\
Vamos a ignorar el proceso de creación de escena con OpenGL, ya que utilizaremos la plataforma \textit{Shadertoy}\footnote{Fue creado por Inigo Quilez}, que nos crea un entorno adecuado para desarrollar nuestro trabajo. \textit{Shadertoy}, tiene construido por detrás, una escena con una superficie formada por 4 vértices, con dos triángulos. La cámara de la escena, con proyección ortogonal y cuyo viewport concuerda con los cuatro vértices.\\\\
A esta superficie se le va a asignar una textura, que como bien sabemos, las coordenadas de la textura \((u,v)\) están normalizadas. Es decir, vamos a trabajar sobre una textura, o shader y así crear una sub-escena. Para cada texel, utilizaremos un \textit{Marcher}, que estimará la escena. Cada elemento de la escena recibirá un material y una textura. Se calcularán las normales de los objetos que, junto al modelo de iluminación, crearemos una escena más viva.
\section{Tipos de GLSL}
Como ya hemos dicho, tiene una sintaxis similar a C, aunque añade y elimina ciertos tipos.
\begin{table}[h]
    \begin{tabularx}{\textwidth}{l|X}
      \toprule
      Tipo & Definición\\
      \midrule
      int & Entero con signo, igual que en C\\
      float & Número real como en C, aunque no existe el tipo \textit{double}. \\
      bool & Igual que en C++, tiene dos valores y ocupa un byte. \\
      vecN & Se trata de un nuevo tipo, vector matemático, es decir, n-upla de floats, están definidos: vec2, vec3, vec4. \\
      matN & Se trata de una matriz cuadrada de floats, tenemos definidos: mat2, mat3, mat4. \\
      matNxM & Una matriz rectangular de dimensiones \(N\times M\), encontramos: mat2x2, mat2x3, mat2x4, mat3x2, mat3x3, mat3x4, mat4x2, mat4x3, mat4x4. \\
      \bottomrule
    \end{tabularx}
  \end{table}
\newpage
% Vectores GLSL
\section{Vectores en GLSL}
El tipo \textit{vector}, definido por una n-úpla \((x, y, z, w)\) ó \((r, g, b, a)\), nos será esencial para la creación de nuestros algoritmos. Utilizaremos el operador ".", para acceder y crear vectores a partir de las componentes descritas anteriormente, sin combinar.
\begin{table}[h]
    \begin{tabularx}{\textwidth}{l|X}
      \toprule
      Accesos & Definición\\
      \midrule
      vector.x & Devuelve un flotante con valor de la primera componente de la variable vector.\\
      vector.yx & Devuelve un vector bidimensional donde la primera componente tiene el valor de \(y\) y la segunda componente, el valor de \(x\). \\
      vector.rrrr & Devuelve un \textit{vec4} donde las compontentes son iguales a la primera componente del vector. \\
      vector.rx & No compilará, ya que no se permite mezclar. \\
      \bottomrule
    \end{tabularx}
  \end{table}
\\
El constructor de \textit{vec} nos va a ofrecer una riqueza semántica.
\begin{table}[h]
    \begin{tabularx}{\textwidth}{l|X}
      \toprule
      Constructores & Definición\\
      \midrule
      vecN(float, \(\cdots\), float) & \(N\) valores flotantes para cada componente.\\
      vecN(vecM, float) & Asigna los primeros valores del vector los valores del segundo, donde \(M+1=N\) y el último atributo, el valor del elemento flotante. \\
      vecN(float, vecM) & Asigna el último atributo, el valor del elemento flotante. Los restantes valores del vector, los valores del segundo con \(M+1=N\). \\
      vecN(vecP, vecQ) & Los \(P\) primeros atributos del vector con \textit{vecP}, los Q restantes, del vector \textit{vecQ}. \\
      \bottomrule
    \end{tabularx}
\end{table}
\\
Algunos de los operadores presentes de forma nativa.
\begin{table}[h]
    \begin{tabularx}{\textwidth}{l|X}
      \toprule
      Función & Definición\\
      \midrule
      length(vecN vector) & Devuelve el módulo del vector.\\
      distance(vecN p1, vecN p2) & Devuelve la distancia entre los extremos de los vectores. \\
      normalize(vecN vector) & Devuelve el vector normalizado. \\
      dot(vecN v1, vecN v2) & Devuelve el valor del \textbf{producto escalar} de ambos vectores. \\
      cross(vecN v1, vecN v2) & Devuelve el vector resultante de realizar el \textbf{producto vectorial}. \\
      \bottomrule
    \end{tabularx}
\end{table}
% Matrices
\newpage
\section{Matrices GLSL}
Para realizar accesos a las componentes de las matrices \textit{mat}, vamos a utilizar el operador de acceso \([y][x]\), que es el mismo operador de acceso a las matrices en C.\\
Veamos alguno de los constructores para la creación de estas matrices \textit{matNxM} y \textit{matN}.
\begin{table}[h]
    \begin{tabularx}{\textwidth}{l|X}
      \toprule
      Constructor & Definición\\
      \midrule
      matN(matM) & Submatriz cuadrada \(N\) superior izquierda de \textit{matM}.\\
      matNxM(matQxP) & Submatriz \(NxM\) superior izquierda de \textit{matQxP}.\\
      matN(float, \(\cdots\), float) & \(N^2\) valores flotantes.\\
      matNxM(float, \(\cdots\), float) & \(N\times M\) valores flotantes. \\
      matN(vecN,..., vecN) & Formado por \(N\) vectores \(N\)-dimensionales. \\
      matNxM(vecM,..., vecM) & Formado por \(N\) vectores \(M\)-dimensionales. \\
      \pbox{10cm}{
      matN(\\
      \tab[1cm]vecM,float,
      \\\tab[1cm]...,
      \\\tab[1cm] vecM, float
      \\)
      }& Formado por \(N\) vectores \((M - 1)\)-dimensionales y \(N\) valores flotantes. \\
      \pbox{10cm}{
      matNxM(\\
      \tab[1cm]vecP,float,
      \\\tab[1cm]...,
      \\\tab[1cm]vecP,float
      \\)
      } & Formado por \(N\) vectores (M -1)-dimensionales y M valores flotantes. \\
      \bottomrule
    \end{tabularx}
  \end{table}
\\
No hemos visto todas las combinaciones posibles, pero sí, las más utilizadas. Estas combinaciones tienen que respetar siempre el tamaño de la matriz, pudiéndose intercalar vectores \textit{vecN} y valores flotantes.\\\\
De la misma forma, vamos a ver alguno de los operadores entre matrices.
\begin{table}[h]
    \begin{tabularx}{\textwidth}{l|X}
      \toprule
      Función & Definición\\
      \midrule
      transpose(mat matrix) & Devuelve la matriz cuadrada traspuesta.\\
      matrix1 * matrix2 & Devuelve el producto de matrices. \\
      matrixCompMult(mat m1, mat m2) & Devuelve el producto de \textit{Hadamar} de matrices con la misma dimensión. \\
      \bottomrule
    \end{tabularx}
\end{table}
\newpage
\section{Operadores Matemáticos}
Encontramos los operadores usuales \(+,-,*,/,\%\). Para los números enteros, los operadores binarios, \({<<, >>, \mid, \And, \string^}\). GLSL permite utilizar operadores de la clase \textit{Math.h} sobre los tipos \textit{float} y \textit{vecN}, que recibirán, a partir de ahora, el nombre de, \textit{genType}\footnote{Definido en el siguiente libro}. Sobre los tipos \textit{genType}, aplicará la función utilizada, componente a componente. Ayudando a tener un código más compacto y legible.
\begin{table}[h]
    \begin{tabularx}{\textwidth}{l|X}
      \toprule
      Función & Definición\\
      \midrule
      radians(genType var)& Devuelve la conversión de grados en radianes. \\
      degrees(genType var) & Devuelve la conversión de radianes en grados. \\
      sin(genType var) & Aplica la función \textit{seno} sobre la variable. \\
      cos(genType var) & Aplica la función \textit{coseno} sobre la variable. \\
      tan(genType var) & Aplica la función \textit{tangente} sobre la variable. \\
      asin(genType var) & Aplica la función \textit{arco seno} sobre la variable. \\
      acos(genType var) & Aplica la función \textit{arco coseno} sobre la variable. \\
      atan(genType var) & Aplica la función \textit{arco tagente} sobre la variable. \\
      \pbox{10cm}{
      pow(\\
      \tab[1cm]genType a,\\
      \tab[1cm]genType b \\
      )} & Calcula la primera variable elevada a la segunda. \\
      exp(genType var) & Calcula el número \(e\) elevado a la variable. \\
      exp2(genType var) & Calcula el número \(2\) elevado a la variable. \\
      log(genType var) & Aplica la función logarítmica sobra la variable. \\
      sqrt(genType var) & Aplica la raíz cuadrada sobre la variable. \\
      \bottomrule
    \end{tabularx}
\end{table}
\\
Veamos algunas de las equivalencias que resulta de utilizar los distintos tipos.
\begin{table}[h]
    \begin{tabularx}{\textwidth}{l|X}
      \toprule
      Ejemplo & Equivalencia\\
      \midrule
      exp(10.0) & exp(10.0)\\
      cos(ve2(a, b)) & vec2(cos(a),cos(b))\\
      pow(3.1415, -1.2) & pow(3.1415, -1.2) \\
      pow(vec2(a,b), n) & Distintos tipos, no compila. \\
      pow(vec2(a,b), vec2(c,d)) & vec2(pow(a, c),pow(b, d)) \\
      \bottomrule
    \end{tabularx}
\end{table}
\newpage
Hay otra serie de funciones que nos será muy útil para la creación de texturas, post-procesados, la creación de \textit{Funciones de Distancia con Signo}, etc. 
\begin{table}[h]
    \begin{tabularx}{\textwidth}{l|X}
      \toprule
      Función & Definición\\
      \midrule
      abs(genType var) & Aplica la función valor absoluto sobre la variable.\\
      sign(genType var) & Aplica la función signo sobre la variable.\\
      floor(genType var) & Aplica la función de redondeo inferior sobre la variable.\\
      round(genType var) & Aplica la función de redondeo normal sobre la variable.\\
      fract(genType var) & Toma las partes fraccionales de la variable.\\
      min(genType a, genType b) & Aplica la función mínimo sobre la variable.\\
      max(genType a, genType b) & Aplica la función máximo sobre la variable.\\
      \pbox{10cm}{
      clamp(\\
      \tab[0.5cm]genType v,\\
      \tab[0.5cm](genType ó float) min, \\
      \tab[0.5cm](genType ó float) max \\
      )} & Aplica la función acotado inferior \textit{min} y superior \textit{max}, sobre la variable.\\
      \pbox{10cm}{
      mix(\\
      \tab[0.5cm]genType a,\\
      \tab[0.5cm]genType b, \\
      \tab[0.5cm](genType ó float ó bool) h \\
      )} & Define \textit{la función de mezcla con peso} sobre la variable.\\
      \bottomrule
    \end{tabularx}
\end{table}
\\
Veamos algunos ejemplos y sus equivalencias.
\begin{table}[h]
    \begin{tabularx}{\textwidth}{l|X}
      \toprule
      Ejemplo & Equivalencia\\
      \midrule
      min(ve2(a, b), ve2(c, d)) & ve2(min(a, c), min(b, d))\\
      clamp(vec2(a,b), m, n) & vec2(clamp(a, m, n),clamp(b, m, n))\\
      min(ve2(a, b), ve2(c, d)) & ve2(min(a, c), min(b, d))\\
      \pbox{10cm}{
      mix(\\
      \tab[0.5cm]vec2(a,b),\\
      \tab[0.5cm]vec2(c,d), \\
      \tab[0.5cm]vec2(e,f)\\
      )} & vec2(mix(a,c,e), mix(b,d,f)) \\
      
      \bottomrule
    \end{tabularx}
\end{table}
\newpage
Los operadores relacionales son los mismos que en el lenguaje C. Encontramos los operadores \(<, <=, >, >=, !, ==, !=\).
\section{Acceso a Texturas}
Desde \textit{Shadertoy}, podemos utilizar diferentes canales y seleccionar distinto tipos de archivos, en nuestro caso, utilizaremos aquellas que aparecen en la pestaña: "\textit{Texturas}". Una vez seleccionada una textura, podemos utilizar la variable de tipo \textit{sampler2D} con nombre \textit{iChannelN} como canal N-ésimo, para acceder a los valores del archivo.\\\\
Existen las siguientes funciones que permiten obtener el color de un píxel de una variable tipo \textit{sampler2D} en las coordenadas de textura \((s,t)\), que se encuentran normalizados.
\begin{table}[h]
    \begin{tabularx}{\textwidth}{l|X}
      \toprule
      Función & Definición\\
      \midrule
      \pbox{10cm}{
      texture(\\
      \tab[1cm]sampler2D iChannelN,\\
      \tab[1cm]vec2 coord \\
      )} & Devuelve un píxel \textit{vec4 rgba} en la coordenada noormalizada \textit{coord}. \\
      \pbox{10cm}{
      textureLod(\\
      \tab[1cm]sampler2D iChannelN,\\
      \tab[1cm]vec2 coord, \\
      \tab[1cm]float lod, \\
      )} & Devuelve un píxel \textit{vec4 rgba} en la coordenada noormalizada \textit{coord} con un nivel de detalle \textit{lod}. \\
      \bottomrule
    \end{tabularx}
\end{table}


	% Marcher
	% https://adrianb.io/2016/10/01/raymarching.html#introduction-to-raymarching
\chapter{Marcher}
% https://books.google.es/books?id=MNqRDwAAQBAJ&pg=PA13&dq=sphere+ray+marcher+graphics&hl=es&sa=X&ved=0ahUKEwj1sqWTgZvrAhUCxhoKHRwBCVIQ6AEIJzAA#v=onepage&q=sphere%20ray%20marcher%20graphics&f=false
Gracias a los avances tecnológicos y al incremento de la potencia de los micro-procesadores, a esto, unido la paralelización de tareas y dispositivos hardwares especializados como la \textit{GPU}, ha permitido que se pueda proponer esta técnica. Cada píxel de nuestra pantalla es calculado por una \textit{hebra} de la GPU, una especie de micro-procesador que trabaja de manera individual, sin memoria y sin intercomunicación con las demás. Todas ejecutan el mismo código, contiene información de la pantalla como la coordenada del pixel que está calculado, la resolución, etc. y devuelve un valor \textit{rgba}, en formato \textit{vec4}.\\\\
Existen técnicas para proyectar una escena en nuestras pantallas. Dentro de esta categoría de técnicas, encontramos las técnicas basadas en "rayos" (\textit{rays}). Para cada pixel de nuestra pantalla, "lanzaremos un rayo" desde un ojo, que es el nombre que recibe el origen de la cámara, en dirección al píxel, suponiendo que nuestra pantalla está en la escena. Si este rayo intersecta sobre una superficie, entonces podemos dibujarlo.\\\\
\begin{figure}[H]
  \centering
  \captionsetup{justification=centering}
  \includegraphics[width=1.0\textwidth]{secciones/imagenes/gpu.png}\label{fig:marcher}
  \caption{Pararelismo de la GPU en dibujado mediante rayos.}
\end{figure}
Definimos lanzar un rayo como aproximar con un vector o utilizar una recta para trazar una escena que encontramos desde el \textit{ojo} en dirección al píxel. Si utilizamos una recta, tratamos de la técnica \textit{raytracing}, que calcula de forma exacta la intersección con los objetos de la escena. Mientras si utilzamos un vector, trataremos de \text{raymarching}, que \textbf{aproxima} la intersección con la escena utilizando una modulación incremental del vector director.\\\\
Computacionalmente, aproximar, suele ser menos costoso que calcular de forma exacta el resultado. Por ello, vamos a utilizar la técnica de \textit{Raymarching}, para aproximar la escena y las \textit{funciones de distancia con signo} para crearla, esta técnica recibe el nombre de \textit{Spheremarching}, que se le ha atribuido el reconocimiento a John Hart en 1996 aunque  se cree que ya otros autores habían investigado acerca de la técnica presentada. \\\\
Utilizaremos el vector director del \textit{ojo} hacia el píxel, incrementando el módulo según el valor de distancia desde la cabeza del vector, de manera recursiva con un número de iteraciones máximas. Recibe el prefijo \enquote{\textit{Sphere}\textendash} ya que, por definición, podemos generar una esfera de radio \(r\), sobre la cabeza del vector, con valor de la función de distancia \(r=sdf(x,y,z)\). Tal que, la esfera, no contine ningún punto en su interior.\\\\
Para fijar el número de iteraciones, se recomienda utilizar una potencia de \(2\). Cuanto mayor sea este número, mejor será la aproximación, aunque, mayor el gasto computacional. No existe un valor por defecto, este dependerá de la escena utilizada y será elegido de forma empírica.\\
Vamos a definir las \textbf{condiciones de parada} de este algoritmo.
\begin{enumerate}
    \item Estar cerca de la isosuperficie. Como el radio de la esfera, indica la distancia a la isosuperfice más cercana, utilizaremos un umbral \(\epsilon\), muy pequeño, para definir la distancia a la que consideramos isosuperficie, en un modelo exacto, \(\epsilon=0.0\).
    \item Superar una cierta distancia recorrida, esta distancia actuará como plano trasero, en caso de superarlo, devolveremos la distancia impuesta para dicho plano. 
    \item Superar el número de iteraciones máximas, en caso afirmativo, devolveremos la distancia al plano trasero.
\end{enumerate}
 En los casos en los que se devuelve la distancia al panel trasero, recibirán el nombre de "\textit{fallo}", que representa un pixel que no ha podido trazar una isosuperficie. Pudiéndose considerar, el fondo de la escena, existen algunos casos, en los que encerraremos nuestra escena en una isosuperficie que actuará como fondo.
\begin{figure}[H]
  \centering
  \captionsetup{justification=centering}%,margin=2cm
  \includegraphics[width=1.0\textwidth]{secciones/imagenes/raymarching.png}\label{fig:spheremarcher}
  \caption{Ejemplo del algoritmo \textit{Spheremarching}}
\end{figure}
\newpage
\begin{lstlisting}
#define PASOS 128 // Número Máximo de Iteraciones.
#define EPSILON 0.001
#define MAXIMO 20.0 // Distancia del Plano Trasero.

// Pasamos el origen del ojo y la dirección, nos devuelve la distancia al objeto más cercano del ojo en dicha dirección.
float SphereMarching(
    vec3 ojo, 
    vec3 direccion
){
    float distancia = 0.0;
    // Realizamos "PASOS" iteraciones de marching.
    for(int i = 0; i < PASOS; ++i){
        // Calculamos el vector (rayo).
        vec3 rayo = ojo + direccion * distancia;
        // Aproximamos el radio de la esfera más próxima a una isosuperficie
        float radio = escena_sdf(rayo);
        // Si el radio (distancia mínima a la isosuperficie), es muy pequeña, podemos decir que estamos sobre la distancia y devolvemos el módulo del rayo.
        if(radio < EPSILON){
            return distancia;
        }
        // Incrementamos la distancia recorrida si no estamos cerca de la isosuperficie.
        distancia += radio;
        // Comprobamos que no se haya superado la distancia de dibujado máximo. Podemos considerarlos el fondo de la escena.
        if(distancia >= MAXIMO) break;
        return MAXIMO;
    }
}
\end{lstlisting}
\newpage
Como ya se ha comentado antes, este algoritmo es aplicado para cada pixel de la pantalla, si el valor devuelto por el \textit{Marcher} es distinto a un \textit{fallo}, sabremos que estamos muy próximos a una isosuperficie y así, dibujarlo. En caso contrario, lo consideraremos fondo. Con la distancia devuelta, podemos posicionarlo en nuestra escena ya que el valor devuelto es la distancia aproximada a la isosuperficie, por lo que, el vector a dicho punto, aproximado, será:
\[ \Vec{p} = \Vec{ojo} + distancia \cdot  \Vec{direccion} \]
Como ya hemos comentado, \textit{Shadertoy} nos ofrece un entorno para trabajar, además de un pequeño código de ejemplo, el cual modificaremos para adaptarlo al \textit{Marcher}.
\begin{lstlisting}
void mainImage( out vec4 fragColor, in vec2 fragCoord )
{
    // Normalized pixel coordinates (from 0 to 1)
    vec2 uv = fragCoord/iResolution.xy;

    // Time varying pixel color
    vec3 col = 0.5 + 0.5*cos(iTime+uv.xyx+vec3(0,2,4));

    // Output to screen
    fragColor = vec4(col,1.0);
}
\end{lstlisting}
Podemos observar tres variables importantes, que se han comentado anteriormente.
\begin{enumerate}
    \item \textit{fragColor}. Se trata de una variable de \textbf{salida}, es el valor del pixel, es decir, una 4-upla \textit{rgba} de tipo \textit{vec4}. Cada elemento está en el intervalo \([0,1]\).
    \item \textit{fragCoord}. Es una variable de \textbf{entrada}, la coordenada del pixel en pantalla, donde la primera componente representa la coordenada \(x\) y la segunda, la \(y\).
    \item \textit{iResolution}. Contiene el ancho y el alto de la pantalla.
\end{enumerate}
La pantalla en realidad nos referimos al donde se dibuja el resultado.
\newpage
\begin{lstlisting}
void mainImage(
    out vec4 fragColor, 
    in vec2 fragCoord
){
    // Normalizamos las coordendas y las reescalamos para mantener el ratio de aspecto. Transladamos al centro de la pantalla.
    vec2 uv = (fragCoord - iResolution.xy * 0.5) / min(iResolution.y, iResolution.x);
    // Definimos el ojo y la pantalla, que se encuentra en nuestra escena.
    vec3 ojo = vec3(0.0, 0.0, -1.0);
    vec3 pantalla = vec3(uv, 0.0);
    // La dirección del rayo es el vector normalizado que apunta desde el ojo hasta la pantalla (píxel).
    vec3 direccion = normalize(pantalla-ojo);
    // Con esto, ya podemos utilizar nuestro Sphere marcher.
    float distancia = SphereMarching(ojo, direccion);
    // El marcher nos ha devuelto una distancia inferior al plano trasero, estamos sobre la isosuperficie.
    if(distancia < MAXIMO){
        // Estamos aproximadamente sobre la isosuperficie.
        // La posición aproximada es la siguiente.
        vec3 p = ojo + direccion * distancia;
        // Utilizamos el color blanco para dibujar la isosuperficie.
        fragColor = vec4(1.0);
    }else{ // El marcher ha fallado.
        // El color negro para pintar el fondo.
        fragColor = vec4(vec3(0.0), 1.0);
    }
}
\end{lstlisting}
\newpage
Si intentáramos ejecutar este código, veríamos que no compilaría, esto es debido a que no hemos definido aún nuestra escena. La función \textit{escena\_sdf} que se encuentra dentro de la función \textit{SphereMarching}, contiene nuestra escena como una \textit{Funcíon de distancia con Signo}, vamos a definir la escena más simple, aunque en capítulos posteriores, veremos formas de crear nuestras propias escenas y funciones de distancia con signo.
\begin{lstlisting}
/* 
No vamos a entrar, aún, en como se define una escena mediante Funciones de Distancia con Signo.
Aunque el siguiente código, representa:
Una esfera en el la coordenada (0,0,0) de radio 0.2 unidades.
*/
float escena_sdf(vec3 p){
    return length(p - vec3(0.0)) - 0.2;
}
\end{lstlisting}
Veamos una pequeña pincelada de como se ha definido esta función. Se calcula el módulo del punto \(\Vec{p}\), esto define una \textit{Función de Distancia con Signo} positiva y cuya isosuperficie es únicamente un punto, \(S=\{(0,0,0)\}\).\\Es fácil observar que si restamos \(r\) al la distancia, estamos creando una isosuperficie esférica de radio \(r\). Aquellos puntos cuya distancias valen \(r\), acabarán anulándose y definiendo la \textit{isosuperficie}. Las distancias inferiores, tomarán valores negativos y los superiores, positivos.
\[S=\{\Vec{q} \in \mathbb{R}^3 / SDFEsfera_r(\Vec{q})=0\}\]
\[ SDFEsfera_r(\Vec{p})=\vert\vert\Vec{p}\vert\vert - r  \]
\begin{figure}[H]
  \centering
  \captionsetup{justification=centering}%,margin=2cm
  \includegraphics[width=1.0\textwidth]{secciones/imagenes/starting/sdf1.png}\label{fig:hello}
  \caption{"Hola mundo" del algoritmo \textit{SDF}.}
\end{figure}
Al solo utilizar dos colores, blanco y negro, no tenemos sensación de profundidad, esto se conseguirá utilizando un modelo de iluminación, que incluyen luces y sombras.
	% Iluminacion
	\chapter{Modelo de iluminación}
\section{Introducción}
En este capítulo vamos a ver los principios de los modelos de iluminación, así como operadores importantes. Seguidamente presentaremos el modelo de iluminación Phong, que es y ha sido utilizado desde los años 70.\\\\Un modelo de iluminación es esencial para el diseño artístico de la escena, simular propiedades físicas como por ejemplo, reflejos, refracción, etc. Además, las sombras dan sensación de profundidad a una escena.\\\\
En este capítulo hablaremos de dos elementos importantes, luces y sombras. Aunque no lo parezca, todos ellos hacen uso de una propiedad fundamental de las superficies, el vector normal. Vamos a crear el algoritmo definidio en el apatartado \textit{Preliminares}.
\begin{lstlisting}
// Cálculo de la normal de la isosuperficie estimado por un rayo.
vec3 Normal(vec3 p){
     // f(x1,...,xn)
     float fxyz = escena_sdf(p);
     // f(x1,..,xi+h,xn)
     float fxhyz = escena_sdf(p + vec3(EPSILON, 0.0, 0.0));
     float fxyhz = escena_sdf(p + vec3(0.0, EPSILON, 0.0));
     float fxyzh = escena_sdf(p + vec3(0.0, 0.0, EPSILON));
     // Utilizamos la definicion de derivadas parciales para devolver el gradiente, que se trata de la normal de la isosuperficie, como hemos definido en los Preliminares.
     return vec3(
         (fxhyz - fxyz) / EPSILON,
         (fxyhz - fxyz) / EPSILON,
         (fxyzh - fxyz) / EPSILON
     );
}
\end{lstlisting}
\newpage
Vamos a ahora a presentar el \textit{producto escalar}, implementado de forma nativa en \textit{GLSL} con el nombre de "\textit{dot}", esencial para los modelos de iluminación.
\[\Vec{r} \cdot  \Vec{v} = r_xv_x + r_yv_y + r_zv_z = \vert r\vert\vert v\vert\cos(\alpha)\]
Si ambos son vectores directores, es decir, normalizados y en el origen, resulta \(\Vec{r} \cdot \Vec{v} = \cos(\alpha)\). El valor \(\alpha\) es el ángulo entre los dos vectores sobre el plano que forman, en caso de \(\mathbb{R}^2\), la componente \(z\) sería nula. La imagen del operador es el intervalo \([-1,1]\)\\Veamos alguna de las propiedades, si ambos vectores son perpendiculares, con \(\alpha=\pm\dfrac{\pi}{2}\), el \textit{producto escalar} será \(\Vec{r}\cdot\Vec{v}=\cos\left(\pm\dfrac{\pi}{2}\right)=0\). En el caso en el que sean son paralelos, \(\alpha=\{0,\pi\}\), el resultado será  \(\Vec{r}\cdot\Vec{v}=\cos(\{0, \pi\})=\pm 1\), según la dirección de ambos.\\\\ El lenguaje \textit{GLSL} presenta dos operaciones vectoriales que utilizaremos en modelo, estas son.
%https://es.m.wikipedia.org/wiki/Ley_de_Snell
\begin{table}[h]
    \begin{tabularx}{\textwidth}{l|X}
        \toprule
        Función & Definición\\
        \midrule
        \pbox{10cm}{
          reflect(\\
          \tab[1cm]vecN a,\\
          \tab[1cm]vecN n, \\
          )} & El vector \(\Vec{n}\) debe estar normalizado, este operador devuelve el vector \(\Vec{a}\) reflectado respecto de \(\Vec{n}\),
        \[\Vec{r}=\Vec{a} - 2(\Vec{n} \cdot \Vec{a})\Vec{n}\]
        \begin{minipage}{1.0\textwidth}
          \centering
          \includegraphics[width=.25\textwidth]{secciones/imagenes/reflect.jpeg}
        \end{minipage}
        \\
        \pbox{10cm}{
        refract(\\
          \tab[1cm]vecN a,\\
          \tab[1cm]vecN n, \\
          \tab[1cm]float k, \\
          )} & El vector \(\Vec{n}\) debe estar normalizado, este operador devuelve el vector \(\Vec{a}\) refractado respecto de \(\Vec{n}\), con \(k\) como factor de medio. Según la \textit{ley de refracción de Snell-Descartes}.
        \[\Vec{r}=k\left(\Vec{a} - \left(\left(\Vec{n} \cdot \Vec{a}\right)+\sqrt{\dfrac{1}{k^2}-(\Vec{a}\cdot\Vec{n})^2}\right)\Vec{n}\right)\]
        \begin{minipage}{1.0\textwidth}
          \centering
          \includegraphics[width=.3\textwidth]{secciones/imagenes/refract.jpeg}
        \end{minipage}\\
        \bottomrule
    \end{tabularx}
\end{table}
\newpage
\section{Luz e Intensidad}
En este apartado veremos la definición de \textit{Intensidad lumínica}, así como dos tipos de luces existentes, luz direccional y radial.\\\\
 Definimos la \textit{intensidad lumínica} en un punto como un factor multiplicativo al material asignado al punto de la \textit{isosuperficie}, representa cómo de iluminado está. Como es un factor multiplicativo, el valor de \(0.0\), representa la intensidad nula u oscuridad. Mientras que el valor \(1.0\) representa el valor más iluminado.\\\\ 
El operador \textit{producto escalar} nos debería dar una breve intuición del papel importante que juega en el cálculo de la intensidad. Como esta no puede ser negativa, definimos el operador producto escalar positivo y normalizado "\(\cdot_{[0,1]}\)".
\[\cdot_{[0,1]}:\mathbb{R}^2\times\mathbb{R}^2\longrightarrow[0,1] : \Vec{a}\cdot_{[0,1]}\Vec{b}=\max\left(\dfrac{\Vec{a}\cdot \Vec{b}}{\vert\vert\Vec{a}\vert\vert\vert\vert \Vec{b}\vert\vert}, 0\right)\]
Vamos a utilizar el "\textit{Modelo de Iluminación de Phong}", presentado en 1973 por \textit{Tuong Phong} como un modelo de iluminación empírico. Para ello, vamos a ver como el modelo se descompone en tres etapas. La primera, el cálculo de la \textbf{Intensidad Ambiente}, que se trata de un valor \(I_a \in [0,1]\) que indica cuanto de iluminada está la isosuperficie, de manera inicial o si no hubieran luces. 
\begin{figure}[H]
  \centering
  \captionsetup{justification=centering}%,margin=2cm
  \includegraphics[width=0.8\textwidth]{secciones/imagenes/lightmodel/ambiental.png}\label{fig:ambient}
  \caption{Intensidad Ambiental sobre la esfera.}
\end{figure}
Por otro lado, la \textbf{Intensidad Especular}, que es aportada de manera colectiva sobre el punto aproximado \(\Vec{p}\) de la \textit{isosuperficie}, para cada una de las luces de la escena \(\Vec{l_i}\in L\), donde \(\Vec{l_i}\) representa la posición de la luz, se comprueba como incide la luz sobre la superficie, con respecto de su normal. 
\[I_d = \sum_{\Vec{l_i}\in L} \Vec{n}\cdot_{[0, 1]}(\Vec{l_i}-\Vec{p})\]
Donde \(\Vec{n}\) es la normal de la \textit{isosuperficie} en el punto \(\Vec{p}\). Es fácil observar que, la intensidad debería ser máxima cuando los rayos inciden en de manera paralela en sentido contrario al vector normal y nulo en caso de que sean perpendiculares u opuestos.
\begin{figure}[H]
  \centering
  \captionsetup{justification=centering}%,margin=2cm
  \includegraphics[width=0.8\textwidth]{secciones/imagenes/lightmodel/difusa.png}\label{fig:difusse}
  \caption{Intensidad Difusa sobre la esfera.}
\end{figure}
Veamos finalmente la \textbf{Intensidad Especular}, esta intensidad indica como incide la luz reflectada  por la \textit{isosuperficie} en el la dirección del ojo.\\
Definimos el operador de reflexión "\(\veebar\)"\footnote{La demostración la podemos encontrar en...}, de un vector \(\Vec{a}\) sobre un vector director \(\Vec{n}\).
\[\Vec{a}\veebar\Vec{n}=\Vec{a} - 2(\Vec{n} \cdot \Vec{a})\Vec{n}\]
La ecuación final de la "\textit{Intensidad Especular}" para todas las luces de la escena es,
\[I_e = \sum_{\Vec{l_i}\in L} \Vec{ojo}\cdot_{[0, 1]}\left(\left(\Vec{l_i}-\Vec{p}\right) \veebar \Vec{n}\right)\]
Algunos autores aportan una leve modificación de esta ecuación, aplicando un \textit{homomorfismo} polinómico con grado exponencial.
\[h_k:[0,1]\longrightarrow[0,1] , h_k(x)=x^{2^k}\]
\[I_d = \sum_{\Vec{l_i}\in L} h_k\left(\Vec{ojo}\cdot_{[0, 1]}\left(\left(\Vec{l_i}-\Vec{p}\right) \veebar \Vec{n}\right)\right)\]
Donde \(k\in\mathbb{R}^{+}\) y este tiene efecto sobre el radio de rayos reflejados.
\begin{figure}[H]
  \centering
  \captionsetup{justification=centering}%,margin=2cm
  \subfloat[Intensidad especular con \(h_0\)]{\includegraphics[width=0.33\textwidth]{secciones/imagenes/lightmodel/especular-0.png}\label{fig:specular-0}}
  \subfloat[Intensidad especular con \(h_1\)]{\includegraphics[width=0.33\textwidth]{secciones/imagenes/lightmodel/especular-1.png}\label{fig:specular-1}}
  \subfloat[Intensidad especular con \(h_3\)]{\includegraphics[width=0.33\textwidth]{secciones/imagenes/lightmodel/especular-2.png}\label{fig:specular-2}}
  \caption{Intensidad Difusa con distintos homomorfismos}
\end{figure}
El modelo final, definido por la \textit{Intesidad del modelo de Phong} se calcula como la suma de las intensidades expuestas anteriormente.
\[I_{Phong}=I_a+\sum_{\Vec{l_i}\in L} \mathrlap{\underbrace{\phantom{\Vec{n}\cdot_{[0, 1]}(\Vec{l_i}-\Vec{p})}}_{\text{Intensidad Difusa}}}\Vec{n}\cdot_{[0, 1]}(\Vec{l_i}-\Vec{p}) + \mathrlap{\underbrace{\phantom{h_k\left(\Vec{ojo}\cdot_{[0, 1]}\left(\left(\Vec{l_i}-\Vec{p}\right) \veebar \Vec{n}\right)\right)}}_{\text{Intensidad Especular}}}h_k\left(\Vec{ojo}\cdot_{[0, 1]}\left(\left(\Vec{l_i}-\Vec{p}\right) \veebar \Vec{n}\right)\right)\]
Como hemos dicho antes, este es un factor, multiplicativo, en particular, multiplica al valor del material o en este caso, el color \textit{rgba} devuelto que era, el blanco.
\begin{figure}[H]
  \centering
  \captionsetup{justification=centering}%,margin=2cm
  \includegraphics[width=0.8\textwidth]{secciones/imagenes/lightmodel/phong.png}\label{fig:phong}
  \caption{Intensidad Phong sobre la esfera con \(h_3\).}
\end{figure}
En algunas implementaciones, se utiliza para cada luz, un factor de atenuación que depende de la distancia de la superficie a la luz, esta función converge a cero en el infinito. En particular, OpenGL\footnote{A partir de la version X.X empezó...} utiliza la siguiente función:
\[p(x)=\dfrac{1}{ax^2+bx+c}\]
donde \(a,b,c \in \mathbb{R}^{+}_{0}\) son factores que dan una riqueza artística al modelo de iluminación, así como el \textit{homomorfismo} utilizado en la \textit{Intensidad Especular}.\\\\En luces direccionales, el punto se considera estar en el infinito, sustituyéndose \(\Vec{l_i}-\Vec{p}\) por el vector director de la luz \(\Vec{d_i}\). En caso de utilizar una función de atenuación, el valor que tomaría  será cero, por lo que la \textit{Intensidad Especular} quedaría anulada. Además, para la \textit{Intensidad Difusa}, utilizaremos el vector director de la luz direccional en vez de la diferencia del punto aproximado \(\Vec{p}\) y la posición de la luz \(\Vec{l_i}\).
\newpage
Veamos un ejemplo práctico en código.
\begin{lstlisting}
// Homomorfismo
float h3(float h){return pow(h,pow(2.,3.));}
// Definimos el operador producto escalar normalizado positivo.
float dot01(vec3 a, vec3 b){ 
    return max(dot(a,b)/(normalize(a)*normalize(b)), 0.0);
}
// Modelo de iluminación Phong
float ModeloIluminacion(vec3 direccion, vec3 p){
    // Calculamos la normal del punto.
    vec3 normal = Normal(p);
    // Ayuda al marcher a escapar de la isosuperficie
    p = p + normal * 0.1;
    // Modelo
    float intensidad = 0.0;
    // Intensidad Ambiente Global
    intensidad += 0.2;
    // Intensidad de cada Luz
    // Luz 1.
    vec3 posicion_luz_1 = vec3(2., 4., 1.);
    vec3 d_luz_1 = posicion_luz_1 - p;
    float dst_luz_1 = length(d_luz_1);
    // Intensidad Difusa
    intensidad += dot01(d_luz_1, normal);
    // Intensidad Especular, en caso de ser una luz direccional, podemos ignorar esta componente ya que la posición es considerada estar en el infinito y por ello, f_difusa = 0
    vec3 r_luz_1 = reflect(d_luz_1, normal);
    intensidad += f_difusa(dst_luz_1) * h3(dot01(r_luz_1, direccion));
    // ... Utilizamos el mismo esquema para las demás luces.
    // Devolvemos la intensidad en el rango [0, 1].
    return clamp(intensidad, 0.0, 1.0);
}
\end{lstlisting}
\newpage
\section{Sombras}
Vamos a ver la técnica más sencilla para calcular las sombras, pero es importante mencionar que hablaremos únicamente de la \textit{umbra} de una la sombra. La \textit{umbra} sucede cuando la fuente de luz es ocluida completamente por una superficie. Haciendo que la luz no actue sobre la intensidad del punto.\\\\Una vez aproximado un punto \(p\) de la \textit{isosuperficie}, diremos que está en \textit{umbra}, si es ocluido por un objeto en dirección a la luz, o lo que es lo mismo, podemos utilizar el marcher desde \(\Vec{p}\) en dirección a la luz \(\Vec{l_i}\), cuyo plano trasero contiene a \(\Vec{l_i}\). Si este traza otro punto \(\Vec{q}\) en esa dirección, este estará en \textit{umbra}.\\\\
Vamos a realizar una pequeña modificación sobre el vector \(\Vec{p}\) que ayudará a agilizar el \textit{Marcher}, ya que las primeras iteraciones de este, intentará "escapar" de la isosuperficie, para ello, vamos a empujarlo de la superficie, utilizando la normal.
\[\Vec{p'}=\Vec{p} + \Vec{n} \cdot k\]
Donde \(k\in\mathbb{R}^{+}_{0}\) y funciona como un factor de empuje de la superficie, se trata de un valor empírico que ayuda al marcher a salir de la isosuperficie, ya que en las primeras iteraciones, el radio de las esferas está muy próximo a \(0,0\). \\\\
% Imagen Desplazamiento 
Se ha realizado además una leve modificación del marcher, ahora este aceptará un tercer argumento, que indica la distancia del plano trasero, que anteriormente estaba fijado por el plano trasero \textit{MAXIMO}. Esto nos será útil para parar el marcher cuando hemos trazado la distancia la distancia a la luz.\\\\
Vamos a crear un modelo de iluminación con \textbf{dos} luces, una radial y otra direccional. Además, utilizaremos un plano en donde proyectar la sombra. La luz direccional es perpendicular al plano.
\begin{figure}[H]
  \centering
  \captionsetup{justification=centering}%,margin=2cm
  \includegraphics[width=0.8\textwidth]{secciones/imagenes/lightmodel/sombra_dura.png}\label{fig:shadow}
  \caption{Modelo de iluminación y sombras sobre la escena definida.}
\end{figure}
\newpage
\begin{lstlisting}
// Se ha añadido un tercer argumento.
float SphereMarching(in vec3 ojo, in vec3 direccion, float distancia_plano){
    float distancia = 0.0;
    for(int i = 0; i < PASOS; ++i){
        vec3 rayo = ojo + direccion * distancia;
        float radio = escena_sdf(rayo);
        if(radio < EPSILON){
            return distancia;
        }
        distancia += radio;
        // Ahora depende del tercer argumento
        if(distancia > distancia_plano)break;
    }
    return distancia_plano;
}
// Phong + Sombras duras
float ModeloIluminacion(vec3 direccion, vec3 p){
    // Calculamos la normal del punto.
    vec3 normal = Normal(p);
    // Ayuda al marcher a escapar de la isosuperficie
    p = p + normal * 0.1;
    // Modelo
    float intensidad = 0.0;
    // Intensidad Ambiente Global
    intensidad += 0.2;
    // Luz 1.
    vec3 posicion_luz_1 = vec3(2., 4., 1.);
    vec3 d_luz_1 = posicion_luz_1 - p;
    vec3 dir_luz_1 = normalize(d_luz_1);
    float dst_luz_1 = length(d_luz_1);
    // En el caso de que se trate de una luz direccional, utilizaremos el plano MAXIMO, utilizado antes.
    if(SphereMarching(pd, dir_luz_1, dst_luz_1) >= dst_luz_1){
        // Intensidad Difusa
        intensidad += dot01(d_luz_1, normal);
        // Intensidad Especular (Si no es direccional)
        vec3 r_luz_1 = reflect(d_luz_1, normal);
        intensidad += f_difusa(dst_luz_1) * h3(dot01(r_luz_1, direccion));
    }
    // ... Repetimos el esquema anterior.
    return clamp(intensidad, 0.0, 1.0);
}
\end{lstlisting}
%https://www.shadertoy.com/view/wtfBW8
\newpage

    % Funciones de Distancia con Signo
	\chapter{Funciones de Distancia con Signo}
\section{Introducción}
Hemos presentado anteriomente las funciones de distancia con signo como definición matemática, pero ahora debemos cuestionarnos, ¿Cómo podemos definir funciones de distancia con signo propias?, ¿Qué operadores existen?

%https://www.iquilezles.org/www/articles/distfunctions/distfunctions.htm
\begin{definition}
Una función de distancia con signo, se dice exacta si está definido sobre la métrica euclídea.
\end{definition}

Vamos a definir algunas \textit{funciones de distancia con signo} \textbf{primitivas} sobre \(\mathbb{R}^2\) y posteriormente sobre \(\mathbb{R}^3\). Las definiremos centradas en el \((0,0)\) y en el \((0,0,0)\), respectivamente. Que posteriormente podremos trasladarlo o rotarlo utilizando operadores.
%https://www.iquilezles.org/www/articles/distfunctions2d/distfunctions2d.htm
\section{Primitivas sobre \(\mathbb{R}^2\)}
Estas primitivas son fundamentales para finalmente contruir distintas funciones sobre \(\mathbb{R}^3\) ya que muchas de estas primitivas, son facilmente generalizables o se utilizan junto a un operador como extrusión o revolución figuras sobre.\\\\
Para visualizar estas funciones, vamos a hacer uso también de \textit{Shadertoy} y vamos a modificar levemente el código por defecto. Utilizaremos el color rojo para indicar el interior de la figura, el color azul para el exterior y el color blanco para indicar que está próximo a una isosuperficie, utilizando un valor empírico \(\epsilon=0,01\).
\begin{lstlisting}
#define EPSILON 0.01
void mainImage( out vec4 fragColor, in vec2 fragCoord )
{
    vec2 p = (fragCoord-iResolution.xy * 0.5)/min(iResolution.x, iResolution.y);
    // Aplicamos la funcion de distancia con signo sdf junto con los parámetros que lo definen.
    float d = ...;
    
    vec3 col;
    if(abs(d) < EPSILON){
        col = vec3(1.0);
    }else{
        if(d < 0.0){
    		col = vec3(1.0, 0., 0.);
    	}else{
    		col = vec3(0., 0., 1.0);
    	}
    	// Número de repeticiones.
        float k = 10.0;
        col = col * (0.5 + 0.5 * (fract(abs(d) * k)));
    }
    fragColor = vec4(col, 1.);
}
\end{lstlisting}

\subsection{Circunsferencia exacta}
La definición de esta figura es muy simple, la distancia desde cualquier punto \(\Vec{p}\) hasta \((0,0)\) está definido por el módulo del vector \(\vert\vert\Vec{p}\vert\vert\), esto crea una función de distancia positiva con un isoperímetro \(S=\{(0,0)\}\), si le restamos el radio \(r\), conseguiremos lo querido, anular aquellos puntos a distancia \(r\). Cuando el vector está en el interior de la circusnferencia (el módulo del vector es inferior al radio), \(\vert\vert\Vec{p}\vert\vert < r \longrightarrow \vert\vert\Vec{p}\vert\vert - r=d < 0\) siendo la distancia negativa o el interior de la figura. En caso de que el módulo del vector coincide con el radio, \(\vert\vert\Vec{p}\vert\vert = r \longrightarrow \vert\vert\Vec{p}\vert\vert - r=d = 0\), es decir, estamos sobre el isoperímetro. Mientras que si el módulo es mayor, \(\vert\vert\Vec{p}\vert\vert > r \longrightarrow \vert\vert\Vec{p}\vert\vert - r=d > 0\) la distancia será positiva en el exterior.
\begin{lstlisting}
// Circunsferencia Exacta
float SDFCircunsferencia(vec2 p, float r){
    return length(p) - r;
}
\end{lstlisting}
\begin{figure}[H]
  \centering
  \captionsetup{justification=centering}%,margin=2cm
  \includegraphics[width=0.8\textwidth]{secciones/imagenes/sdf_circunsferencia.jpeg}\label{fig:circ}
  \caption{Circunsferencia FDS radio \(0.3\)}
\end{figure}

\subsection{Rectángulo exacto}
Para calcular de forma exacta esta función vamos a definir algunos operadores sobre los vectores, que están definidos en GLSL.
\[\vert(x_0,x_1,\dots,x_n)\vert=(\vert x_0\vert,\vert x_1\vert,\dots,\vert x_n \vert)\]
\[\max\left((x_0,x_1,\dots,x_n), k\right)=(\max( x_0, k), \max(x_1, k),\dots, \max(x_n, k))\]
El operador valor absoluto es muy útil ya que, en caso de existir simetría axial en x e y, que lo hay, podemos reducir nuestro problema a un único cuadrante. Debemos tener en cuenta que la las medidas originales son reducidas a la mitad ya que solo estamos observando el primer cuadrante que representa \(\dfrac{1}{4}\) de la figura. Estas medidas son expresadas de manera vectorial tales que \[\Vec{s'}=(w', h')=\left(\dfrac{w}{2},\dfrac{h}{2}\right)=\dfrac{\Vec{s}}{2}\]
Una vez hecho esto, vamos a reducir el problema en 4 regiones.
\begin{figure}[H]
  \centering
  \captionsetup{justification=centering}%,margin=2cm
  \includegraphics[width=0.8\textwidth]{secciones/imagenes/sdf_rect_calc.jpeg}\label{fig:subproblem}
  \caption{División del rectángulo en 4 regiones o subproblemas.}
\end{figure}
Vamos a preservar la métrica euclídea suponiendo que se ha posicionado el punto \(\Vec{p}=\vert(x,y)\vert=(\vert x\vert, \vert y \vert)\) en cada una de las regiones.
\begin{enumerate}
    \item \textbf{Región 1}. Se trata de encontrar la distancia al lado derecho, esto es sencillo, bastaría con hacer \(\nabla x=\vert x\vert-w'\).
    \item \textbf{Región 2}. De manera similar al anterior, la distancia al lado superior, será \(\nabla y=\vert y\vert-h'\).
    \item \textbf{Región 3}. En esta región vamos a calcular la distancia a la esquina del rectángulo, \(\vert\vert \vert\Vec{p}\vert-\Vec{s'}\vert\vert = \sqrt{\left(\vert x\vert-w'\right)^2+\left(\vert y\vert-h'\right)^2}\).
    \item \textbf{Región 4}. Esta región puede considerarse la más difícil de calcular, debemos utilizar el mismo cálculo que el utilizado para los lados, pero nos quedaremos con la distancia máxima del cálculo, ya que esta es negativa, al estar en el interior de la figura. Deberá anularse la componente cuanto este cálculo sea positivo, es decir, esté en la \textbf{Región 1, 2}.
    \[argmax(\Vec{a})=min(max(\Vec{a}_x, \Vec{a}_y), 0)\]
    Por lo que la distancia interior, será \(argmax(\vert\Vec{p}\vert-\Vec{s'})\).
\end{enumerate}

Podríamos crear una función a trozos, definida mediante diferentes \textit{ifs} para cada una de las regiones, pero queremos que además de ser exacta, sea eficiente, por lo que vamos a mirar en que relación tienen las diferentes regiones según la posición del punto. Es importante observar que la \textbf{Región 3} contiene en su ecuación a las \textbf{Regiónes 1,2}, por lo que deben existir casos particulares. Cuando \(\nabla \vert x\vert =\vert x\vert-w'=0\longrightarrow \sqrt{\left(\vert y\vert-h'\right)^2} = \vert y\vert-h'=\nabla \vert y\vert\), que concuerda con la \textbf{Región 2} en la recta \(y=h'\) pero cuando \(\nabla \vert x\vert < 0\), vemos que la componente \(y\) no queda isolada, para que esto ocurra, haremos \(\max\left(\nabla \vert x\vert, 0\right)\), haciendo que la \textbf{Región 2 y 3} queden unificadas. De manera equivalente para la \textbf{Región 1 y 3}, tenemos \(\max\left(\nabla \vert y\vert, 0\right)\), esto hace que la ecuación unificada para las \textbf{Regiones 1, 2 y 3} sea la siguiente,
\[SDFRectangulo_{\Vec{s'}}(\Vec{p})\approx \vert\vert\max\left(\vert\Vec{p}\vert-\Vec{s'},0\right)\vert\vert\]
Finalmente, no es difícil observar que \(\nabla\vert x\vert=\nabla\vert y\vert=0\) cuando nos encontramos en la \textbf{Región 4}, por lo que bastará sumar la distancia de esta región para obtener la ecuación exacta.
\[SDFRectangulo_{\Vec{s'}}(\Vec{p})= \vert\vert\max\left(\vert\Vec{p}\vert-\Vec{s'},0\right)\vert\vert + argmax(\vert \Vec{p}\vert - {s'})\]
\begin{lstlisting}
// Rectángulo Exacto
float SDFRectangulo(vec2 p, vec2 s){
    vec2 a = abs(p) - s;
    return length(max(a, 0.0)) + min(max(a.x, a.y), 0.0);
}
\end{lstlisting}
\begin{figure}[H]
  \centering
  \captionsetup{justification=centering}%,margin=2cm
  \includegraphics[width=0.8\textwidth]{secciones/imagenes/sdf_rectangulo.jpeg}\label{fig:rectaangulo}
  \caption{Rectángulo FDS de dimensiones \(\Vec{s'}=(0.4, 0.2)\)}
\end{figure}

\subsection{Recta exacta}
Veamos la distancia a una recta que pasa por dos puntos \(\Vec{a},\Vec{b}\). Pero antes, definimos la proyección escalar de \(\Vec{a}\) sobre \(\Vec{b}\) tal que:
\[ \text{proy}_{\Vec{b}}\Vec{a}=\left(\dfrac{\Vec{a}\cdot\Vec{b}}{\vert \Vec{b}\vert}\right)\dfrac{\Vec{b}}{\vert\Vec{b}\vert}=\left(\dfrac{\Vec{a}\cdot\Vec{b}}{\vert \Vec{b}\vert^2}\right)\Vec{b}=\left(\dfrac{\Vec{a}\cdot \Vec{b}}{\Vec{b}\cdot \Vec{b}}\right)\Vec{b}\]

\begin{figure}[H]
  \centering
  \captionsetup{justification=centering}%,margin=2cm
  \includegraphics[width=0.3\textwidth]{secciones/imagenes/proyeccion.png}\label{fig:proyection}
  \caption{Proyeccinón \(\Vec{a}\) sobre \(\Vec{b}\)}
\end{figure}
Para aplicar esta ecuación, vamos a calcular los dos vectores, el vector de dirección \(\Vec{v}=\Vec{b}-\Vec{a}\) de la recta, que está centrado en el origen. Por otro lado, el vector a proyectar \(\Vec{w}=\Vec{p}-\Vec{a}\) que depende de la posición \(\Vec{p}\), centrado también en el origen. Definimos la función de distancia con signo, cómo:
\[SDFRecta_{\Vec{a},\Vec{b}}(\Vec{p})=\vert\vert (\Vec{p}-\Vec{a}) - \text{proy}_{\Vec{p}-\Vec{a}}\left(\Vec{b}-\Vec{a}\right)\vert\vert\]

\begin{lstlisting}
// Operador Proyección a sobre b
vec2 proy(in vec2 a, in vec2 b){
    return b * dot(b, a) / dot(b, b);
}
// Línea Exacto
float SDFRecta(vec2 p, vec2 a, vec2 b){
    vec2 v = p - a;
    vec2 w = b - a;
    return length(v -  proy(v, w));
}
\end{lstlisting}
\begin{figure}[H]
  \centering
  \captionsetup{justification=centering}%,margin=2cm
  \includegraphics[width=0.8\textwidth]{secciones/imagenes/sdf_recta.jpeg}\label{fig:recta}
  \caption{FDS Recta que pasa por \(\Vec{a}=(0.2, 0.2), \Vec{b}=(0.0, 0.1)\)}
\end{figure}

\subsection{Segmento exacto}
Se trata de un caso particular del operador proyección entre dos vectores \(\Vec{a}, \Vec{b}\). 
\[ \text{proy}_{\Vec{b}}\Vec{a}=\left(\dfrac{\Vec{a}\cdot \Vec{b}}{\Vec{b}\cdot \Vec{b}}\right)\Vec{b}\]
Observamos que \(\dfrac{\Vec{a}\cdot \Vec{b}}{\Vec{b}\cdot \Vec{b}}\) es el factor de proyección que está en \(\mathbb{R}\). Cuando este es cero, la proyección será \((0,0)\) y cuando sea \(1\), tomará el vector \(\Vec{b}\). Solo tendremos que limitar este valor de este factor en el intervalor \([0,1]\):
\[ \text{proy[0,1]}_{\Vec{b}}\Vec{a}=\max\left(\min\left(\dfrac{\Vec{a}\cdot \Vec{b}}{\Vec{b}\cdot \Vec{b}}, 0\right), 1\right)\Vec{b}\]

\begin{lstlisting}
// Operador Proyección [0,1] a sobre b
vec2 proy01(in vec2 a, in vec2 b){
    return b * clamp(dot(b, a) / dot(b, b), 0., 1.);
}
// Segmento Exacto
float SDFSegmento(vec2 p, vec2 a, vec2 b){
    vec2 v = p - a;
    vec2 w = b - a;
    return length(v -  proy01(v, w));
}
\end{lstlisting}

\begin{figure}[H]
  \centering
  \captionsetup{justification=centering}%,margin=2cm
  \includegraphics[width=0.8\textwidth]{secciones/imagenes/sdf_segmento.jpeg}\label{fig:segmento}
  \caption{FDS Recta que pasa por \(\Vec{a}=(-0.2, -0.2), \Vec{b}=(0.3, 0.4)\)}
\end{figure}

Existen infinitas \textit{funciones de distancia con signo exactas}, aunque muchas no han sido propuestas. Vamos, ahora, a presentar ahora algunas transformaciones u operadores que preservan su exactitud, así como otros operadores que no lo hacen y veremos su consecuencia cuando vayamos a utilizarlas para generar superficies sobre \(\mathbb{R}^3\).\\\\
Aunque no hemos podido definir más funciones, estas nociones nos ayudará a entender la definición de algunas otras funciones, por ejemplo, la definición exacta del triángulo, utiliza tres segmentos y coordenadas baricéntricas\footnote{Podemos encontrar la definición en el siguiente enlace así como la definición de este tipo de coordendas}.

\newpage

\section{Operadores sobre \(\mathbb{R}^2\)}
Vamos a ver operadores imprescindibles para poder manipular nuestra escena así como crear nuevas funciones de distancia con signo exactas.

\begin{definition}
Llamaremos \textit{Isometría} a la aplicación \(f\) entre dos espacios métricos que conservan la distancia entre los puntos. \[\forall \Vec{v},\Vec{w} \in\mathbb{R}^2, d(\Vec{v},\Vec{w})=d(f(\Vec{v}),f(\Vec{w}))\]
\end{definition}

Como estamos trabajando sobre la métrica euclídea, \(d(\Vec{v},\Vec{w})=d((x,y),(z,w))=\vert\vert \Vec{v}-\Vec{w}\vert\vert=\sqrt{(x-z)^2+(y-w)^2}\) 
Vamos a ver tres isometrías afín\footnote{Una isometría afín ...}, la tralsación, la rotación y la simetría axial. Que a partir de ahora haremos uso de aquí en adelante.

\subsection{Operador de traslación}
Dado un vector \(\Vec{t}\), definimos una traslación como el operador adición  sobre cualquier punto \(\Vec{p}\) con \(\Vec{t}\).
\[f(\Vec{p})=\text{traslacion}(\Vec{p},\Vec{t})=\Vec{p}\pm\Vec{t}\]
Vamos a demostrar que la traslación es una isometría, \(\forall \Vec{v},\Vec{w}\in\mathbb{R}^2\):
\[d(f(\Vec{v}), f(\Vec{w}))=d(\Vec{v}\pm\Vec{t}, \Vec{w}\pm\Vec{t})=\vert\vert (\Vec{v}\pm\Vec{t})-(\Vec{w}\pm\Vec{t})\vert\vert=\vert\vert \Vec{v}-\Vec{w}\vert\vert=d(\Vec{v}, \Vec{w})\]
Como se ha comentado anteriormente, las \textit{funciones de distancia con signo} han sido definidas sobre el origen, es por ello que ahora podemos trasladarnos utilizando el operador propuesto, un ejemplo sería.
\begin{lstlisting}
float escena_sdf(vec2 p){
    // Trasladamos el vector p hacia 0.1 izquierda y 0.2 a la derecha.
    vec2 pt = p - vec2(0.1, 0.2);
    return SDFCircunsferencia(pt, 0.3);
}
\end{lstlisting}

En el ejemplo anterior, hemos utilizado la resta para desplazar hacia la posición deseada, nos lo podemos imaginar como lo que en realidad movemos es el plano y no las coordenadas.

\begin{figure}[H]
  \centering
  \captionsetup{justification=centering}%,margin=2cm
  \includegraphics[width=0.8\textwidth]{secciones/imagenes/sdf_traslacion.jpeg}\label{fig:traslacion}
  \caption{Traslación FDS \(\Vec{t}=(0.1, 0.2)\)}
\end{figure}

\subsection{Operador de rotación}
Veamos como rotar una figura, esto nos puede ayudar para definir un rombo, por ejemplo. Pero antes, vamos a definir la matriz de rotación con sentido horario y ángulo \(\alpha\) en radianes:
\[ 
\text{rot}(\alpha)=\begin{pmatrix}
    +\cos(\alpha) & -\sin(\alpha)\\
    +\sin(\alpha) & +\cos(\alpha)
\end{pmatrix}
\]
que aplicada sobe un vector \(\Vec{v}=\begin{pmatrix}
    x\\
    y
\end{pmatrix}\),
\[ 
f(\Vec{p})=\text{rotacion}_\alpha(\Vec{p})=\Vec{v}\cdot\text{rot}(\alpha)=\begin{pmatrix}
    x\\
    y
\end{pmatrix}^t\cdot\begin{pmatrix}
    +\cos(\alpha) & -\sin(\alpha)\\
    +\sin(\alpha) & +\cos(\alpha)
\end{pmatrix}
\]
\[\text{rotacion}_\alpha(\Vec{p})=\begin{pmatrix}
    +x\cos(\alpha) + y\sin(\alpha)\\
    -x\sin(\alpha) + y\cos(\alpha)
\end{pmatrix}
\]
Vamos a demostrar que este operador es también una \textit{isometría}, \(\forall \Vec{v},\Vec{w}\in\mathbb{R}^2\):
\[d(f(\Vec{v}), f(\Vec{w}))=d(\Vec{v}\cdot \text{rot}(\alpha), \Vec{w}\cdot \text{rot}(\alpha))=\]\[\vert\vert \Vec{v}\cdot \text{rot}(\alpha)- \Vec{w}\cdot \text{rot}(\alpha)\vert\vert=\vert\vert(\Vec{v}-\Vec{w})\cdot \text{rot}(\alpha)\vert\vert\]
Como \(\text{rot}(\alpha)\) es ortogonal, 
\footnote{Aplicamos una propiedad de las matrices ortogonales cuya demostración se puede encontrar en el siguiente enlace https://math.stackexchange.com/questions/1754712/orthogonal-matrix-norm} \(\vert\vert A\cdot\text{rot}(\alpha)\vert\vert=\vert\vert A\vert\vert\)
\[\vert\vert(\Vec{v}-\Vec{w})\cdot \text{rot}(\alpha)\vert\vert=\vert\vert\Vec{v}-\Vec{w})\vert\vert=d(\Vec{v},\Vec{w})\]
En código,
\begin{lstlisting}
#define PI 3.1415
mat2 rot(float a){
    return mat2(
        +cos(a), -sin(a), 
        +sin(a), +cos(a)
    );
}
// Escena
float escena_sdf(vec2 p){
    // Rotacion del el vector p 45 grados o pi / 4 radianes.
    vec2 pr = p * rot(45. * PI / 180.);
    return SDFRectangulo(pr, vec2(0.3));
}
\end{lstlisting}

\begin{figure}[H]
  \centering
  \captionsetup{justification=centering}%,margin=2cm
  \includegraphics[width=0.8\textwidth]{secciones/imagenes/sdf_rotacion.jpeg}\label{fig:rotacion}
  \caption{Rectángulo (cuadrado) FDS con rotación \(\alpha=\dfrac{\pi}{4}\)}
\end{figure}

\subsection{Operador de simetría}
La simetría realiza un reflejo sobre una recta o alguno de los ejes, para ello, vamos a utilizar dos puntos que definen a la recta \(\Vec{a}, \Vec{b}\in\mathbb{R}^2\).
Para este cálculo, miraremos la proyección del punto \(\Vec{p}\) sobre la recta y trasladarlo hacia el otro lado de la recta sobre la dirección al punto proyectado.
\[\text{simetria}_{\Vec{a}}(\Vec{b})=2(\text{proy}_{\Vec{a}}(\Vec{b})-\Vec{b})\]
Donde \(\Vec{a}\) es un vector director de la recta que pasa por el \(0,0\). Para una recta formada por dos puntos \(\Vec{a}, \Vec{b}\), el operador de simetría será:
\[\text{simetria}_{\Vec{a},\Vec{b}}(\Vec{p}) = \text{simetria}_{\Vec{b}-\Vec{a}}(\Vec{p}-\Vec{a})+\Vec{p}\]
Existen casos particulares definidos sobre los ejes, dado un punto \(\Vec{p}=(x,y)\), definimos el operador de simetría sobre el eje x como \(\Vec{p}_x=(-x,y)\) y para el eje y, el punto simétrico \(\Vec{p}_y=(x,-y)\). Este operador será útil más adelante para optimizar cálculo de funciones de distancia con signo duplicadas.
\newpage
Veamos la definición en código,
\begin{lstlisting}
// Simetría sobre el origen
vec2 simetria(vec2 a, vec2 b){    
    return 2. * (proy(b, a) - b);
}
// Sobre cualquier recta genérica formado por dos puntos, a y b.
vec2 simetria(vec2 p, vec2 a, vec2 b){    
    return p + simetria(b - a, p - a);
}
\end{lstlisting}
Veamos el ejemplo en práctica sobre la recta definida en \fullref{fig:recta} con la función de distancia con signo definida en \fullref{fig:segmento}.
\begin{lstlisting}
float escena_sdf(vec2 p){
    // Recta Simetría
    vec2 a = vec2(0.2, 0.2);
    vec2 b = vec2(0.0, 0.1);
    // Simetría
    vec2 ps = simetria(p, a, b);
    
    return SDFSegmento(
        ps,
        vec2(-0.2, -0.2), 
        vec2(0.3, 0.4)
    );
}
\end{lstlisting}

\begin{figure}[H]
  \centering
  \captionsetup{justification=centering}%,margin=2cm
  \includegraphics[width=0.8\textwidth]{secciones/imagenes/sdf_simetria.jpeg}\label{fig:simetria}
  \caption{Simetría FDS ejemplos anteriores}
\end{figure}

Veamos ahora dos operadores entre distintas \textit{funciones de distancia con signo}. Estos operadores van a ser vitales para crear escenas complejas ya que van a permitir agregar o sustraer \textit{funciones de distancia con signo exactas}.

\subsection{Operador de agregación}
Ya vista la función de traslación, ahora nos puede ser útil ver como agregar dos \textit{funciones de distancia con signo}, para así poder crear escenas más complejas. Como su propio nombre indica, si tenemos dos funciones de esta categoría, las cuales devuelven la distancia de manera individual a la superficie o perímetro más cercano, debemos quedarnos con la distancia más pequeña de las dos. Es por eso que el operador de agregación, se define como:
\[\text{Agregacion}(\Vec{p}, f, g) = \min(f(\Vec{p}), g(\Vec{p})) \]
que haremos uso de la función definida por el lenguaje \textit{GLSL}, \textit{min}. Veamos un ejemplo con dos ejemplos de \textit{isometrías}, la traslación \fullref{fig:traslacion} y la rotación \fullref{fig:rotacion}.
\begin{lstlisting}
// Escena
float escena_sdf(vec2 p){
    // Rotamos el rectángulo 45 grados - f
    vec2 pr = p * rot(PI / 180. * 45.0);
    // Trasladamos el rectangulo hacia 0.1 - g izquierda y 0.2 a la derecha.
    vec2 pt = p - vec2(0.4, 0.15);
    // Unión de dos FDS
    return min(
        SDFRectangulo(pr, vec2(0.3)), // f
        SDFCircunsferencia(pt, 0.3)   // g
    );
}
\end{lstlisting}

\begin{figure}[H]
  \centering
  \captionsetup{justification=centering}%,margin=2cm
  \includegraphics[width=0.8\textwidth]{secciones/imagenes/sdf_add.jpeg}\label{fig:add}
  \caption{ Adición de dos FDS de ejemplos anteriores}
\end{figure}
Se puede encadenar varias \textit{FDS} componiendo múltiples veces el operador \textit{min} sobre las distintas figuras. \\\\
Como vemos, este operador funciona como un \textit{or} en el diagrama de Venn\footnote{El diagrama de Venn}.

\subsection{Operador de substracción}
Antes hemos visto que el operador \enquote{\(\min\)} combina dos \textit{FDS}. El operador \enquote{\(\max\)} devuelve la distancia más lejana, esto quiere decir que devuelve siempre el exterior de alguna de las dos figuras en caso de existir. El interior de las figuras en caso de que ambas figuras se solapen, ya que, en caso de estar en el interior, nos quedaremos con la máxima distancia negativa.

\begin{figure}[H]
  \centering
  \captionsetup{justification=centering}%,margin=2cm
  \includegraphics[width=0.8\textwidth]{secciones/imagenes/sdf_subtract-1.jpeg}\label{fig:disyunccion}
  \caption{ Disyunción de dos FDS de ejemplos anteriores}
\end{figure}

Vemos que el operador funciona como \textit{and} en el diagrama de Venn. Pero en la práctica, resulta poco útil, por lo que vamos a definir otra operación que junto a este, nos ayudará crear el operador de substracción. Podemos a cualquier \textit{función de distancia con signo} cambiar el interior por el exterior, multiplicando por \(-1\).

\begin{figure}[H]
  \centering
  \captionsetup{justification=centering}%,margin=2cm
  \includegraphics[width=0.8\textwidth]{secciones/imagenes/sdf_subtract-2.jpeg}\label{fig:negative}
  \caption{ Interior por Exterior Circunsferencia FDS}
\end{figure}

Viendo la imagen, observamos que las distancias negativas representan el exterior y ahora, junto con la definición dada anteriormente de substracción, aquella zona donde las distancias negativas coinciden, o mejor, aquellos puntos donde las figuras no se solapan, serán los resultantes. Podemos definir la substracción de una función de distancia con signo \(f\) otra \(g\) tal que:
\[\text{substraccion}(\Vec{p}, f,g)=max(f(\Vec{p}), -g(\Vec{p}))\]
En código,
\begin{lstlisting}
float escena_sdf(vec2 p){
    // Rotamos el rectángulo 45 grados
    vec2 pr = p * rot(PI / 180. * 45.0);
    
    // Trasladamos el rectangulo hacia 0.1 izquierda y 0.2 a la derecha.
    vec2 pt = p - vec2(0.4, 0.15);
    // Sustraemos Una circunsferencia de un rectángulo.
    return max(
        SDFRectangulo(pr, vec2(0.3)),
        -SDFCircunsferencia(pt, 0.3)
    );
}
\end{lstlisting}

\begin{figure}[H]
  \centering
  \captionsetup{justification=centering}%,margin=2cm
  \includegraphics[width=0.8\textwidth]{secciones/imagenes/sdf_subtract-3.jpeg}\label{fig:substraction}
  \caption{Substracción de una circunsferencia a un rectangulo FDS}
\end{figure}

Cabe destacar que esta figura no preserva las distancias, por lo que el objeto restado tendrá el espacio distorsionado, veremos en los siguientes capítulos, como solucionarlo.\\\\ Vamos a ver otras transformaciones que manipulan el espacio, esto hará que estas funciones no sean exactas, pero hay situaciones en las que al no conocer la fórmula exacta, resulta más fácil manipular el espacio.

\subsection{Operador de escalado}
Veamos la deformación más sencilla y como podemos solucionar esta deformación para así no tener que manipular el \textit{Marcher}.
Supongamos que existe una isometría que reduce las distancas, es decir:
\[f(\Vec{p})=\Vec{p}\cdot \dfrac{1}{k}, k \in \mathbb{R}^{+}_{0}\]
Vemos que \(k\) es el factor de escalado, veamos que no es una isometría y como lo solucionamos, \(\Vec{v},\Vec{w}\in \mathbb{R}^2\).
\[d(f(\Vec{v}),g(\Vec{w}))=d(\Vec{v}\cdot \dfrac{1}{k}, \Vec{w}\cdot \dfrac{1}{k}) = \vert\vert \Vec{v}\cdot \dfrac{1}{k} - \Vec{w}\cdot \dfrac{1}{k}\vert\vert=\vert\vert (\Vec{v} - \Vec{w})\cdot \dfrac{1}{k}\vert\vert\]
Vemos que la distancia transformada es proporcional a la distancia inicial con factor \(k\). Para solucionar esto, multiplicamos la distancia por el factor \(k\) y consiguiendo así una isometría.
En código,

\begin{lstlisting}
// Escalado con factor k
vec2 escena_sdf(vec2 p){
    // Factor de escalado, k=0.5, mitad original. 
    float k = 0.5;
    // Escalamos
    vec2 pk = p / k; 
    // Dividimos la distancia entre el factor de escalado para conseguir una isometría.
    return primitiva(pk) * k;
}
\end{lstlisting}

La función \textit{primitiva} representa cualquier \textit{función de distancia con signo}. Si \(k<1\) estamos escalando hacia abajo, cuando \(k=1\), estamos preservando el tamaño original, finalmente, con \(k>1\) estamos escalando hacia arriba.\\\\
Veamos un ejemplo complejo, por ejemplo, escalamos el ejemplo anterior con un factor de \(k=0.5\) o la mitad:

\begin{lstlisting}
float escena_sdf(vec2 p){
    // Factor de escalado
    float k = 0.5;
    // Escalamos
    p = p / k;
    
    // Rotamos el rectángulo 45 grados
    vec2 pr = p * rot(PI / 180. * 45.0);
    // Trasladamos el rectangulo hacia 0.1 izquierda y 0.2 a la derecha.
    vec2 pt = p - vec2(0.4, 0.15);
    
    // Sustraemos Una circunsferencia de un rectángulo.
    float d = max(
        SDFRectangulo(pr, vec2(0.3)),
        -SDFCircunsferencia(pt, 0.3)
    );
    // Multiplicamos por la distancia para hacerlo una isometría.
    return d * k;
}
\end{lstlisting}

\begin{figure}[H]
  \centering
  \captionsetup{justification=centering}%,margin=2cm
  \includegraphics[width=0.8\textwidth]{secciones/imagenes/sdf_subtracted_scale.jpeg}\label{fig:substraction}
  \caption{\fullref{fig:substraction_scaled} escalado \(k=0.5\)}
\end{figure}

\subsection{Operador de deformación sin exactitud}
Eisten deformaciones que no preservan la métrica, es decir, hace que la función no sea exacta. En general, consiste en manipular el vector \(\Vec{p}\) antes de ser utilizado en nuestra función de distancia con signo, esto provocará una modificación del \textit{Marcher} cuando lo tratemos en el espacio \( \mathbb{R}^3 \) y su eficiencia.

Se trata de una aplicación \(g:\mathbb{R}^2\longrightarrow \mathbb{R}^2\) el cual no es una isometría, transformando la \textit{función de distancia con signo exacta} en otra, no exacta.
\[ \forall \Vec{x}, \Vec{y} \in \mathbb{R}^2, d(g(x), g(y)) \neq d(x,y) \]
Se recomienda utilizar aplicaciones \( g\) que sean continuas y derivables, para así evitar transformaciones fuertes. Por ejemplo,
\[g(x,y)=(x * cos(y \cdot \pi), y * sin(y \cdot \pi)\]
En código,
% TODO: Circunsferencia -> Circunferencia
\begin{lstlisting}
float sdf(vec2 p){
	// Vec2 No Isometría
	vec2 pn = vec2(
	    p.x * cos(p.y * PI),
	    p.y * sin(p.y * PI)
	);
	return SDFCircunferencia(pn, 0.1);
}
\end{lstlisting}
El resultado de la deformación:
\begin{figure}[H]
  \centering
  \captionsetup{justification=centering}%,margin=2cm
  \includegraphics[width=0.8\textwidth]{secciones/imagenes/sdf_deform.jpeg}\label{fig:deform}
  \caption{Deformación del espacio de una circunsferencia FDS}
\end{figure}

Las aplicaciones explicadas anteriormente, en general, se extienden a \(\mathbb{R}^3\). Por ejemplo, vamos a ver como primitivas generalizan a las de una dimensión superior.

\section{Primitivas en \(\mathbb{R}^3\)}

\subsection{Esfera exacta}
Esta función es generalizada de la ecuación en 2D, la idea es la misma, aquellas distancias que antes eran positivas con valor \(r\), quedan anuladas al restarles \(r\) y por tanto, forman una \textit{isosuperficie}. En código,
\begin{lstlisting}
// Esfera R3
float SDFEsfera(vec3 p, float r){
    return length(p) - r;
}
\end{lstlisting}
Utilizando el \textit{Marcher} y el \textit{modelo de iluminación de Phong}, el resultado es el observado en la \fullref{fig:phong}.

\subsection{Prisma rectangular exacto}
Este generaliza de la misma forma, el valor absoluto situará las coordenadas en el cuadrante positivo, de los ocho presentes. Las medidas utilizadas serán la mitad a la original, es decir, \(\Vec{s’}= \dfrac{\Vec{s}}{2}\). Aunque no vamos a ver la demostración, una idea de esta es similar a la utilizada para demostrar el rectángulo, cada región exterior de cada cara es anulada, el interior se utiliza la distancia al lado más próximo y en la esquina, la distancia euclídea. El resultado.

\begin{lstlisting}
// Prisma R3
float SDFPrisma(vec3 p, vec3 s){
    vec3 pa = abs(p) - s;
    return length(max(pa, 0.)) +
    min(max(max(pa.x, pa.y), pa.z), 0.);
}
\end{lstlisting}

Para la visualización de este resultado, vamos a aplicar una rotación sobre el eje \(YZ\) que lo veremos para \(\mathbb{R}^3\) en la siguiente sección.

\begin{lstlisting}
float escena_sdf(vec3 p){
    // Rotacion plano yz
    vec3 pr = vec3(p.x, p.yz * rot(PI/4.0));
    // Cubo s=0.3 => s'=0.15
    vec3 sp = vec3(0.15);
    return SDFPrisma(pr, sp);
}
\end{lstlisting}

El resultado,
\begin{figure}[H]
  \centering
  \captionsetup{justification=centering}%,margin=2cm
  \includegraphics[width=0.8\textwidth]{secciones/imagenes/sdf_prisma_rect.jpeg}\label{fig:prisma}
  \caption{Prisma Rectangular \(\Vec{s}=\Vec{0.3}\) rotado \(\alpha_{YZ}=\dfrac{\pi}{4}\) FDS}
\end{figure}

\subsection{Plano con signo}
% TODO: Ordenar el texto de la normal.
Un plano con signo significa que dado los dos subespacios en el que se divide \(\mathbb{R}^3\), uno tendrá valores positivos y el otro, negativos. En general, los valores en los que la normal del plano tengan producto escalar negativo con el vector director de estos hacia el plano, serán positivos, para esto, definimos el operador de signo, \(\text{sign}(a)=\pm 1, 0\).\\\\
Esta función nos será muy útil junto con el operador de substracción para cortar figuras. Su demostración se basa en la proyección de un punto al plano, supondremos que el plano contiene el punto \((0,0,0)\) y tiene un vector normal \(\Vec{n}\) \footnote{Podemos definir un plano utilizando únicamente la normal de este y obligando a contener el \(\Vec{0}\)}.
\[\text{proy}_{\Vec{n}}(\Vec{p}) = \Vec{p} - (\Vec{n}\cdot\Vec{p})\Vec{n} \]
Por lo que la distancia con signo a este:
\[ SDFPlano_{\Vec{n}}(\Vec{p})=\text{sign}\left((\Vec{p}-\text{proy}_{\Vec{n}}(\Vec{p})\right) \cdot \Vec{n})\cdot \vert \vert \Vec{p} - \text{proy}_{\Vec{n}}(\Vec{p}) \vert\vert \]
Simplificamos esta ecuación,
\[ SDFPlano_{\Vec{n}}(\Vec{p}) = \text{sign}\left((\Vec{n}\cdot\Vec{p})\Vec{n}\right) \cdot \Vec{n})\cdot \vert \vert (\Vec{n}\cdot\Vec{p})\Vec{n}  \vert\vert \]
Utilizando las siguientes propiedades: \(\Vec{n}k\cdot\Vec{n}=k\) y \(\vert\vert k\Vec{n}\vert\vert=k\vert\vert\Vec{n}\vert\vert\).
\[ SDFPlano_{\Vec{n}}(\Vec{p})=\text{sign}\left(\Vec{n}\cdot\Vec{p}\right)\cdot (\Vec{n}\cdot\Vec{p}) \vert \vert\Vec{n}\vert\vert=\Vec{n}\cdot\Vec{p}\]

En código,

\begin{lstlisting}
// Plano R3
float SDFPlano(vec3 p, vec3 n){
   return dot(p, n);
}
\end{lstlisting}

Veamos un ejemplo, pero antes, vamos a desplazar el plano hacia abajo, aunque el operador de desplazamiento lo hemos visto en la sección anterior para una diension inferior, veremos que es equivalente para esta dimensión. Este desplazamiento lo tenemos que hacer ya que el plano pasa por el \((0,0,0)\), oclusionando la mitad de la pantalla.

\begin{lstlisting}
float escena_sdf(vec3 p){
    // Desplazamiento
    vec3 pt = p - vec3(0., -1., 0.);
    // Plano con normal hacia arriba
    vec3 n = normalize(vec3(0., 1., 0.));
    return SDFPlano(pt, n);
}
\end{lstlisting}

El resultado,

\begin{figure}[H]
  \centering
  \captionsetup{justification=centering}%,margin=2cm
  \includegraphics[width=0.8\textwidth]{secciones/imagenes/sdf_plano.jpeg}\label{fig:plano}
  \caption{Plano \(\Vec{n}=(0,1,0)\) desplazado \(\Vec{t}=(0, -1, 0)\)}
\end{figure}

\subsection{Recta Exacta}

La proyección sobre una recta está definida de manera equivalente para una dimensión superior, al igual que la proyección restringinda al \([0,1]\) o el segmento.
En código,
\begin{lstlisting}
// Proyección a sobre b
vec3 proy(in vec3 a, in vec3 b){
    return b * dot(b, a) / dot(b, b);
}
// Proyección a sobre b restringido 0, 1
vec3 proy01(in vec3 a, in vec3 b){
    return b * clamp(dot(b, a) / dot(b, b), 0., 1.);
}
// Recta SDF
float SDFRecta(vec3 p, vec3 a, vec3 b){
    vec3 v = p - a;
    vec3 w = b - a;
    return length(v -  proy(v, w));
}
// Segmento SDF
float SDFSegmento(vec3 p, vec3 a, vec3 b){
    vec3 v = p - a;
    vec3 w = b - a;
    return length(v -  proy01(v, w));
}
\end{lstlisting}

\begin{figure}[H]
  \centering
  \captionsetup{justification=centering}%,margin=2cm
  \subfloat[Imagen original]{\includegraphics[width=0.4\textwidth]{secciones/imagenes/sdf_recta_3d.jpeg}\label{fig:recta3d}}
  \hfill
  \subfloat[\(f\) aplicada sobre la imagen original]{\includegraphics[width=0.4\textwidth]{secciones/imagenes/sdf_segmento_3d.jpeg}\label{fig:segmento3d}}
  \caption{Recta y Segmento Exacto \(\Vec{a}=\Vec{0}\), \(\Vec{b}=\Vec{0.5}\) SDF}
\end{figure}

Veremos un operador para incrementar el ancho de la recta o del segmento. Algunos de los operadores que vamos a ver también son generalizaciones de los operadores también  vistos en \(\mathbb{R}^2\).

\section{Operadores sobre \(\mathbb{R}^3\)}

\subsection{Operadores Isométricos}
Estos operadores son equivalentes a los \textit{Operadores sobre \(\mathbb{R}^2\)} isométricos vistos en secciones anteriores, encontramos la traslación, la rotación, la simetría y el escalado. Se puede demostrar que los operadores son isometrías de manera equivalente.\\\\
En particular, vamos a ver la rotación ya que esta se ha visto sobre el plano \(\mathbb{R}^2\) que en el espacio 3-dimensional, representa el plano \(\overline{XY}\). Definimos una rotación sobre cada uno de los tres planos como:
\[\text{rotacion}_{\overline{XY}}^\alpha(\Vec{p}) = \left(x\cos(\alpha) + y\sin(\alpha),-x\sin(\alpha) + y\cos(\alpha),z\right) \]
\[\text{rotacion}_{\overline{YZ}}^\alpha(\Vec{p}) = \left(x, y\cos(\alpha) + z\sin(\alpha),-y\sin(\alpha) + z\cos(\alpha)\right) \]
\[\text{rotacion}_{\overline{XZ}}^\alpha(\Vec{p}) = \left(x\cos(\alpha) + z\sin(\alpha),y,-x\sin(\alpha) + z\cos(\alpha)\right) \]
Aunque su definición parece bastante compleja, en código es bastante simple de implementar,
\begin{lstlisting}
// Matriz de Rotacion
mat2 rot(float a){
    return mat2(
        +cos(a), -sin(a), 
        +sin(a), +cos(a)
    );
}
// Rotación del plano XY
vec3 rotXY(vec3 p, float a){
    vec2 pr = p.xy * rot(a);
    return vec3(pr.x, pr.y, p.z);
}
// Rotación del plano YZ
vec3 rotYZ(vec3 p, float a){
    vec2 pr = p.yz * rot(a);
    return vec3(p.x, pr.x, pr.y);
}
// Rotación del plano XZ
vec3 rotXZ(vec3 p, float a){
    vec2 pr = p.xz * rot(a);
    return vec3(pr.x, p.y, pr.y);
}
\end{lstlisting}

El resultado de rotar una cubo sobre el plano \(\overline{YZ}\) lo podemos ver en la \fullref{fig:prisma}, con un ángulo \(\alpha=\dfrac{\pi}{4}\).\\\\
Vamos a ver ahora operadores que transforman figuras exactas sobre \(\mathbb{R}^2\) en otras en \(\mathbb{R}^3\) de manera exacta. Esto nos será muy útil ya que en algunos casos, calcular la \textit{función de distancia con signo exacta} en \(\mathbb{R}^2\) es más sencillo y utilizar alguno de estos operadores, puede ser de gran ayuda. Vamos a ver dos, el operador de \textit{revoloución} y \textit{extrusión}.

\subsection{Operador de ensanchamiento}
Presentamos el operador de engrosamiento, este operador va a crear otras funciones de distancia con signo con mayor superficie.\\\\
Existen figuras las cuales el marcher no es capaz de trazarlos debido a la poca superficie.\\\\  Restaremos un valor \(k\in\mathbb{R}\) a la función de distancia con signo, algunos autores \footnote{Inigo Quilez lo llama salto de isonivel https://www.iquilezles.org/www/articles/distfunctions/distfunctions.htm} llaman a la resta \enquote{salto de \textit{isosuperficie}}, aunque también se puede utilizar para \textit{isoperímetros}.\\\\ Una idea de la demostración es observar que sobre cualquier punto de una isosuperficie se puede posicionar una esfera cuya superficie es proporcional al radio, la resta de \(k\) sobre las distancias provoca un incremento del radio de la esfera y por tanto, de la superficie.  
\[\text{ensanchar}_k(\Vec{p}, f)=f(\Vec{p})-k\]
Veamos el ensanchamiento del segmento visto en \fullref{fig:segmento3d}, esta nueva función es llamada cápsula.
\begin{lstlisting}
// FDS Cápsula
float SDFCapsula(vec3 p, vec3 a, vec3 b, float k){
	return SDFSegmento(p, a, b) - k;
}
\end{lstlisting}
El resultado es el siguiente,
\begin{figure}[H]
  \centering
  \captionsetup{justification=centering}%,margin=2cm
  \includegraphics[width=0.8\textwidth]{secciones/imagenes/sdf_capsula.jpeg}\label{fig:capsula}
  \caption{Segmento ensanchado exacto FDS}
\end{figure}

Podemos observar que este operador devuelve una figura ensanchada con los bordes redondeados, debido a la métrica. Podemos conseguir una figura con bordes redondeados y con el tamaño original, sin ensanchar, combinándolo con el operador de escalado.

\subsection{Operador de revolución}
El primer operador es muy importante ya que nos va a permitir generar más funciones exactas utilizando las \textit{funciones de distancia con signo} de una dimensión inferior. El resultado de este operador es una figura generada por el giro sobre un eje de la figura plana.\\\\
Calculamos  

Vamos a calcular la distancia positiva al isoperímetro, dado el plano \(\overline{AB}\) sobre el que está la figura \(f:\mathbb{R}^2\longrightarrow\mathbb{R}\) y dado un punto \(\Vec{p}\in\mathbb{R}^3\) sobre la escena, la proyección de \(\Vec{p}\) sobre el plano \(\overline{AB}\) será \(\Vec{p}_{a,b}\) con \(c\) como coordenada independiente y cuya distancia de la proyección al isoperímetro será \(f(\Vec{p}_{a,b})\), por definición. Si \(\nabla h=(c-0)\) es la altura respecto de \(\Vec{p}\) sobre el plano \(\overline{AB}\), se crea un triángulo equilátero de lados \(\nabla h\) y \( f(\Vec{a,b})\) (Obsérvese la figura \fullref{fig:proof_rev}) y la distancia \Vec{p} hasta el isperímetro de la figura sobre el plano será la diagonal del triángulo formado.
\[\text{distancia}_{\overline{AB}}(\Vec{p}, f)= \sqrt{(\nabla h)^2+(f(\Vec{p}_{a,b}))^2}=\vert\vert(c, f(\Vec{p}_{a,b})\vert\vert\]

Podemos utilizar esta técnica para ensanchar también el isoperímetro trazado para la revolución:

\[\text{revolucion}_{\overline{AB}}^k(\Vec{p}, f)=\text{distancia}_{\overline{AB}}(\Vec{p}, f)-k\]

Veamos un ejemplo de una figura exacta que podemos generar, por ejemplo, un toro\footnote{Se trata de una figura en forma de donut}. La obtenemos de revolucionar una circunsferencia desplazada. En código,

\begin{lstlisting}
/// FDS Toro
float SDFToro(vec3 p, float r1, float r2){
    // FDS Circunsferencia desplazada izq r1 del centro.
    float d = length(p.xz) - r1;
    // Calculamos altura
    float dh = p.y - 0.;
    // Calculamos la distancia (la diagonal)
    return SDFCircunferencia(vec2(dh, d),r2);
}
\end{lstlisting}

\subsection{Operador de extrusión}
Este es el último operador y consiste en elongar una figura plana hacia un eje. Pudiendo ser finita o infinita. Cada punto \(\Vec{p}\in\mathbb{R}^3\) se proyecta sobre el plano a extruir y se calcula la función de distancia \(\mathbb{R}^2\). La proyección sobre los planos que conforman los ejes se calcula de manera trivial anulando la componente que no pertenece al plano. Sea \(f:\mathbb{R}^2\longrightarrow\mathbb{R}^2\) una \textit{función de distancia con signo}, una extrusión sobre el eje \(\overline{Y}\) será:
\[\text{extrusion}^\infty_{\overline{XZ}}\left(\Vec{p},f\right) = f\left(\Vec{p}_{x,z}\right)=f\left(\Vec{p}_{x}, \Vec{p}_{z}\right)\]
Dado cualquier plano \(\overline{AB}\),
\[\text{extrusion}^\infty_{\overline{AB}}\left(\Vec{p},f\right) = f\left(\Vec{p}_{a,b}\right)=f\left(\Vec{p}_{a}, \Vec{p}_{b}\right)\]
Si la extrusión quisiéramos que fuera finita, de longitud \(h\), podemos utilizar la componente independiente como longitud, esto creará una función exacta con tapadera. Utilizamos la simetría sobre el eje \(\Vec{C}\), haciendo \(h'=\dfrac{h}{2}\). La altura respecto de la tapadera con signo para cualquier punto \(\Vec{p} \in\mathbb{R}^3\) será,
\[\Delta c  =\vert\Vec{p}_c\vert - h'\]
Dividimos el ejercicio en dos subproblemas, por encima de la tapa \(\Delta c \ge 0\) y aquellas por debajo \(\Delta < 0\). Utilizaremos de referencia \fullref{fig:proof2}.

\begin{figure}[H]
  \centering
  \captionsetup{justification=centering}%,margin=2cm
  \includegraphics[width=0.7\textwidth]{secciones/imagenes/proof_2.jpeg}\label{fig:capsula}
  \caption{Visualización de una extrusión genérica}
\end{figure}

\begin{enumerate}
    \item \textbf{Subproblema 1}. La distancia puede ser negativa o positiva y corresponde con la distancia que devuelve nuesta función \(f\) sobre la proyección del punto en el plano \(\overline{AB}\).
    \[d_1=f(\text{proy}_{\overline{AB}}(\Vec{p}))=f(\Vec{p}_{a,b})\]
    \item \textbf{Subproblema 2}. Se trata de distancias positivas, observamos en la imagen que se forma un triángulo rectángulo de altura \(\Delta c\) y de base \(f(\Vec{p}_{a,b})\) equivalente al \textbf{Subproblema 1}. La diagonal \(d\) corresponde a la distancia de \(\Vec{p}\) a la superficie de la tapadera.
    \[d_2=\sqrt{(\Delta c)^2+f(\Vec{p}_{a,b})^2}\]
\end{enumerate}

Veamos como podemos unificar ambos subroblemas como hicimos con el rectángulo. Cuando \(\Delta c < 0\), queremos que \(d_1=d_2\), esto se consigue haciendo que \(\Delta c\) sea 0 cuando este sea negativo,
\[d_2=\sqrt{\max(\Delta c, 0)^2+f(\Vec{p}_{a,b})^2}=\vert\vert (\max(\Delta c, 0), f(\Vec{p}_{a,b}))\vert\vert\]

Vamos a generar el interior y exterior de la figura, los cuales se anularán de manera individual. Anularemos \(d_2\) cuando el punto esté en el interior. Para ello, en la ecuación, \(d_1\) debe anularse cuando este sea negativo.
\[d_{exterior}=\vert\vert \max((\Delta c, f(\Vec{p}_{a,b})), 0)\vert\vert\]

El interior de la figura ocurre cuando \(\Delta c < 0\) y \(d_1 < 0\), por lo que se anularán cuando sean positivos. Tomaremos el máximo de las dos distancias para encontrar la mínima negativa.
\[d_{ext} = \text{argmax}(\Delta c, d_1) = \min(\max(\Delta c, f(\Vec{p}_{a,b})), 0)\]
Finalmente, como se anulan, podemos sumar y la ecuación resultante será:
\[\text{extrusion}^h_{\overline{AB},\Vec{C}}\left(\Vec{p},f\right)=\vert\vert \max((\Delta c, f(\Vec{p}_{a,b})), 0)\vert\vert +  \text{argmax}(\Delta c, f(\Vec{p}_{a,b}))) \]

Un ejemplo de esta técnica es la fabricación del cilindro exacto con y sin tapa de radio \(r\). Los cilindros en código:
\begin{lstlisting}
// FDS circunsferencia
float SDFCircunferencia(vec2 p, float r){
	return length(p) - r;
}
/// FDS Cilindro Inifinito (Sin Tapa)
float SDFCilindroInfinito(vec3 p, float r){
    // Extrusión infinito de un cilindro sobre el plano YZ.
    return SDFCircunferencia(p.yz, r);
}
/// FDS Cilindro Finito (Con Tapa)
float SDFCilindro(vec3 p, float h, float r){
    // Extrusión de un cilindro sobre el plano YZ.
    // Altura sobre la tapa.
    float dc = abs(p.x) - h;
    // FDS Circunferencia de radio r
    float fp = SDFCircunferencia(p.yz, r);
    // Exterior figura
    float dint = length(max(vec2(dc,fp),0.));
    // Interior de la figura
    float dext = min(max(dc, fp), 0.);
    // Sumamos ambos
    return dint + dext;
}
\end{lstlisting}

Los resultados de las funciones con y sin tapa:

\begin{figure}[H]
  \centering
  \captionsetup{justification=centering}%,margin=2cm
  \subfloat[Cilindro sin tapa]{\includegraphics[width=0.4\textwidth]{secciones/imagenes/sdf_cilindro_infinito.jpeg}\label{fig:cilindro_tapa}}
  \hfill
  \subfloat[Cilindro con tapa]{\includegraphics[width=0.4\textwidth]{secciones/imagenes/sdf_cilindro.jpeg}\label{fig:segmento3d}}
  \caption{Cilindro con \(r=0.2\) y \(h'=0.3\)  FDS}
\end{figure}

\subsection{Operador de agregación y substracción}
Vamos a recordar los operadores de agregación y substracción presentados anteriormente en la sección de operadores en \(\mathbb{R}^2\). Ambos operadores tienen la misma definición, por lo que vamos a ver un ejemplo de ambos. Para ello, vamos a utilizar dos esferas con el mismo radio y una separada de la otra, como si tratara de un \textit{diagrama de Venn}.
\begin{lstlisting}
float escena_sdf(vec3 p){
    // Dos esferas, una trasladadas
    // Agregacion
    return min(
        SDFEsfera(p, 0.3),
        SDFEsfera(p - vec3(0.3, 0., 0.), 0.3)
    );
}
\end{lstlisting}
El resultado de la agregación:
\begin{figure}[H]
  \centering
  \captionsetup{justification=centering}%,margin=2cm
  \includegraphics[width=0.8\textwidth]{secciones/imagenes/sdf_add_3d.jpeg}\label{fig:add3d}
  \caption{Agregación de dos esferas \(r=0.3\) y una trasladada}
\end{figure}

Como ya se ha comentado, el operador \enquote{\(\max\)} devuelve la intersección de ambas figuras. Vamos a utilizar la definición del plano con signo para seccionar una figura, ya que una región es sólida y la otra no. \\\\
Rotaremos la escena, utilizaremos el operador de rotación del plano \(\overline{XZ}\) con \(\alpha=\dfrac{\pi}{4}\) y poder ver así el interior:

\begin{lstlisting}
float escena_sdf(vec3 p){
    // Rotamos el plano XZ, pi / 4 rad
    p = rotXZ(p, PI / 2. * 1.2);
	// Sección de una esfera
    return max(
        SDFEsfera(p, 0.3),
        SDFPlano(p - vec3(0., 0., 0.15), vec3(0., 0., 1.))
    );
}
\end{lstlisting}

Se trata de una sección del eje \(\Vec{z}\), además, se ha aplicado el operador de traslación al plano \(\Vec{t}=(0,0,0.3)\), pudiendo así modificar la posición de la sección. El resultado es el siguiente:

\begin{figure}[H]
  \centering
  \captionsetup{justification=centering}%,margin=2cm
  \includegraphics[width=0.8\textwidth]{secciones/imagenes/sdf_seccion_3d.jpeg}\label{fig:seccion}
  \caption{Una esfera \(r=0.3\) seccionada por un plano \(\Vec{n}=(0,0,-1)\) desplazado}
\end{figure}

En el ejemplo anterior hemos visto que el operador \enquote{\(\max\)} tiene una mayor utilidad por si solo. Vamos a presentar ahora este mismo operador pero como lo hemos visto en \fullref{fig:substraction}, vamos a cambiar el interior por el exterior de la figura a substraer, recibiendo el nombre de \enquote{carvado}. En código:

\begin{lstlisting}
float escena_sdf(vec3 p){
    // Rotamos el plano XZ, pi / 4 rad
    p = rotXZ(p, PI / 4.);
    // Dos esferas, una trasladadas
    // Substraccion b en a => max(a, -b)
    return max(
        SDFEsfera(p, 0.3),
        -SDFEsfera(p - vec3(0.3, 0., 0.), 0.3)
    );
}
\end{lstlisting}

El resultado será una esfera en el centro a la que se le ha carvado otra esfera, trasladada. El resultado:

\begin{figure}[H]
  \centering
  \captionsetup{justification=centering}%,margin=2cm
  \includegraphics[width=0.8\textwidth]{secciones/imagenes/sdf_substract_3d.jpeg}\label{fig:sub3d}
  \caption{Dos esferas \(r=0.3\) y substracción de la trasladada}
\end{figure}

Este último operador no devuelve una función exacta, veamos por qué:

\subsection{Operador de deformación no exacta}

Veamos el último operador, recordemos que este no conserva la métrica, provocando lo que llamaremos \enquote{artefactos} y que veremos en el siguiente capítulo como solventarlos. Vamos a aplicar la siguiente deformación:
\[g(\Vec{p})=(
\Vec{p}_x \cos(10\Vec{p}_y) + \Vec{p}_z\sin(10\Vec{p}_y),
\Vec{p}_y,
\Vec{p}_x\sin(10\Vec{p}_y) + \Vec{p}_z\cos(10\Vec{p}_y)
)
\]
Esta deformación es conocida como \textit{torsión}, consiste en rotar un eje a medida que se incrementa la distancia a este. Utilizaremos un cubo para este ejemplo, en código:

\begin{lstlisting}
float escena_sdf(vec3 p){
    // Rotamos la escena
    vec2 ry = p.yz * rot(PI / 4.0);
    p = vec3(p.x, ry.x, ry.y);
    // Deformación "g"
    float k = 10.0; // periodo.
    float a = p.y * k;
    p = vec3(
    	+p.x * cos(a) + p.z * sin(a),
    	+p.y,
        -p.x * sin(a) + p.z * cos(a)
    );
	// Dibujamos un prisma.
    return SDFPrisma(p, vec3(0.2));
}
\end{lstlisting}

El resultado obtenido es el siguiente:

\begin{figure}[H]
  \centering
  \captionsetup{justification=centering}%,margin=2cm
  \includegraphics[width=0.8\textwidth]{secciones/imagenes/sdf_twist.jpeg}\label{fig:twist}
  \caption{Deformación torsión de un cubo \(\Vec{l}=\Vec{0.2}\)}
\end{figure}

Vemos que el resultado no es el esperado, por ejemplo, la tapadera es no es correcta en el lado inferior, vemos que este se ha deformado, el cual llamaremos \textit{artefacto}. En la siguiente sección vamos a presentar técnicas para solucionar estos problemas, aunque sean costosas.
	% Artefactos
	\chapter{Resolución de artefactos}
En el capítulo anterior hemos visto dos tipos de \textit{funciones de distancia con signo} exactas e inexactas. Las funciones exactas devuelven la escena de manera correcta, ignorando imperfecciones por coma flotante. Esto es debido a que la métrica es la euclídea y el \textit{Marcher} siempre traza una esfera con ningún punto en el interior, de ahí su nombre, \textit{Spheremarching}. Cuando tratamos de funciones inexactas, la esferas trazadas también se deforman, por lo que las distancias también lo hacen y por tanto, pueden contener puntos. \\\\
Encontramos dos problemas cuando tratamos de \textit{funciones de distancia con signo inexactas}, en el \textit{Marcher} puede sobreestimar la distancia o subestimar. \\\\

\section{Sobreestimación de la distancia}
El término sobreestimar lo definimos como superar la distancia mínima real a la superficie. Vamos a ver dos situaciones de esta sobre estimación: Nos encontramos dentro de la figura, con una distancia negativa o se atraviese completamente la superficie. 

\begin{figure}[H]
  \centering
  \subfloat[Estima dentro de la superficie]{\includegraphics[width=0.4\textwidth]{secciones/imagenes/estimation/sobreestimar-interior.png}\label{fig:estima_dentro}}
  \hfill
  \captionsetup{justification=centering}%,margin=2cm
  \subfloat[Estima fuera de la superficie]{\includegraphics[width=0.4\textwidth]{secciones/imagenes/estimation/sobreestimar-exterior.png}\label{fig:estima_fuera}}
  \caption{Dos formas de sobreestimar una superficie}
\end{figure}

Es difícil visualizar las circunsferencias deformadas, por lo que se ha realizado un esquema de las dos situaciones que podemos encontrar. 

\subsection{Sobrestimación en el interior}
Al poder subestimar, podemos encontrarnos en el interior, haciendo que la siguiente distancia trazada \(d_{n+1}\) sea negativa. Como teníamos definido estar sobre una superficie cuando \(d_{n+1}<\epsilon\), ya que \(d_{n+1}\) siempre es positiva cuando trazábamos desde fuera de la superficie  hacia una \textit{función de distancia con signo exacta}. Ahora que se puede sobreestimar, implicará que \(d_{n+1}\) pueda ser negativo y que debido a la condición de parada expuesta, estos puntos del interior son considerados superficie. Para solucionarlo, diremos que estamos sobre una superficie, si y solo si \( \vert d_{n+1} \vert < \epsilon\), este cambio forzará al \textit{Marcher} a que el rayo tenga que salir del interior de la superficie. Veamos el cambio en el código del algoritmo: 
\begin{lstlisting}
float SphereMarching(
    vec3 ojo, 
    vec3 direccion
){
    float distancia = 0.0;
    for(int i = 0; i < PASOS; ++i){
        vec3 p = ojo + direccion * distancia;
        float radio = escena_sdf(p);
        // Ahora esta d_{n+1} puede ser negativa, por lo que miramos el valor absoluto. 
        if(abs(radio) < EPSILON){
            return distancia;
        }
        // radio puede ser positivo o negativo, en caso de ser negativo, intentará escapar del interior hacia el exterior.
        distancia += radio;
        if(distancia >= MAXIMO) break;
    }
    return MAXIMO;
}
\end{lstlisting}
Veamos el efecto que implica este cambio sobre la deformación vista en \fullref{fig:twist}.

\begin{figure}[H]
  \centering
  \captionsetup{justification=centering}%,margin=2cm
  \subfloat[\textit{Marcher} orginal]{\includegraphics[width=0.4\textwidth]{secciones/imagenes/sdf/3d/sdf_twist.png}\label{fig:twistoriginal}}
  \hfill
  \subfloat[\textit{Marcher} sin sobreestimación en el interior]{\includegraphics[width=0.4\textwidth]{secciones/imagenes/sdf/3d/sin_sobreestimacion_interior.png}\label{fig:stimateext}}
  \caption{Comparativa de los cambios realizados en el \textit{Marcher}. A la izquierda, el \textit{Marcher} original, a la derecha, el marcher con \(\vert d_{n+1}\vert < \epsilon\)}
\end{figure}

Aunque los últimos cambios son poco apreciables, podemos observar que se han reducido el número de artefactos. Por ejemplo, ahora la tapa es algo más exacta. 

\section{Sobreestimación de la distancia en el exterior}
Este segundo caso ocurre cuando la deformación aplicada hace que el rayo atraviese la superficie, como ocurre en el ejemplo \textit{\(B\)}.
Esta sobreestimación afecta considerablemente a la eficiencia del \textit{Marcher}. La solución es escalar la distancia más cercana a la superficie, obligando a que la bola deformada no pueda contener ningún punto, lo que provocará un escalado del módulo del rayo. Esto a su vez conlleva a requerir un mayor número de iteraciones.
\[d'_{n+1}=\left(\sum^{n-1}_{i=0} d_{i}\right) + f(\Vec{rayo})\cdot k \leq d_{n+1}\]
El vector \(\Vec{rayo}\) hace referencia al vector lanzado desde el ojo hasta el pixel, \(f\) es nuestra escena y \(k\in[0,1]\) es un factor de escalado para la evitar la sobreestimación, el valor es tomado de manera manual.

\begin{lstlisting}
#define FACTOR_SOBREESTIMACION k
float SphereMarching(
    vec3 ojo, 
    vec3 direccion
){
    float distancia = 0.0;
    for(int i = 0; i < PASOS; ++i){
        vec3 p = ojo + direccion * distancia;
        // Escalado del la bola
        float radio = escena_sdf(p) * FACTOR_SOBREESTIMACION;
        if(abs(radio) < EPSILON){
            return distancia;
        }
        distancia += radio;
        if(distancia >= MAXIMO) break;
    }
    return MAXIMO;
}
\end{lstlisting}
Veamos como afecta el factor \(k\) al trazado de la escena por el \textit{Marcher} sin sobreestimación en el interior.

\begin{figure}[H]
  \centering
  \captionsetup{justification=centering}%,margin=2cm
  \subfloat[k=1.0]{\includegraphics[width=0.3\textwidth]{secciones/imagenes/sdf/3d/sin_sobreestimacion_interior.png}\label{fig:twistoriginal1}}
  \hfill
  \subfloat[k=0.75]{\includegraphics[width=0.3\textwidth]{secciones/imagenes/sdf/3d/sobreestimacion_75.png}\label{fig:twist75}}
  \hfill
  \subfloat[k=0.50]{\includegraphics[width=0.3\textwidth]{secciones/imagenes/sdf/3d/sobreestimacion_50.png}\label{fig:twist50}}
  \caption{\(k\in\{1, 0.75, 0.5\}\) respectivamente sobre el \textit{Marcher} anterior}
\end{figure}

Vemos que la figura es \enquote{exacta} con \(k=0.5\) si observamos la esquina inferior izquierda de la tapadera. Este factor, o reducción de la esfera a la mitad, va a provocar que se requiera el doble de iteraciones para trazar una escena.

\subsection{Subestimación de la distancia}

Esto ocurre cuando el \textit{Marcher} ha acabado y la distancia recorrida por el rayo es inferior al plano trasero, es decir, sigue en la escena. Esto puede ocurrir principalmente cuando el rayo pasa perpendicularmente muy cerca a una superficie \(f(\Vec{rayo}) \ge \epsilon\). La siguiente imagen ilustra este problema:

\begin{figure}[H]
  \centering
  \captionsetup{justification=centering}%,margin=2cm
  \includegraphics[width=0.8\textwidth]{secciones/imagenes/estimation/subestimacion.png}\label{fig:subestimacion}
  \caption{Ejemplo de subestimación de una superficie}
\end{figure}

Estos puntos serán tratados como \textit{fallos} y por tanto, como píxeles de fondo, aunque la escena no lo fuera. Utilizar un factor de sobreestimación \(k > 1\) no sería la solución ya que crearía \textit{artefactos}, la solución propuesta es el incremento del número de iteraciones del algoritmo, en código, incrementar la definición de \textit{PASOS}, provocando un mayor gasto computacional.
	% Materiales
	\chapter{Materiales\label{ch:materiales}}
En este último capítulo, vamos a ver como dar color y texturas a los elementos de nuestra escena. Identificaremos cada elemento asignando un entero positivo \(id \in \mathbb{N}\) que será devuelto junto con la distancia a este objeto, es decir, vamos a devolver un \textit{vec2} cuya componente \enquote{x} será la distancia y cuya componente \enquote{y}, el identificador \(id\). Asignaremos la constante \(id=-1\) cuando no se ha trazado ningún objeto.\\\\
En primer luegar, vamos a modificar el \textit{Marcher} para que este pueda devolver ambos valores:

\begin{lstlisting}
// Devolvemos dos elementos, distancia e id.
vec2 SphereMarching(vec3 ojo, vec3 direccion){
    float distancia = 0.0;
    // Realizamos PASOS iteraciones de marching.
    for(int i = 0; i < PASOS; ++i){
        vec3 p = ojo + direccion * distancia;
        // La escena devuelve el radio de la bola y el id del elemento
        vec2 info = escena_sdf(p);
        // info.x contiene la distancia
        if(info.x < EPSILON){
            // info.y contiene el id de un elemento de la escena.
            // Devolvemos la distancia acumulada (o distancia del ojo a la superficie) y el id.
            return vec2(distancia, info.y);
        }
        // incrementamos la distancia
        distancia += info.x;
        if(distancia >= MAXIMO) break;
    }
    // Devolvemos un id desconocido.
    return vec2(MAXIMO, -1);
}
\end{lstlisting}

El vector devuelto con nombre \textit{info}, toma los valores directamente de la escena, por lo que vamos a modificar el esquema de nuestra función \enquote{\textit{escena\_sdf}}, este ahora devolverá la distancia más cercana a un objeto y su identificador. El esquema será el siguiente:

\begin{lstlisting}
vec2 escena_sdf(vec3 p){
    // Identificador inicial y la distancia máxima.
    float id = -1.0;
    float min_dist = MAXIMO;
    
    // El esquema es el siguiente para cada figura de nuestra escena.
    // 1.Creamos nuestra primera figura.
    float sdf_0 = ....;
    // 2.Comprobamos que esta figura es la más cercana encontrada hasta el momento.
    if(sdf_0 < min_dist){
        // 2.1 En caso afirmativo, actualizamos los valores.
        // Asignamos el id de esta figura.
        id = 0.;
        // Actualizamos la distancia mínima como la distancia a esa figura.
        min_dist = sdf_0;
    }
    
    // Repetimos este esquema para cada elemento de la escena,
    float sdf_1 = ...;
    if(sdf_1 < min_dist){
        id = 1.;
        min_dist = sdf_1;
    }
    ...
    // Finalmente, devolvemos la distancia mínima y el objeto que la devuelve.
    return vec2(min_dist, id);
}
\end{lstlisting}

Al devolver ahora dos componentes, debemos modificar todas las funciones que hacían uso de esta función, donde encontramos \enquote{Normal} y \enquote{ModeloIluminacion}. Vamos a utilizar el valor devuelto, asignar un material, para ello, vamos a crear una nueva función \enquote{obtenerMaterial} Pongamos como ejemplo una sección de un toro y una esfera, el código y apliquemos el esquema utilizado para \enquote{\textit{escena\_sdf}}. El resultado:

\begin{lstlisting}
vec2 escena_sdf(vec3 p){
    // Identificador inicial y distancia máxima.
    float id = -1.0;
    float min_dist = MAXIMO;
    // Toro de radio interno 0.3 y radio externo 0.05.
    // Seccionado por un plano n = -z.
    // Rotamos el toro 
    vec3 pr = rotYZ(p, PI / 4.);
    float sdf_0 = max(
        SDFToro(pr, 0.3, 0.05),
        SDFPlano(p, vec3(0., 0., -1.))
    );
    // Comprobamos que sea la mas cercana.
    if(sdf_0 < min_dist){
        // Identificador del toro
        id = 0.;
        min_dist = sdf_0;
    }
    // Esfera de radio 0.2
    float sdf_1 = SDFEsfera(p, 0.2);
    // Comprobamos que sea la mas cercana.
    if(sdf_1 < min_dist){
        // identificador de la esfera.
        id = 1.;
        min_dist = sdf_1;
    }
    // Finalmente, devolvemos la distancia mínima y el objeto que la devuelve.
    return vec2(min_dist, id);
}
\end{lstlisting}


\begin{figure}[H]
  \centering
  \captionsetup{justification=centering}%,margin=2cm
  \includegraphics[width=1.0\textwidth]{secciones/imagenes/material/materiales.png}\label{fig:material}
  \caption{Materiales asignados a las distintas figuras.}
\end{figure}

Enlace del ejemplo:\url{https://www.shadertoy.com/view/wlBBRR}
	
	\newpage
	
	\bibliography{bibliografia.bib}
	\bibliographystyle{apalike}
	\printbibliography

    \thispagestyle{empty}

\chapter{Agradecimientos}
Agradecer a mi tutor, Prof. Dr. Juan Carlos Torres Cantero, por la ayuda proporcionada en el desarrollo del proyecto y su corrección. Además,
agradecer al profesor, Jesús García Miranda, por su constante apoyo durante estos cuatro años de carrera.\\\\
Agradecer a toda mi familia por confiar siempre en mi, ayudándome tanto sentimental como en lo económico, pudiéndo desarrollar unos estudios universitarios dignos  que todos deberíamos recibir. Destacar a mis abuelos maternos, con los que he convivido desde los 9 años y a los que no podré agradecer lo suficiente. \\\\
Quiero agradecer a la persona que me inspiró a desarrollar este proyecto, Íñigo Quilez, divulgador del tema y uno de los creadores de la plataforma \textit{Shadertoy}. Además de poner un granito de arena más a mi pasión por las matemáticas.\\\\
Finalmente, agradecer a mi amigo y compañero de la escuela, José Francisco Morales Garrido, que ha sido un pilar fundamental tanto emocional como personal, al que aprecio mucho.
    \pagestyle{empty}

\hfill

\vfill


\pdfbookmark[0]{Colofón}{colofon}
\section*{Colofón}
Este trabajo ha sido escrito en Granada en junio de 2020 y acabado en septiembre de 2020.\\\\
Ha sido compuesto utilizando \LaTeX\ con el estilo proporcionado por el paquete \texttt{classicthesis}, desarrollado por André Miede e Ivo Pletikosić e inspirado en el del libro de Robert Bringhurst, «\emph{The Elements of Typographic Style}».
Puede obtenerse una copia de dicho paquete en
\begin{center}
\url{https://bitbucket.org/amiede/classicthesis/}.
\end{center}
Las tipografías utilizadas han sido \emph{EBGaramond} para el cuerpo, \emph{Garamond Math} para las matemáticas, \emph{Open Sans} para las leyendas y \emph{Go Mono} para el código.
\bigskip
	
\end{document}

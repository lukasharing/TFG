\chapter{Lenguaje GLSL\label{ch:glsl}}

Vamos a presentar el lenguaje \textit{OpenGL Shading Language}, \textit{GLSL} \cite{glsl} del que hace uso la tecnología web actual, \textit{WebGL}\footnote{Esta información es tomada del consorcio industrial \textit{Khronos} de estándares abiertos. \url{https://www.khronos.org/webgl/}}. Este lenguaje tiene una sintaxis similar a C y que comparte con muchos otros lenguajes del mismo propósito, por ejemplo, \textit{High Level Shader Language}\footnote{Es un lenguaje desarrollado por la empresa \textit{Microsoft} para su \textit{pipeline} gráfico de su colección de \textit{APIS}, \textit{DirectX}.}, \textit{HLSL} \cite{hlsl}.\\\\
Estos lenguajes son utilizados para la creación de \textit{shaders}, una aplicación ejecutada por la \textit{GPU} capaz de modificar la geometría, \textit{Vertex Shader} o el color, \textit{Fragment Shader}. Estos son ejecutados en una de las etapas de procesado gráfico enlazados por una \textit{API}. Una \textit{API} define un conjunto de protocolos para la intercomunicación con otros, en particular, \textit{OpenGL Shading Language} usa como \textit{API} \textit{WebGL}. Vamos a presentar algunos de los tipos presentes en el lenguaje, funciones matemáticas y operadores de acceso a texturas.\\\\

\section{Tipos}
\begin{table}[H]
    \begin{tabularx}{\textwidth}{l|X}
      \toprule
      Tipo & Definición\\
      \midrule
      int & Entero con signo como.\\
      float & Número real, con precisión de 32 bits. \\
      bool & Ocupa un 8bits y representa dos valores, \textit{true} o \textit{false}. \\
      vecN & Un vector matemático, N-upla de floats, encontramos definidos: vec2, vec3, vec4. \\
      matN & Se trata de una matriz cuadrada N-dimensional, de floats, definidas: mat2, mat3, mat4. \\
      matNxM & Una matriz rectangular de dimensiones \(N\times M\), encontramos: mat2x2, mat2x3, mat2x4, mat3x2, mat3x3, mat3x4, mat4x2, mat4x3, mat4x4. \\
      \bottomrule
    \end{tabularx}
    \caption{Algunos tipos presentes en el lenguaje \label{eq:types}}
\end{table}
\newpage
\section{Enlaces\label{sec:enlaces}}
\begin{enumerate}
    \item \textbf{out} \textless tipo\textgreater \textless variable\textgreater. Enlaza la variable con otra de distinto nivel, tal que, al finalizar el shader, su valor es asignado al enlazado.
    \item \textbf{in} \textless tipo\textgreater \textless variable\textgreater . Enlaza la variable con otra de distinto nivel, tal que al inicio del shader, su valor es asignado por el enlazado otro.
    \item \textbf{uniform} \textless tipo\textgreater \textless variable\textgreater. Enlaza la variable con una variable global, de solo lectura.
\end{enumerate}


\section{Vectores}
El tipo vector, \textit{vecN}, definido por una tupla: \((x, y[, z[, w]])\) ó \((r, g[, b[, a]])\). Cuyos constructores aportan una riqueza semántica al lenguaje, por ejemplo:
\begin{table}[H]
    \begin{tabularx}{\textwidth}{l|X}
      \toprule
      Constructores & Definición\\
      \midrule
      vecN(float, \(\cdots\), float) & \(N\) valores flotantes para cada componente.\\
      vecN(vecM, float) & Asigna los primeros valores del vector los valores del segund y el último, el elemento flotante, con \(M+1=N\) . \\
      vecN(float, vecM) & Asigna al primer atributo, el valor flotante. Los \(M\) restantes, con las comentes del vector \textit{vecM}, donde \(M+1=N\). \\
      vecN(vecP, vecQ) & Los \(P\) primeros atributos del vector asignados por las componentes de \textit{vecP} y los Q restantes, de las componentes del vector \textit{vecQ}, \(N = P + Q\). \\
      \bottomrule
    \end{tabularx}
    \caption{Constructor de vectores en GLSL \label{eq:vecN}}
\end{table}
El operador de acceso, \enquote{.}, a las compentes del vector, devolviendo un nuevo vector o \textit{float} formado por las valores en el orden accedido. Presentamos algunos de los operadores vectoriales implementados en el lenguaje:
\begin{table}[H]
    \begin{tabularx}{\textwidth}{l|X}
      \toprule
      Función & Definición\\
      \midrule
      length(vecN vector) & Devuelve el módulo del vector.\\
      distance(vecN p1, vecN p2) & Distancia entre dos puntos. \\
      normalize(vecN vector) & Devuelve el vector normalizado. \\
      dot(vecN v1, vecN v2) & \textbf{Producto escalar} de ambos vectores. \\
      cross(vecN v1, vecN v2) & \textbf{producto vectorial} de los dos vectores. \\
      \bottomrule
    \end{tabularx}
    \caption{Funciones con vectores en GLSL \label{eq:vecopN}}
\end{table}

Veamos algunos ejemplos de vectores puestos en práctica:

\begin{lstlisting}

// Constructores
vec3 vector = vec3(1.0);
vec3 vector1 = vec3(1.0, 1.0, 1.0);
// vector1 y vector son idénticos.
vec3 vector2 = vec3(vec2(0.0), 1.0);
vec4 vector3 = vec4(1.0, vec2(0.0), 1.0);
vec4 vector4 = vec4(vec2(0.0), vec2(1.0));

// Operadores de acceso
float valor1 = vector1.r;
// Esto es siempre cierto
if(valor1 == vector1.x){...}
// Repetimos la componente accedida:
vec3 vector5 = vector1.xxx;
// Mezclamos las componentes:
vec3 vector6 = vector3.yzw;
// Utilizamos los atributos rgba
vec4 vector7 = vector1.rggb;

// Operadores
vec3 normal = normalize(vector6);
float modulo = length(vector4);
float distancia = distance(vector5, vector6);
float escalar = dot(vector3, vector4);
vec3 vectorial = cross(vector1, vector2);
vec2 vector8 = vec2(
    length(vector.xz), 
    vector.y
);
\end{lstlisting}

Algunos ejemplos de errores sintácticos:
\begin{lstlisting}
// 1. Dimensionalidad incorrecta:
vec3 error = vec3(1.0, 0.5);
vec2 error = vector.rrr;
// 2. Mezcla de atributos.
vec2 error = vector.rx;
// 3. Error en argumentos
float error = dot(vector2, vector3);

\end{lstlisting}

\section{Matrices}
Las matrices \textit{matNxM} y \textit{matN}, formadas por \(N\times M\) y \(N^2\) componentes, respectivamente.
El operador de acceso a las componentes es similar al lenguaje C, del tal forma que: \([j][i]\) accede a la celda de la fila \textit{j-ésima} y columna \textit{i-ésima}.\\\\
Alguno de los constructores para la creación de estas matrices son:
\begin{table}[H]
    \begin{tabularx}{\textwidth}{l|X}
      \toprule
      Constructor & Definición\\
      \midrule
      matN(matM) & Submatriz cuadrada \(N\) superior izquierda de \textit{matM}.\\
      matNxM(matQxP) & Submatriz \(NxM\) superior izquierda de \textit{matQxP}.\\
      matN(float, \(\cdots\), float) & \(N^2\) valores flotantes.\\
      matNxM(float, \(\cdots\), float) & \(N\times M\) valores flotantes. \\
      matN(vecN,..., vecN) & Formado por \(N\) vectores \(N\)-dimensionales. \\
      matNxM(vecM,..., vecM) & Formado por \(N\) vectores \(M\)-dimensionales. \\
      \pbox{10cm}{
      matN(\\
      \tab[1cm]vecM,float,
      \\\tab[1cm]...,
      \\\tab[1cm] vecM, float
      \\)
      }& Formado por \(N\) vectores \((M - 1)\)-dimensionales y \(N\) valores flotantes. \\
      \pbox{10cm}{
      matNxM(\\
      \tab[1cm]vecP,float,
      \\\tab[1cm]...,
      \\\tab[1cm]vecP,float
      \\)
      } & Formado por \(N\) vectores (M -1)-dimensionales y M valores flotantes. \\
      \bottomrule
    \end{tabularx}
    \caption{Constructor de matrices en GLSL \label{eq:consmatN}}
  \end{table}
Hemos visto algunos ejemplos, pero existen una gran variedad de combinaciones posibles. Estas combinaciones tienen que respetar siempre el tamaño de la matriz, pudiéndose intercalar vectores \textit{vecN} y valores flotantes.\\\\
De la misma forma, vamos a ver alguno de los operadores entre matrices.
\begin{table}[H]
    \begin{tabularx}{\textwidth}{l|X}
      \toprule
      Función & Definición\\
      \midrule
      transpose(mat matrix) & Devuelve la transpuesta de la matriz.\\
      matrix1 * matrix2 & Devuelve el producto de matrices, donde matrix1 es una matriz \(N\times M\) y matrix 2 otra \(M\times K\). \\
      determinant(matN matrix) & Devuelve el determinante de una matriz cuadradas. \\
      \bottomrule
    \end{tabularx}
    \caption{Funciones con matrices en GLSL \label{eq:matN}}
\end{table}
Algunos ejemplos de contrucción de matrices, acceso a sus componentes y operadores:
\begin{lstlisting}

// Constructores
mat2 matrix2x2 = mat2(0.0);
mat3 matrix3x3 = mat3(
    1.0, vec2(1.0),
    vec2(0.0), 1.0,
    1.0, 0.0, 1.0
);
// Submatrix
mat2 submatrix2x2 = mat2(matrix1);
mat2x3 matrix2x3 = mat2x3(
    vec3(1.0),
    1.0, 0.0, 1.0
);

// Acceso
// Desde j,i
float valor0 = matrix2x2[0][0];
// Desde j y su componente
float valor1 = matrix2x2[0].x;
// Esto siempre es cierto
if(valor0 == valor1){ ... }
// Obtenemos la fila al completo
vec2 vector1 = matrix2x2[0];

// Operadores
// Transpuesta
mat3 tmatrix3x3 = transpose(matrix3x3);
// Multiplicación
mat2x3 mulmatrix2x3 = matrix2x2 * matrix2x3;
\end{lstlisting}

Errores
\begin{lstlisting}
// 1. Error número de componentes
mat2 matriz = mat2(1.0, 0.0);
// 2. Error de dimensionalidad.
mat2 matriz = mat2(vec3(1.0), vec3(0.0));
// 3. Error en argumentos
mat2 matrix = matrix2x2 * matrix3x3;
\end{lstlisting}
\section{Operadores matemáticos}
Encontramos los operadores usuales: \(+,-,*,/,\%\). Para los números enteros, tenemos también los operadores binarios: \({<<, >>, \mid, \And, \string^}\). Agrupamos \textit{float} y \textit{vecN} con el nombre de \textit{genType} (\cite{glsl}, \enquote{8. Built-in Functions} páginas 140-141), para reunir los tipos de argumentos. Cuando utilizamos un operador sobre el tipo \textit{vecN}, este se aplicará sobre cada una de sus componentes. Algunos operadores trigonométricos y exponenciales son los siguientes:
\begin{table}[H]
    \begin{tabularx}{\textwidth}{l|X}
      \toprule
      Función & Definición\\
      \midrule
      radians(genType var)& Conversión de grados en radianes. \\
      degrees(genType var) & Conversión de radianes en grados. \\
      sin(genType var) & Aplica \(f(x)=\sin(x)\) sobre la variable. \\
      cos(genType var) & Aplica \(f(x)=\cos(x)\) sobre la variable. \\
      tan(genType var) & Aplica \(f(x)=\tan(x)\) sobre la variable. \\
      asin(genType var) & Aplica \(f(x)=\arcsin(x)\) sobre la variable. \\
      acos(genType var) & Aplica la \(f(x)=\arccos(x)\) sobre la variable. \\
      atan(genType var) & Aplica \(f(x)=\arctan(x)\) sobre la variable. \\
      \pbox{10cm}{
      pow(\\
      \tab[1cm]genType a,\\
      \tab[1cm]genType b \\
      )} & Calcula la primera variable elevada a la segunda. \\
      exp(genType var) & Aplica \(f(x)=\exp{x}\) sobre la variable. \\
      exp2(genType var) & Aplica \(f(x)=2^{x}\) sobre la variable.  \\
      log(genType var) & Aplica \(f(x)=\log(x)\) sobre la variable.  \\
      sqrt(genType var) & Aplica \(f(x)=\sqrt{x}\) sobre la variable. \\
      \bottomrule
    \end{tabularx}
    \caption{Algunos operadores matemáticos en GLSL \label{eq:matop1}}
\end{table}
Ejemplos de equivalencias de los operadores:
\begin{lstlisting}
#define PI 3.1415
vec2 radians = vec2(PI, PI/2));
//resultado: aprox vec2(180, 90)
float sink = sin(PI / 2.);// aprox 1.0
float powk = pow(2., 2.); // 4.0
// Ecuacion compleja
float complejo = log(exp(-cos(0.))); // -1.0 
vec2 complejo2 = cos(acos(vec2(0.)));//vec2(0)
\end{lstlisting}
Veamos un error y cual es la solución al problema.
\begin{lstlisting}
float error = pow(vec2(0., 2.), 3.);
// Solución
float solucion = pow(vec2(0., 2.), vec2(3.));
\end{lstlisting}

Otros operadores, también importantes, son: 
\begin{table}[H]
    \begin{tabularx}{\textwidth}{l|X}
      \toprule
      Función & Definición\\
      \midrule
      abs(genType var) & Aplica \(f(x)=\vert x\vert\) sobre la variable. \\
      sign(genType var) & Aplica la función signo sobre la variable.\\
      floor(genType var) & Aplica \(f(x)=\lfloor x\rfloor\) sobre la variable, redondeo inferior. \\
      ceil(genType var) & Aplica \(f(x)=\lceil x\rceil\) sobre la variable, redondeo superior.\\
      round(genType var) & Aplica \(f(x)=\left[ x\right]\) sobre la variable, redoneo normal.\\
      fract(genType var) & Toma la parte fraccional de la variable, \(f(x)=x-\lfloor x\rfloor\).\\
      min(genType a, genType b) & Toma el mínimo de las variables.\\
      max(genType a, genType b) & Toma el máximo de las variables.\\
      \pbox{10cm}{
      clamp(\\
      \tab[0.5cm]genType v,\\
      \tab[0.5cm](genType ó float) min, \\
      \tab[0.5cm](genType ó float) max \\
      )} & Aplica la función acotado inferior \textit{min} y superior \textit{max}, sobre la variable.\\
      \pbox{10cm}{
      mix(\\
      \tab[0.5cm]genType a,\\
      \tab[0.5cm]genType b, \\
      \tab[0.5cm](genType ó float ó bool) h \\
      )} & Aplica \textit{la función de mezcla con peso}\ref{def:mix}, sobre la variable, donde \(f(x) = a, g(x) = b\text{ y } t = h\).\\
      \bottomrule
    \end{tabularx}
    \caption{Más operadores matemáticos en GLSL \label{eq:matop2}}
\end{table}
Algunos otros ejemplos y sus equivalencias:
\begin{lstlisting}
float absk = abs(-1.0);// 1.0
vec2 absv = abs(vec2(-1., 2.));//vec2(1., 2.)
float roundk = 1.23; // 1.00
float floork = 1.53; // 1.00
float ceilk = 1.23; // 2.0
float fractk = 1.23; // 0.23
vec2 min = min(
    vec2(1.0, -1.0),
    vec2(0.0, 2.0)
); // vec2(0.0, -1.0)
float mix1 = mix(1.0, 2.0, 0.5); // 1.5
float mix2 = mix(1.0, 3.0, true); // 3.0
\end{lstlisting}
Los operadores relacionales son los mismos que en el lenguaje C: 
\[<, <=, >, >=, !, ==, !=\]
\section{Acceso a texturas}
Desde \textit{Shadertoy}, podemos utilizar diferentes canales y seleccionar distintos tipos de archivos, en nuestro caso, utilizaremos aquellas que aparecen en la pestaña: "\textit{Texturas}". Una vez seleccionada una textura, podemos utilizar la variable de tipo \textit{sampler2D} con nombre \textit{iChannelN} como canal N-ésimo, para acceder a los valores del archivo.\\\\
Existen las siguientes funciones que permiten obtener el color de un píxel de una variable tipo \textit{sampler2D} en las coordenadas de textura \((s,t)\), que se encuentran normalizados.
\begin{table}[H]
    \begin{tabularx}{\textwidth}{l|X}
      \toprule
      Función & Definición\\
      \midrule
      \pbox{10cm}{
      texture(\\
      \tab[1cm]sampler2D iChannelN,\\
      \tab[1cm]vec2 coord \\
      )} & Devuelve un píxel \textit{vec4 rgba} en la coordenada noormalizada \textit{coord}. \\
      \pbox{10cm}{
      textureLod(\\
      \tab[1cm]sampler2D iChannelN,\\
      \tab[1cm]vec2 coord, \\
      \tab[1cm]float lod, \\
      )} & Devuelve un píxel \textit{vec4 rgba} en la coordenada noormalizada \textit{coord} con un nivel de detalle \textit{lod}. \\
      \bottomrule
    \end{tabularx}
\end{table}
Un ejemplo de acceso a texturas, algoritmo de suavizado:
\begin{lstlisting}
// Sea (x, y) las coordenadas normalizadas.
// Tamaño de la máscara para su suavizado.
#define K 2.0
// Imagen (w, h) en el canal iChannel0
float px = 1. / w, py = 1. / h;
// Suavizado
vec3 pixel = vec3(0.0);
for(float i = -K; i <= K; i += 1.0){     
    for(float j = -K; j <= K; j += 1.0){
        // Pixel de la máscara
        float cx = x + px * i;
        float cy = y + py * j;
        pixel += texture(
            iChannel0,
            vec2(cx, cy)
        ).rgb;
    }
}
// Calculamos la media, de los píxeles
pixel /= pow(2. * K + 1., 2.);
\end{lstlisting}

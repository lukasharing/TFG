\chapter{Conclusiones}

Se trata de un modelo matemático muy sólido, donde, en pocas líneas, podemos crear una escena compleja. Además, nos abre las puertas a un acercamiento a las matemáticas más allá a las vistas en el recorrido universitario. La creciente popularidad de esta técnica ha hecho que se popularice en la industria de los videojuegos, aunque requiera de una alta potencia gráfica. Esta técnica requiere de un alto entendimiento matemático y la capacidad de poder demostrar algunas de las primitivas propuestas, hasta la proposición de nuevas. El principal inconveniente se observa cuando tratamos de funciones de distancia inexactas, obtenidas de los operadores entre \textit{FDS} o deformaciones, provocando un decremento en la eficiencia del algoritmo.\\\\
Una propuesta de trabajo futuro es la investigación de fórmulas exactas y eficientes, que preserven la exactitud en los resultados, la utilización de modelos le iluminación más complejos, por ejemplo, añadir refracción o acotar de manera matemática el factor de empuje sobre una superficie. Un estudio matemático sobre la convergencia del algoritmo o propuesta de optimización del mismo.\\\\
Hemos tratado un proyecto muy interesante que ayuda a comprender de mejor manera las escenas y como se forman, un acercamiento al formalismo matemático de la técnica utilizada y algunas técnicas de demostración matemática.
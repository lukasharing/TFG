\section{Conclusiones}

\SectionPage

\begin{frame}{Conclusiones}
    \begin{itemize}
        \item Se trata de una técnica novedosa y con un ámplio campo de estudio, que requiere de un elevado conocimiento matemático.
        \item El modelo de iluminación es esencial para la creación de escenas tridimensionales.
        \item Utilizar funciones de distancia con signo exactas, que ayudan a la convergencia del algoritmo.
        \item La sub/sobreestimación, requiere de un mayor ejercicio computacional.
        \item Los materiales dan una riqueza visual al ejercicio, en caso de texturización, utilizaremos proyecciones sobre las coordenadas \((u,v)\).
    \end{itemize}
\end{frame}
\documentclass[spanish]{beamer}


\usetheme{UiB}


\usepackage[spanish, es-nodecimaldot]{babel}

\usepackage{xcolor}
\usepackage{listings}
\lstset{
    basicstyle=\tiny,
    extendedchars=true,
    inputencoding=utf8,
	keywordstyle=\bfseries,
	commentstyle=\color{OliveGreen}\itshape,
	morekeywords={sage},
	captionpos=b,
	language=C,
	frame=single,
}
\renewcommand{\lstlistingname}{Listado}


\newcommand\tab[1][1cm]{\hspace*{#1}}

\usepackage{csquotes}       % Quotation marks
\usepackage{microtype}      % Improved typography
\usepackage{amssymb}        % Mathematical symbols
\usepackage{mathtools}      % Mathematical symbols
\usepackage[absolute, overlay]{textpos} % Arbitrary placement
\setlength{\TPHorizModule}{\paperwidth} % Textpos units
\setlength{\TPVertModule}{\paperheight} % Textpos units
\usepackage{tikz}
\usetikzlibrary{overlay-beamer-styles}  % Overlay effects for TikZ

\usepackage[nottoc]{tocbibind}

% Remove first entry

\author{Lukas Häring García}
\title{Trabajo de fin de grado}
\subtitle{Funciones de distancia con signo}


\begin{document}

    \begin{frame}{Índice General}
        \tableofcontents
    \end{frame}
    
    
    % Introduccion
    \pdfbookmark[1]{Introducción}{Introducción}
\addcontentsline{toc}{chapter}{\tocEntry{Introducción}}
\chapter*{Introducción}
Para la confección de este trabajo, ha sido indispensable la ayuda de mi tutor del proyecto, Prof. Dr. Juan Carlos Torres Cantero y al departamento de Lenguajes y Sistemas Informáticos.

\section*{Motivación}
El continuo interés personal sobre las matemáticas y la informática gráfica, empujada por la divulgación de Iñigo Quilez en su blog personal, ha sido el detonante del desarrollo de este proyecto, que agrupa los fundamentos matemáticos y algoritmos. Además, mencionar y agradecer a los autores a \textit{Tuong Phong} y \textit{Jhon C. Hart}, investigadores y padres de muchas de las técnicas que vamos a presentar, indispensables para el desarollo de este trabajo.

\section*{Objetivos}
El objetivo de este trabajo es la experimentación de nuevas técnicas de renderizado y  dar visibilidad al fundamento matemático que lo sustenta. Presentaremos un nuevo lenguaje de programación no visto durante la carrera, que será utilizado durante todo el desarrollo del trabajo. Desarrollaremos técnicas para formar nuestras propias escenas, que son creadas mediante funciones matemáticas, demostrando cada una de las propiedades presentadas.

\section*{Estructura}
Nuestro trabajo está estructurado en siete grandes apartados. Un primer apartado, en el que introducimos los principales conceptos que vamos a emplear a lo largo de nuestro trabajo, para demostrar el dominio de los conceptos trabajados durante estos cuatro años en la Universidad de Granada. Un segundo apartado, en el que presentamos el lenguaje de programación utilizado, \textit{GLSL}. Un tercer apartado, en el que desarrollamos el concepto de \textit{Marcher}, presentando el algoritmo principal de nuestro proyecto, \textit{Spheremarcher} y que será esencial para el trazado de una escena. En el cuarto capítulo, presentaremos algunos operadores importantes para la creación de un modelo de iluminación, junto a dos tipos de luces: radiales y direccionales. Haremos uso de un modelo de iluminación conocido con el nombre de \textit{Modelo Phong}, el cual divide la intensidad lumínica en tres, ambiental, difusa y especular. Un quinto capítulo, donde veremos en profundidad las \textit{funciones de distancia con signo}, en particular, veremos algunas funciones primitivas y sus respectivas demostraciones, que nos dará una intuición general para desarrollar nuevas. Además, vamos a presentar operadores para transformar estas funciones, donde encontramos: la traslación, rotación, escalado y simetría, presentando el concepto de \textit{isometría}. Un operador de deformación, que hará que tengamos que distinguir este tipo de funciones como exactas o inexactas y sus diferencias. Presentaremos también operadores entre dos \textit{funciones de distancia con signo}, el de adición, intersección y substracción. Para las \textit{funciones de distancia con signo} tridimensionales, definiremos dos operadores en este espacio que hacen uso de las funciones definidas en el espacio \textit{bidimensionales} para generar nuevas de manera exacta a partir de una extrusión o una revolución.
Una de las consecuencias de no trabajar con funciones exacta es la aparición de \enquote{artefactos}, en este penúltimo apartado, veremos como solucionarlos y que consecuencias tienen. Finalmente, en este último capítulo, veremos como pasar de una escena trazada en escala de grises a una escena con materiales para cada elemento de nuestra escena, realizaremos dos ejemplos, uno utilizando colores uniformes para cada figura y otro, utilizando una proyección de textura.\\\\
    
    % Lenguaje
    \section{Lenguaje GLSL}

\SectionPage

\begin{frame}{Tipos}
    
    Mantiene una sintaxis similar a C, encontramos los siguientes tipos más importantes.
    \vfill
    
    \begin{itemize}
        \item \textbf{int}. Entero con signo.
        \item \textbf{float}. Número real, con precisión de 32 bits.
        \item \textbf{bool}. Ocupa un byte, \textit{true} o \textit{false}.
        \item \textbf{vecN}. Vector matemático, \(N\)-úpla de floats.\\Definidos: vec2, vec3, vec4.
        \item \textbf{matN}. Matriz cuadrada de dimension \(N\).\\Encontramos: mat2, mat3, mat4.
        \item \textbf{matNxM}. Matriz de dimensiones \(N\times M\).\\Encontramos: mat2x2, mat2x3, mat2x4, mat3x2, mat3x3, mat3x4, mat4x2, mat4x3, mat4x4.
    \end{itemize}
    
\end{frame}

\subsection{Vectores}
\begin{frame}{Vectores}

    El  tipo  vector, vecN,  definido  por  una  t-úpla:  \((x,y[,z[,w]])\) ó \((r,g[,b[,a]])\). Utilizaremos el operador \enquote{.} para acceder y copiar estas componentes.
    \vfill
    
    \begin{columns}[onlytextwidth]
        \begin{column}{0.45\textwidth}
            {\Large Constructores}
            \begin{itemize}
                \item vecN(float,···, float)
                \item vecN(vecM, float)
                \item vecN(float, vecM)
                \item vecN(vecP, vecQ)
            \end{itemize}
        \end{column}
        
        \begin{column}{0.45\textwidth}
            {\Large Funciones}
            \begin{itemize}
                \item length(vecN vector)
                \item distance(vecN p1, vecN p2)
                \item normalize(vecN vector)
                \item dot(vecN v1, vecN v2)
                \item cross(vecN v1, vecN v2)
            \end{itemize}
        \end{column}
        
    \end{columns}

\end{frame}

\subsection{Matrices}
\begin{frame}{Matrices}
    
    Las matrices \textit{matNxM} y \textit{matN}, formadas por \(N\times M\) y \(N^2\) componentes flotantes, respectivamente. El operador de acceso a las componentes es similar al lenguaje C, del tal forma que: \([j][i]\) accede a la celda de la fila \textit{j-ésima} y columna \textit{i-ésima}.
    \vfill
    
    \begin{columns}[onlytextwidth]
        \begin{column}{0.45\textwidth}
            {\Large Constructores}
            \begin{itemize}
                \item matNxM(float, \(\cdots\), float)
                \item matNxM(float, \(\cdots\), float)
                \item matN(vecN,\(\cdots\), vecN)
                \item matNxM(vecM,\(\cdots\), vecM)
                \item matN(matM)
            \end{itemize}
        \end{column}
        
        \begin{column}{0.45\textwidth}
            {\Large Funciones}
            \begin{itemize}
                \item transpose(mat matrix)
                \item matrix1 * matrix2
                \item determinant(matN matrix)
            \end{itemize}
        \end{column}
        
    \end{columns}

\end{frame}

\subsection{Operadores matemáticos}
\begin{frame}{Operadores matemáticos}
    
    Agrupamos \textit{float} y \textit{vecN} con el nombre de \textit{genType} para reunir los tipos de argumentos. Cuando utilizamos un operador sobre el tipo \textit{vecN}, este se aplicará sobre cada una de sus componentes.
    
    \vfill
    
    \begin{columns}[onlytextwidth]
        \begin{column}{0.47\textwidth}
            \begin{itemize}
                \item radians(genType var)
                \item sin(genType var)
                \item tan(genType var)
                \item asin(genType var)
                \item atan(genType var)
                \item pow(genType a, genType b)
                \item exp(genType var)
                \item sqrt(genType var)
                \item sqrt(genType var)
            \end{itemize}
        \end{column}
        
        \begin{column}{0.47\textwidth}
            \begin{itemize}
                \item abs(genType a)
                \item sign(genType a)
                \item min(genType a, genType b)
                \item max(genType a, genType b)
                \item mix(\\
                    \tab[0.5cm]genType a,\\
                    \tab[0.5cm]genType b, \\
                    \tab[0.5cm](genType ó float ó bool) h \\
                )
            \end{itemize}
        \end{column}
        
    \end{columns}

\end{frame}


    % \begin{itemize}
    %     \item
    %     Bullet lists are marked with a red box.
    % \end{itemize}

    % \begin{enumerate}
    %     \item
    %     \label{enum:item}
    %     Numbered lists are marked with a white number inside a red box.
    % \end{enumerate}

    % \begin{description}
    %     \item[Description] highlights important words with red text.
    % \end{description}

    % Items in numbered lists like \enumref{enum:item} can be referenced with a red box.

    % \begin{example}
    %     \begin{itemize}
    %         \item
    %         Lists change colour after the environment.
    %     \end{itemize}
    % \end{example}
    
    % Marcher
    % https://adrianb.io/2016/10/01/raymarching.html#introduction-to-raymarching
\chapter{Marcher}
% https://books.google.es/books?id=MNqRDwAAQBAJ&pg=PA13&dq=sphere+ray+marcher+graphics&hl=es&sa=X&ved=0ahUKEwj1sqWTgZvrAhUCxhoKHRwBCVIQ6AEIJzAA#v=onepage&q=sphere%20ray%20marcher%20graphics&f=false
Gracias a los avances tecnológicos y al incremento de la potencia de los micro-procesadores, a esto, unido la paralelización de tareas y dispositivos hardwares especializados como la \textit{GPU}, ha permitido que se pueda proponer esta técnica. Cada píxel de nuestra pantalla es calculado por una \textit{hebra} de la GPU, una especie de micro-procesador que trabaja de manera individual, sin memoria y sin intercomunicación con las demás. Todas ejecutan el mismo código, contiene información de la pantalla como la coordenada del pixel que está calculado, la resolución, etc. y devuelve un valor \textit{rgba}, en formato \textit{vec4}.\\\\
Existen técnicas para proyectar una escena en nuestras pantallas. Dentro de esta categoría de técnicas, encontramos las técnicas basadas en "rayos" (\textit{rays}). Para cada pixel de nuestra pantalla, "lanzaremos un rayo" desde un ojo, que es el nombre que recibe el origen de la cámara, en dirección al píxel, suponiendo que nuestra pantalla está en la escena. Si este rayo intersecta sobre una superficie, entonces podemos dibujarlo.\\\\
\begin{figure}[H]
  \centering
  \captionsetup{justification=centering}
  \includegraphics[width=1.0\textwidth]{secciones/imagenes/gpu.png}\label{fig:marcher}
  \caption{Pararelismo de la GPU en dibujado mediante rayos.}
\end{figure}
Definimos lanzar un rayo como aproximar con un vector o utilizar una recta para trazar una escena que encontramos desde el \textit{ojo} en dirección al píxel. Si utilizamos una recta, tratamos de la técnica \textit{raytracing}, que calcula de forma exacta la intersección con los objetos de la escena. Mientras si utilzamos un vector, trataremos de \text{raymarching}, que \textbf{aproxima} la intersección con la escena utilizando una modulación incremental del vector director.\\\\
Computacionalmente, aproximar, suele ser menos costoso que calcular de forma exacta el resultado. Por ello, vamos a utilizar la técnica de \textit{Raymarching}, para aproximar la escena y las \textit{funciones de distancia con signo} para crearla, esta técnica recibe el nombre de \textit{Spheremarching}, que se le ha atribuido el reconocimiento a John Hart en 1996 aunque  se cree que ya otros autores habían investigado acerca de la técnica presentada. \\\\
Utilizaremos el vector director del \textit{ojo} hacia el píxel, incrementando el módulo según el valor de distancia desde la cabeza del vector, de manera recursiva con un número de iteraciones máximas. Recibe el prefijo \enquote{\textit{Sphere}\textendash} ya que, por definición, podemos generar una esfera de radio \(r\), sobre la cabeza del vector, con valor de la función de distancia \(r=sdf(x,y,z)\). Tal que, la esfera, no contine ningún punto en su interior.\\\\
Para fijar el número de iteraciones, se recomienda utilizar una potencia de \(2\). Cuanto mayor sea este número, mejor será la aproximación, aunque, mayor el gasto computacional. No existe un valor por defecto, este dependerá de la escena utilizada y será elegido de forma empírica.\\
Vamos a definir las \textbf{condiciones de parada} de este algoritmo.
\begin{enumerate}
    \item Estar cerca de la isosuperficie. Como el radio de la esfera, indica la distancia a la isosuperfice más cercana, utilizaremos un umbral \(\epsilon\), muy pequeño, para definir la distancia a la que consideramos isosuperficie, en un modelo exacto, \(\epsilon=0.0\).
    \item Superar una cierta distancia recorrida, esta distancia actuará como plano trasero, en caso de superarlo, devolveremos la distancia impuesta para dicho plano. 
    \item Superar el número de iteraciones máximas, en caso afirmativo, devolveremos la distancia al plano trasero.
\end{enumerate}
 En los casos en los que se devuelve la distancia al panel trasero, recibirán el nombre de "\textit{fallo}", que representa un pixel que no ha podido trazar una isosuperficie. Pudiéndose considerar, el fondo de la escena, existen algunos casos, en los que encerraremos nuestra escena en una isosuperficie que actuará como fondo.
\begin{figure}[H]
  \centering
  \captionsetup{justification=centering}%,margin=2cm
  \includegraphics[width=1.0\textwidth]{secciones/imagenes/raymarching.png}\label{fig:spheremarcher}
  \caption{Ejemplo del algoritmo \textit{Spheremarching}}
\end{figure}
\newpage
\begin{lstlisting}
#define PASOS 128 // Número Máximo de Iteraciones.
#define EPSILON 0.001
#define MAXIMO 20.0 // Distancia del Plano Trasero.

// Pasamos el origen del ojo y la dirección, nos devuelve la distancia al objeto más cercano del ojo en dicha dirección.
float SphereMarching(
    vec3 ojo, 
    vec3 direccion
){
    float distancia = 0.0;
    // Realizamos "PASOS" iteraciones de marching.
    for(int i = 0; i < PASOS; ++i){
        // Calculamos el vector (rayo).
        vec3 rayo = ojo + direccion * distancia;
        // Aproximamos el radio de la esfera más próxima a una isosuperficie
        float radio = escena_sdf(rayo);
        // Si el radio (distancia mínima a la isosuperficie), es muy pequeña, podemos decir que estamos sobre la distancia y devolvemos el módulo del rayo.
        if(radio < EPSILON){
            return distancia;
        }
        // Incrementamos la distancia recorrida si no estamos cerca de la isosuperficie.
        distancia += radio;
        // Comprobamos que no se haya superado la distancia de dibujado máximo. Podemos considerarlos el fondo de la escena.
        if(distancia >= MAXIMO) break;
        return MAXIMO;
    }
}
\end{lstlisting}
\newpage
Como ya se ha comentado antes, este algoritmo es aplicado para cada pixel de la pantalla, si el valor devuelto por el \textit{Marcher} es distinto a un \textit{fallo}, sabremos que estamos muy próximos a una isosuperficie y así, dibujarlo. En caso contrario, lo consideraremos fondo. Con la distancia devuelta, podemos posicionarlo en nuestra escena ya que el valor devuelto es la distancia aproximada a la isosuperficie, por lo que, el vector a dicho punto, aproximado, será:
\[ \Vec{p} = \Vec{ojo} + distancia \cdot  \Vec{direccion} \]
Como ya hemos comentado, \textit{Shadertoy} nos ofrece un entorno para trabajar, además de un pequeño código de ejemplo, el cual modificaremos para adaptarlo al \textit{Marcher}.
\begin{lstlisting}
void mainImage( out vec4 fragColor, in vec2 fragCoord )
{
    // Normalized pixel coordinates (from 0 to 1)
    vec2 uv = fragCoord/iResolution.xy;

    // Time varying pixel color
    vec3 col = 0.5 + 0.5*cos(iTime+uv.xyx+vec3(0,2,4));

    // Output to screen
    fragColor = vec4(col,1.0);
}
\end{lstlisting}
Podemos observar tres variables importantes, que se han comentado anteriormente.
\begin{enumerate}
    \item \textit{fragColor}. Se trata de una variable de \textbf{salida}, es el valor del pixel, es decir, una 4-upla \textit{rgba} de tipo \textit{vec4}. Cada elemento está en el intervalo \([0,1]\).
    \item \textit{fragCoord}. Es una variable de \textbf{entrada}, la coordenada del pixel en pantalla, donde la primera componente representa la coordenada \(x\) y la segunda, la \(y\).
    \item \textit{iResolution}. Contiene el ancho y el alto de la pantalla.
\end{enumerate}
La pantalla en realidad nos referimos al donde se dibuja el resultado.
\newpage
\begin{lstlisting}
void mainImage(
    out vec4 fragColor, 
    in vec2 fragCoord
){
    // Normalizamos las coordendas y las reescalamos para mantener el ratio de aspecto. Transladamos al centro de la pantalla.
    vec2 uv = (fragCoord - iResolution.xy * 0.5) / min(iResolution.y, iResolution.x);
    // Definimos el ojo y la pantalla, que se encuentra en nuestra escena.
    vec3 ojo = vec3(0.0, 0.0, -1.0);
    vec3 pantalla = vec3(uv, 0.0);
    // La dirección del rayo es el vector normalizado que apunta desde el ojo hasta la pantalla (píxel).
    vec3 direccion = normalize(pantalla-ojo);
    // Con esto, ya podemos utilizar nuestro Sphere marcher.
    float distancia = SphereMarching(ojo, direccion);
    // El marcher nos ha devuelto una distancia inferior al plano trasero, estamos sobre la isosuperficie.
    if(distancia < MAXIMO){
        // Estamos aproximadamente sobre la isosuperficie.
        // La posición aproximada es la siguiente.
        vec3 p = ojo + direccion * distancia;
        // Utilizamos el color blanco para dibujar la isosuperficie.
        fragColor = vec4(1.0);
    }else{ // El marcher ha fallado.
        // El color negro para pintar el fondo.
        fragColor = vec4(vec3(0.0), 1.0);
    }
}
\end{lstlisting}
\newpage
Si intentáramos ejecutar este código, veríamos que no compilaría, esto es debido a que no hemos definido aún nuestra escena. La función \textit{escena\_sdf} que se encuentra dentro de la función \textit{SphereMarching}, contiene nuestra escena como una \textit{Funcíon de distancia con Signo}, vamos a definir la escena más simple, aunque en capítulos posteriores, veremos formas de crear nuestras propias escenas y funciones de distancia con signo.
\begin{lstlisting}
/* 
No vamos a entrar, aún, en como se define una escena mediante Funciones de Distancia con Signo.
Aunque el siguiente código, representa:
Una esfera en el la coordenada (0,0,0) de radio 0.2 unidades.
*/
float escena_sdf(vec3 p){
    return length(p - vec3(0.0)) - 0.2;
}
\end{lstlisting}
Veamos una pequeña pincelada de como se ha definido esta función. Se calcula el módulo del punto \(\Vec{p}\), esto define una \textit{Función de Distancia con Signo} positiva y cuya isosuperficie es únicamente un punto, \(S=\{(0,0,0)\}\).\\Es fácil observar que si restamos \(r\) al la distancia, estamos creando una isosuperficie esférica de radio \(r\). Aquellos puntos cuya distancias valen \(r\), acabarán anulándose y definiendo la \textit{isosuperficie}. Las distancias inferiores, tomarán valores negativos y los superiores, positivos.
\[S=\{\Vec{q} \in \mathbb{R}^3 / SDFEsfera_r(\Vec{q})=0\}\]
\[ SDFEsfera_r(\Vec{p})=\vert\vert\Vec{p}\vert\vert - r  \]
\begin{figure}[H]
  \centering
  \captionsetup{justification=centering}%,margin=2cm
  \includegraphics[width=1.0\textwidth]{secciones/imagenes/starting/sdf1.png}\label{fig:hello}
  \caption{"Hola mundo" del algoritmo \textit{SDF}.}
\end{figure}
Al solo utilizar dos colores, blanco y negro, no tenemos sensación de profundidad, esto se conseguirá utilizando un modelo de iluminación, que incluyen luces y sombras.
    
    % Modelo de iluminacion
    \chapter{Modelo de iluminación}
\section{Introducción}
En este capítulo vamos a ver los principios de los modelos de iluminación, así como operadores importantes. Seguidamente presentaremos el modelo de iluminación Phong, que es y ha sido utilizado desde los años 70.\\\\Un modelo de iluminación es esencial para el diseño artístico de la escena, simular propiedades físicas como por ejemplo, reflejos, refracción, etc. Además, las sombras dan sensación de profundidad a una escena.\\\\
En este capítulo hablaremos de dos elementos importantes, luces y sombras. Aunque no lo parezca, todos ellos hacen uso de una propiedad fundamental de las superficies, el vector normal. Vamos a crear el algoritmo definidio en el apatartado \textit{Preliminares}.
\begin{lstlisting}
// Cálculo de la normal de la isosuperficie estimado por un rayo.
vec3 Normal(vec3 p){
     // f(x1,...,xn)
     float fxyz = escena_sdf(p);
     // f(x1,..,xi+h,xn)
     float fxhyz = escena_sdf(p + vec3(EPSILON, 0.0, 0.0));
     float fxyhz = escena_sdf(p + vec3(0.0, EPSILON, 0.0));
     float fxyzh = escena_sdf(p + vec3(0.0, 0.0, EPSILON));
     // Utilizamos la definicion de derivadas parciales para devolver el gradiente, que se trata de la normal de la isosuperficie, como hemos definido en los Preliminares.
     return vec3(
         (fxhyz - fxyz) / EPSILON,
         (fxyhz - fxyz) / EPSILON,
         (fxyzh - fxyz) / EPSILON
     );
}
\end{lstlisting}
\newpage
Vamos a ahora a presentar el \textit{producto escalar}, implementado de forma nativa en \textit{GLSL} con el nombre de "\textit{dot}", esencial para los modelos de iluminación.
\[\Vec{r} \cdot  \Vec{v} = r_xv_x + r_yv_y + r_zv_z = \vert r\vert\vert v\vert\cos(\alpha)\]
Si ambos son vectores directores, es decir, normalizados y en el origen, resulta \(\Vec{r} \cdot \Vec{v} = \cos(\alpha)\). El valor \(\alpha\) es el ángulo entre los dos vectores sobre el plano que forman, en caso de \(\mathbb{R}^2\), la componente \(z\) sería nula. La imagen del operador es el intervalo \([-1,1]\)\\Veamos alguna de las propiedades, si ambos vectores son perpendiculares, con \(\alpha=\pm\dfrac{\pi}{2}\), el \textit{producto escalar} será \(\Vec{r}\cdot\Vec{v}=\cos\left(\pm\dfrac{\pi}{2}\right)=0\). En el caso en el que sean son paralelos, \(\alpha=\{0,\pi\}\), el resultado será  \(\Vec{r}\cdot\Vec{v}=\cos(\{0, \pi\})=\pm 1\), según la dirección de ambos.\\\\ El lenguaje \textit{GLSL} presenta dos operaciones vectoriales que utilizaremos en modelo, estas son.
%https://es.m.wikipedia.org/wiki/Ley_de_Snell
\begin{table}[h]
    \begin{tabularx}{\textwidth}{l|X}
        \toprule
        Función & Definición\\
        \midrule
        \pbox{10cm}{
          reflect(\\
          \tab[1cm]vecN a,\\
          \tab[1cm]vecN n, \\
          )} & El vector \(\Vec{n}\) debe estar normalizado, este operador devuelve el vector \(\Vec{a}\) reflectado respecto de \(\Vec{n}\),
        \[\Vec{r}=\Vec{a} - 2(\Vec{n} \cdot \Vec{a})\Vec{n}\]
        \begin{minipage}{1.0\textwidth}
          \centering
          \includegraphics[width=.25\textwidth]{secciones/imagenes/reflect.jpeg}
        \end{minipage}
        \\
        \pbox{10cm}{
        refract(\\
          \tab[1cm]vecN a,\\
          \tab[1cm]vecN n, \\
          \tab[1cm]float k, \\
          )} & El vector \(\Vec{n}\) debe estar normalizado, este operador devuelve el vector \(\Vec{a}\) refractado respecto de \(\Vec{n}\), con \(k\) como factor de medio. Según la \textit{ley de refracción de Snell-Descartes}.
        \[\Vec{r}=k\left(\Vec{a} - \left(\left(\Vec{n} \cdot \Vec{a}\right)+\sqrt{\dfrac{1}{k^2}-(\Vec{a}\cdot\Vec{n})^2}\right)\Vec{n}\right)\]
        \begin{minipage}{1.0\textwidth}
          \centering
          \includegraphics[width=.3\textwidth]{secciones/imagenes/refract.jpeg}
        \end{minipage}\\
        \bottomrule
    \end{tabularx}
\end{table}
\newpage
\section{Luz e Intensidad}
En este apartado veremos la definición de \textit{Intensidad lumínica}, así como dos tipos de luces existentes, luz direccional y radial.\\\\
 Definimos la \textit{intensidad lumínica} en un punto como un factor multiplicativo al material asignado al punto de la \textit{isosuperficie}, representa cómo de iluminado está. Como es un factor multiplicativo, el valor de \(0.0\), representa la intensidad nula u oscuridad. Mientras que el valor \(1.0\) representa el valor más iluminado.\\\\ 
El operador \textit{producto escalar} nos debería dar una breve intuición del papel importante que juega en el cálculo de la intensidad. Como esta no puede ser negativa, definimos el operador producto escalar positivo y normalizado "\(\cdot_{[0,1]}\)".
\[\cdot_{[0,1]}:\mathbb{R}^2\times\mathbb{R}^2\longrightarrow[0,1] : \Vec{a}\cdot_{[0,1]}\Vec{b}=\max\left(\dfrac{\Vec{a}\cdot \Vec{b}}{\vert\vert\Vec{a}\vert\vert\vert\vert \Vec{b}\vert\vert}, 0\right)\]
Vamos a utilizar el "\textit{Modelo de Iluminación de Phong}", presentado en 1973 por \textit{Tuong Phong} como un modelo de iluminación empírico. Para ello, vamos a ver como el modelo se descompone en tres etapas. La primera, el cálculo de la \textbf{Intensidad Ambiente}, que se trata de un valor \(I_a \in [0,1]\) que indica cuanto de iluminada está la isosuperficie, de manera inicial o si no hubieran luces. 
\begin{figure}[H]
  \centering
  \captionsetup{justification=centering}%,margin=2cm
  \includegraphics[width=0.8\textwidth]{secciones/imagenes/lightmodel/ambiental.png}\label{fig:ambient}
  \caption{Intensidad Ambiental sobre la esfera.}
\end{figure}
Por otro lado, la \textbf{Intensidad Especular}, que es aportada de manera colectiva sobre el punto aproximado \(\Vec{p}\) de la \textit{isosuperficie}, para cada una de las luces de la escena \(\Vec{l_i}\in L\), donde \(\Vec{l_i}\) representa la posición de la luz, se comprueba como incide la luz sobre la superficie, con respecto de su normal. 
\[I_d = \sum_{\Vec{l_i}\in L} \Vec{n}\cdot_{[0, 1]}(\Vec{l_i}-\Vec{p})\]
Donde \(\Vec{n}\) es la normal de la \textit{isosuperficie} en el punto \(\Vec{p}\). Es fácil observar que, la intensidad debería ser máxima cuando los rayos inciden en de manera paralela en sentido contrario al vector normal y nulo en caso de que sean perpendiculares u opuestos.
\begin{figure}[H]
  \centering
  \captionsetup{justification=centering}%,margin=2cm
  \includegraphics[width=0.8\textwidth]{secciones/imagenes/lightmodel/difusa.png}\label{fig:difusse}
  \caption{Intensidad Difusa sobre la esfera.}
\end{figure}
Veamos finalmente la \textbf{Intensidad Especular}, esta intensidad indica como incide la luz reflectada  por la \textit{isosuperficie} en el la dirección del ojo.\\
Definimos el operador de reflexión "\(\veebar\)"\footnote{La demostración la podemos encontrar en...}, de un vector \(\Vec{a}\) sobre un vector director \(\Vec{n}\).
\[\Vec{a}\veebar\Vec{n}=\Vec{a} - 2(\Vec{n} \cdot \Vec{a})\Vec{n}\]
La ecuación final de la "\textit{Intensidad Especular}" para todas las luces de la escena es,
\[I_e = \sum_{\Vec{l_i}\in L} \Vec{ojo}\cdot_{[0, 1]}\left(\left(\Vec{l_i}-\Vec{p}\right) \veebar \Vec{n}\right)\]
Algunos autores aportan una leve modificación de esta ecuación, aplicando un \textit{homomorfismo} polinómico con grado exponencial.
\[h_k:[0,1]\longrightarrow[0,1] , h_k(x)=x^{2^k}\]
\[I_d = \sum_{\Vec{l_i}\in L} h_k\left(\Vec{ojo}\cdot_{[0, 1]}\left(\left(\Vec{l_i}-\Vec{p}\right) \veebar \Vec{n}\right)\right)\]
Donde \(k\in\mathbb{R}^{+}\) y este tiene efecto sobre el radio de rayos reflejados.
\begin{figure}[H]
  \centering
  \captionsetup{justification=centering}%,margin=2cm
  \subfloat[Intensidad especular con \(h_0\)]{\includegraphics[width=0.33\textwidth]{secciones/imagenes/lightmodel/especular-0.png}\label{fig:specular-0}}
  \subfloat[Intensidad especular con \(h_1\)]{\includegraphics[width=0.33\textwidth]{secciones/imagenes/lightmodel/especular-1.png}\label{fig:specular-1}}
  \subfloat[Intensidad especular con \(h_3\)]{\includegraphics[width=0.33\textwidth]{secciones/imagenes/lightmodel/especular-2.png}\label{fig:specular-2}}
  \caption{Intensidad Difusa con distintos homomorfismos}
\end{figure}
El modelo final, definido por la \textit{Intesidad del modelo de Phong} se calcula como la suma de las intensidades expuestas anteriormente.
\[I_{Phong}=I_a+\sum_{\Vec{l_i}\in L} \mathrlap{\underbrace{\phantom{\Vec{n}\cdot_{[0, 1]}(\Vec{l_i}-\Vec{p})}}_{\text{Intensidad Difusa}}}\Vec{n}\cdot_{[0, 1]}(\Vec{l_i}-\Vec{p}) + \mathrlap{\underbrace{\phantom{h_k\left(\Vec{ojo}\cdot_{[0, 1]}\left(\left(\Vec{l_i}-\Vec{p}\right) \veebar \Vec{n}\right)\right)}}_{\text{Intensidad Especular}}}h_k\left(\Vec{ojo}\cdot_{[0, 1]}\left(\left(\Vec{l_i}-\Vec{p}\right) \veebar \Vec{n}\right)\right)\]
Como hemos dicho antes, este es un factor, multiplicativo, en particular, multiplica al valor del material o en este caso, el color \textit{rgba} devuelto que era, el blanco.
\begin{figure}[H]
  \centering
  \captionsetup{justification=centering}%,margin=2cm
  \includegraphics[width=0.8\textwidth]{secciones/imagenes/lightmodel/phong.png}\label{fig:phong}
  \caption{Intensidad Phong sobre la esfera con \(h_3\).}
\end{figure}
En algunas implementaciones, se utiliza para cada luz, un factor de atenuación que depende de la distancia de la superficie a la luz, esta función converge a cero en el infinito. En particular, OpenGL\footnote{A partir de la version X.X empezó...} utiliza la siguiente función:
\[p(x)=\dfrac{1}{ax^2+bx+c}\]
donde \(a,b,c \in \mathbb{R}^{+}_{0}\) son factores que dan una riqueza artística al modelo de iluminación, así como el \textit{homomorfismo} utilizado en la \textit{Intensidad Especular}.\\\\En luces direccionales, el punto se considera estar en el infinito, sustituyéndose \(\Vec{l_i}-\Vec{p}\) por el vector director de la luz \(\Vec{d_i}\). En caso de utilizar una función de atenuación, el valor que tomaría  será cero, por lo que la \textit{Intensidad Especular} quedaría anulada. Además, para la \textit{Intensidad Difusa}, utilizaremos el vector director de la luz direccional en vez de la diferencia del punto aproximado \(\Vec{p}\) y la posición de la luz \(\Vec{l_i}\).
\newpage
Veamos un ejemplo práctico en código.
\begin{lstlisting}
// Homomorfismo
float h3(float h){return pow(h,pow(2.,3.));}
// Definimos el operador producto escalar normalizado positivo.
float dot01(vec3 a, vec3 b){ 
    return max(dot(a,b)/(normalize(a)*normalize(b)), 0.0);
}
// Modelo de iluminación Phong
float ModeloIluminacion(vec3 direccion, vec3 p){
    // Calculamos la normal del punto.
    vec3 normal = Normal(p);
    // Ayuda al marcher a escapar de la isosuperficie
    p = p + normal * 0.1;
    // Modelo
    float intensidad = 0.0;
    // Intensidad Ambiente Global
    intensidad += 0.2;
    // Intensidad de cada Luz
    // Luz 1.
    vec3 posicion_luz_1 = vec3(2., 4., 1.);
    vec3 d_luz_1 = posicion_luz_1 - p;
    float dst_luz_1 = length(d_luz_1);
    // Intensidad Difusa
    intensidad += dot01(d_luz_1, normal);
    // Intensidad Especular, en caso de ser una luz direccional, podemos ignorar esta componente ya que la posición es considerada estar en el infinito y por ello, f_difusa = 0
    vec3 r_luz_1 = reflect(d_luz_1, normal);
    intensidad += f_difusa(dst_luz_1) * h3(dot01(r_luz_1, direccion));
    // ... Utilizamos el mismo esquema para las demás luces.
    // Devolvemos la intensidad en el rango [0, 1].
    return clamp(intensidad, 0.0, 1.0);
}
\end{lstlisting}
\newpage
\section{Sombras}
Vamos a ver la técnica más sencilla para calcular las sombras, pero es importante mencionar que hablaremos únicamente de la \textit{umbra} de una la sombra. La \textit{umbra} sucede cuando la fuente de luz es ocluida completamente por una superficie. Haciendo que la luz no actue sobre la intensidad del punto.\\\\Una vez aproximado un punto \(p\) de la \textit{isosuperficie}, diremos que está en \textit{umbra}, si es ocluido por un objeto en dirección a la luz, o lo que es lo mismo, podemos utilizar el marcher desde \(\Vec{p}\) en dirección a la luz \(\Vec{l_i}\), cuyo plano trasero contiene a \(\Vec{l_i}\). Si este traza otro punto \(\Vec{q}\) en esa dirección, este estará en \textit{umbra}.\\\\
Vamos a realizar una pequeña modificación sobre el vector \(\Vec{p}\) que ayudará a agilizar el \textit{Marcher}, ya que las primeras iteraciones de este, intentará "escapar" de la isosuperficie, para ello, vamos a empujarlo de la superficie, utilizando la normal.
\[\Vec{p'}=\Vec{p} + \Vec{n} \cdot k\]
Donde \(k\in\mathbb{R}^{+}_{0}\) y funciona como un factor de empuje de la superficie, se trata de un valor empírico que ayuda al marcher a salir de la isosuperficie, ya que en las primeras iteraciones, el radio de las esferas está muy próximo a \(0,0\). \\\\
% Imagen Desplazamiento 
Se ha realizado además una leve modificación del marcher, ahora este aceptará un tercer argumento, que indica la distancia del plano trasero, que anteriormente estaba fijado por el plano trasero \textit{MAXIMO}. Esto nos será útil para parar el marcher cuando hemos trazado la distancia la distancia a la luz.\\\\
Vamos a crear un modelo de iluminación con \textbf{dos} luces, una radial y otra direccional. Además, utilizaremos un plano en donde proyectar la sombra. La luz direccional es perpendicular al plano.
\begin{figure}[H]
  \centering
  \captionsetup{justification=centering}%,margin=2cm
  \includegraphics[width=0.8\textwidth]{secciones/imagenes/lightmodel/sombra_dura.png}\label{fig:shadow}
  \caption{Modelo de iluminación y sombras sobre la escena definida.}
\end{figure}
\newpage
\begin{lstlisting}
// Se ha añadido un tercer argumento.
float SphereMarching(in vec3 ojo, in vec3 direccion, float distancia_plano){
    float distancia = 0.0;
    for(int i = 0; i < PASOS; ++i){
        vec3 rayo = ojo + direccion * distancia;
        float radio = escena_sdf(rayo);
        if(radio < EPSILON){
            return distancia;
        }
        distancia += radio;
        // Ahora depende del tercer argumento
        if(distancia > distancia_plano)break;
    }
    return distancia_plano;
}
// Phong + Sombras duras
float ModeloIluminacion(vec3 direccion, vec3 p){
    // Calculamos la normal del punto.
    vec3 normal = Normal(p);
    // Ayuda al marcher a escapar de la isosuperficie
    p = p + normal * 0.1;
    // Modelo
    float intensidad = 0.0;
    // Intensidad Ambiente Global
    intensidad += 0.2;
    // Luz 1.
    vec3 posicion_luz_1 = vec3(2., 4., 1.);
    vec3 d_luz_1 = posicion_luz_1 - p;
    vec3 dir_luz_1 = normalize(d_luz_1);
    float dst_luz_1 = length(d_luz_1);
    // En el caso de que se trate de una luz direccional, utilizaremos el plano MAXIMO, utilizado antes.
    if(SphereMarching(pd, dir_luz_1, dst_luz_1) >= dst_luz_1){
        // Intensidad Difusa
        intensidad += dot01(d_luz_1, normal);
        // Intensidad Especular (Si no es direccional)
        vec3 r_luz_1 = reflect(d_luz_1, normal);
        intensidad += f_difusa(dst_luz_1) * h3(dot01(r_luz_1, direccion));
    }
    // ... Repetimos el esquema anterior.
    return clamp(intensidad, 0.0, 1.0);
}
\end{lstlisting}
%https://www.shadertoy.com/view/wtfBW8
\newpage

    
    % Funciones de distancia con signo
    \section{Funciones de distancia con signo (FDS)}

\SectionPage

\subsection{Primitivas sobre \(\mathbb{R}^2\)}
\begin{frame}[fragile]{Primitivas sobre \(\mathbb{R}^2\)}
    
    Funciones de distancia con signo de \cite{2ddistinigo} y \cite{3ddistinigo},
    
    \begin{columns}[c, onlytextwidth]
        \column{1.5in}
            \begin{figure}[H]
              \centering
              \includegraphics[width=1.0\textwidth]{imagenes/sdf/2d/sdf_circunsferencia.png}
            \end{figure}
        
        \column{\dimexpr\paperwidth-10pt}
        
            \begin{lstlisting}
    float SDFCircunsferencia(vec2 p, float r){
        return length(p) - r;
    }
            \end{lstlisting}
        
    \end{columns}
    
    \begin{columns}[c, onlytextwidth]
        \column{\dimexpr\paperwidth-140pt}
            \begin{lstlisting}
    float SDFRectangulo(vec2 p, vec2 s){
            vec2 a = abs(p) - s;
            float extr = length(max(a, 0.0));
            float intr = min(max(a.x, a.y), 0.0);
            return extr + intr;
    }
            \end{lstlisting}
    
        \column{1.5in}
            \begin{figure}[H]
              \centering
              \includegraphics[width=1.0\textwidth]{imagenes/sdf/2d/sdf_rectangulo.png}
            \end{figure}
        
    \end{columns}
    
    \begin{columns}[c, onlytextwidth]
        \column{1.5in}
            \begin{figure}[H]
              \centering
              \includegraphics[width=1.0\textwidth]{imagenes/sdf/2d/sdf_segmento.png}
            \end{figure}
        
        \column{\dimexpr\paperwidth-10pt}
            \begin{lstlisting}
    vec2 proy01(in vec2 a, in vec2 b){
        return b * clamp(dot(b, a) / dot(b, b), 0., 1.);
    }
    float SDFSegmento(vec2 p, vec2 a, vec2 b){
        vec2 v = p - a;
        vec2 w = b - a;
        return length(v -  proy01(v, w));
    }
            \end{lstlisting}
        
    \end{columns}

\end{frame}

% PRIMITIVAS R2
\begin{frame}[fragile]{Operadores sobre \(\mathbb{R}^2\)}

    \begin{columns}[c, onlytextwidth]
        \column{1.5in}
            \begin{figure}[H]
              \centering
              \includegraphics[width=1.0\textwidth]{imagenes/sdf/2d/sdf_traslacion.png}
            \end{figure}
        
        \column{\dimexpr\paperwidth-10pt}
        
            \begin{lstlisting}
                float escena_sdf(vec2 p){
                    vec2 pt = p - vec2(0.1, 0.2);
                    return SDFCircunsferencia(pt, 0.3);
                }
            \end{lstlisting}
        
    \end{columns}
    
    \begin{columns}[c, onlytextwidth]
        \column{\dimexpr\paperwidth-140pt}
            \begin{lstlisting}
    #define PI 3.1415
    mat2 rot(float a){
        return mat2(+cos(a), -sin(a), +sin(a), +cos(a));
    }
    float escena_sdf(vec2 p){
        vec2 pr = p * rot(45. * PI / 180.);
        return SDFRectangulo(pr, vec2(0.3));
    }
            \end{lstlisting}
    
        \column{1.5in}
            \begin{figure}[H]
              \centering
              \includegraphics[width=1.0\textwidth]{imagenes/sdf/2d/sdf_rotacion.png}
            \end{figure}
        
    \end{columns}
    
    \begin{columns}[c, onlytextwidth]
        \column{1.5in}
            \begin{figure}[H]
              \centering
              \includegraphics[width=1.0\textwidth]{imagenes/sdf/2d/sdf_simetria.png}
            \end{figure}
        
        \column{\dimexpr\paperwidth-10pt}
    \begin{lstlisting}
    vec2 simetria(vec2 p, vec2 a, vec2 b){
        return a + simetria(b - a, p - a);
    }
    float escena_sdf(vec2 p){
        vec2 a = vec2(0.2, 0.2), b = vec2(0.0, 0.1);
        vec2 ps = simetria(p, a, b);
        return SDFSegmento(ps, vec2(-0.2, -0.2), vec2(0.3, 0.4));
    }
            \end{lstlisting}
        
    \end{columns}

\end{frame}

% OPERADORES R2
\begin{frame}[fragile]{Operadores sobre \(\mathbb{R}^2\)}

    \begin{columns}[c, onlytextwidth]
        \column{1.5in}
            \begin{figure}[H]
              \centering
              \includegraphics[width=1.0\textwidth]{imagenes/sdf/2d/sdf_add.png}
            \end{figure}
        
        \column{\dimexpr\paperwidth-10pt}
        
            \begin{lstlisting}
    float escena_sdf(vec2 p){
        vec2 pr = p * rot(PI / 180. * 45.0);
        vec2 pt = p - vec2(0.4, 0.15);
        return min(
            SDFRectangulo(pr, vec2(0.3)), // f
            SDFCircunsferencia(pt, 0.3)   // g
        );
    }
            \end{lstlisting}
        
    \end{columns}
    
    \begin{columns}[c, onlytextwidth]
        \column{\dimexpr\paperwidth-140pt}
            \begin{lstlisting}
    float escena_sdf(vec2 p){
        vec2 pr = p * rot(PI / 180. * 45.0);
        vec2 pt = p - vec2(0.4, 0.15);
        return max(
            SDFRectangulo(pr, vec2(0.3)),
            -SDFCircunsferencia(pt, 0.3)
        );
    }
            \end{lstlisting}
    
        \column{1.5in}
            \begin{figure}[H]
              \centering
              \includegraphics[width=1.0\textwidth]{imagenes/sdf/2d/sdf_subtract-3.png}
            \end{figure}
        
    \end{columns}
    
    \begin{columns}[c, onlytextwidth]
        \column{1.5in}
            \begin{figure}[H]
              \centering
              \includegraphics[width=1.0\textwidth]{imagenes/sdf/2d/sdf_deform.png}
            \end{figure}
        
        \column{\dimexpr\paperwidth-10pt}
            \begin{lstlisting}
    float sdf(vec2 p){
    	vec2 pn = vec2(
    	    p.x * cos(p.y * PI),
    	    p.y * sin(p.y * PI)
    	);
    	return SDFCircunferencia(pn, 0.1);
    }
            \end{lstlisting}
        
    \end{columns}

\end{frame}


\subsection{Primitivas sobre \(\mathbb{R}^3\)}
% OPERADORES R3
\begin{frame}[fragile]{Primitivas sobre \(\mathbb{R}^3\)}

    \begin{columns}[c, onlytextwidth]
        \column{1.5in}
            \begin{figure}[H]
              \centering
              \includegraphics[width=1.0\textwidth]{imagenes/sdf/3d/sdf_prisma_rect.png}
            \end{figure}
        
        \column{\dimexpr\paperwidth-10pt}
        
            \begin{lstlisting}
    float SDFPrisma(vec3 p, vec3 s){
        vec3 pa = abs(p) - s;
        return length(max(pa, 0.)) +
        min(max(max(pa.x, pa.y), pa.z), 0.);
    }
            \end{lstlisting}
        
    \end{columns}
    
    \begin{columns}[c, onlytextwidth]
        \column{\dimexpr\paperwidth-140pt}
            \begin{lstlisting}
    float SDFPlano(vec3 p, vec3 n){
       return dot(p, n);
    }
            \end{lstlisting}
    
        \column{1.5in}
            \begin{figure}[H]
              \centering
              \includegraphics[width=1.0\textwidth]{imagenes/sdf/3d/sdf_plano.png}
            \end{figure}
        
    \end{columns}
    
    \begin{columns}[c, onlytextwidth]
        \column{1.5in}
            \begin{figure}[H]
              \centering
              \includegraphics[width=1.0\textwidth]{imagenes/sdf/3d/sdf_capsula.png}
            \end{figure}
        
        \column{\dimexpr\paperwidth-10pt}
            \begin{lstlisting}
    float SDFSegmento(vec3 p, vec3 a, vec3 b){
        vec3 v = p - a;
        vec3 w = b - a;
        return length(v -  proy01(v, w));
    }
    float SDFCapsula(vec3 p, vec3 a, vec3 b, float k){
        return SDFSegmento(p, a, b) - k;
    }
            \end{lstlisting}
        
    \end{columns}

\end{frame}

\begin{frame}[fragile]{Operadores sobre \(\mathbb{R}^3\)}

    \begin{columns}[c, onlytextwidth]
        \column{1.5in}
            \begin{figure}[H]
              \centering
              \includegraphics[width=1.0\textwidth]{imagenes/sdf/3d/sdf_toro.png}
            \end{figure}
        
        \column{\dimexpr\paperwidth-10pt}
        
            \begin{lstlisting}
    float SDFToro(vec3 p, float rx, float r){   
        vec2 rev = vec2(length(p.xz), p.y);
        vec2 pt = rev - vec2(rx, 0.);
        return SDFCircunferencia(pt, r);
    }
            \end{lstlisting}
        
    \end{columns}
    
    \begin{columns}[c, onlytextwidth]
        \column{\dimexpr\paperwidth-140pt}
            \begin{lstlisting}
    float SDFCilindro(vec3 p, float r){
        vec3 n = normalize(vec3(1, 0, 0));
        vec2 proy = proyPlano(p, n).yz;
        return SDFCircunferencia(proy, r);
    }
            \end{lstlisting}
    
        \column{1.5in}
            \begin{figure}[H]
              \centering
              \includegraphics[width=1.0\textwidth]{imagenes/sdf/3d/sdf_cilindro_infinito.png}
            \end{figure}
        
    \end{columns}
    
    \begin{columns}[c, onlytextwidth]
        \column{1.5in}
            \begin{figure}[H]
              \centering
              \includegraphics[width=1.0\textwidth]{imagenes/sdf/3d/sdf_twist.png}
            \end{figure}
        
        \column{\dimexpr\paperwidth-10pt}
            \begin{lstlisting}
    float escena_sdf(vec3 p){
        vec2 ry = p.yz * rot(PI / 4.0);
        p = vec3(p.x, ry.x, ry.y);
        float a = p.y * 10.0;
        p = vec3(
            +p.x * cos(a) + p.z * sin(a),
            +p.y,
            -p.x * sin(a) + p.z * cos(a)
        );
        p.xz * rot(a)
        return SDFPrisma(p, vec3(0.2));
    }
            \end{lstlisting}
        
    \end{columns}

\end{frame}
    
    % Resolución de Artefactos
    \chapter{Resolución de artefactos}
En el capítulo anterior hemos visto dos tipos de \textit{funciones de distancia con signo} exactas e inexactas. Las funciones exactas devuelven la escena de manera correcta, ignorando imperfecciones por coma flotante. Esto es debido a que la métrica es la euclídea y el \textit{Marcher} siempre traza una esfera con ningún punto en el interior, de ahí su nombre, \textit{Spheremarching}. Cuando tratamos de funciones inexactas, la esferas trazadas también se deforman, por lo que las distancias también lo hacen y por tanto, pueden contener puntos. \\\\
Encontramos dos problemas cuando tratamos de \textit{funciones de distancia con signo inexactas}, en el \textit{Marcher} puede sobreestimar la distancia o subestimar. \\\\

\section{Sobreestimación de la distancia}
El término sobreestimar lo definimos como superar la distancia mínima real a la superficie. Vamos a ver dos situaciones de esta sobre estimación: Nos encontramos dentro de la figura, con una distancia negativa o se atraviese completamente la superficie. 

\begin{figure}[H]
  \centering
  \subfloat[Estima dentro de la superficie]{\includegraphics[width=0.4\textwidth]{secciones/imagenes/estimation/sobreestimar-interior.png}\label{fig:estima_dentro}}
  \hfill
  \captionsetup{justification=centering}%,margin=2cm
  \subfloat[Estima fuera de la superficie]{\includegraphics[width=0.4\textwidth]{secciones/imagenes/estimation/sobreestimar-exterior.png}\label{fig:estima_fuera}}
  \caption{Dos formas de sobreestimar una superficie}
\end{figure}

Es difícil visualizar las circunsferencias deformadas, por lo que se ha realizado un esquema de las dos situaciones que podemos encontrar. 

\subsection{Sobrestimación en el interior}
Al poder subestimar, podemos encontrarnos en el interior, haciendo que la siguiente distancia trazada \(d_{n+1}\) sea negativa. Como teníamos definido estar sobre una superficie cuando \(d_{n+1}<\epsilon\), ya que \(d_{n+1}\) siempre es positiva cuando trazábamos desde fuera de la superficie  hacia una \textit{función de distancia con signo exacta}. Ahora que se puede sobreestimar, implicará que \(d_{n+1}\) pueda ser negativo y que debido a la condición de parada expuesta, estos puntos del interior son considerados superficie. Para solucionarlo, diremos que estamos sobre una superficie, si y solo si \( \vert d_{n+1} \vert < \epsilon\), este cambio forzará al \textit{Marcher} a que el rayo tenga que salir del interior de la superficie. Veamos el cambio en el código del algoritmo: 
\begin{lstlisting}
float SphereMarching(
    vec3 ojo, 
    vec3 direccion
){
    float distancia = 0.0;
    for(int i = 0; i < PASOS; ++i){
        vec3 p = ojo + direccion * distancia;
        float radio = escena_sdf(p);
        // Ahora esta d_{n+1} puede ser negativa, por lo que miramos el valor absoluto. 
        if(abs(radio) < EPSILON){
            return distancia;
        }
        // radio puede ser positivo o negativo, en caso de ser negativo, intentará escapar del interior hacia el exterior.
        distancia += radio;
        if(distancia >= MAXIMO) break;
    }
    return MAXIMO;
}
\end{lstlisting}
Veamos el efecto que implica este cambio sobre la deformación vista en \fullref{fig:twist}.

\begin{figure}[H]
  \centering
  \captionsetup{justification=centering}%,margin=2cm
  \subfloat[\textit{Marcher} orginal]{\includegraphics[width=0.4\textwidth]{secciones/imagenes/sdf/3d/sdf_twist.png}\label{fig:twistoriginal}}
  \hfill
  \subfloat[\textit{Marcher} sin sobreestimación en el interior]{\includegraphics[width=0.4\textwidth]{secciones/imagenes/sdf/3d/sin_sobreestimacion_interior.png}\label{fig:stimateext}}
  \caption{Comparativa de los cambios realizados en el \textit{Marcher}. A la izquierda, el \textit{Marcher} original, a la derecha, el marcher con \(\vert d_{n+1}\vert < \epsilon\)}
\end{figure}

Aunque los últimos cambios son poco apreciables, podemos observar que se han reducido el número de artefactos. Por ejemplo, ahora la tapa es algo más exacta. 

\section{Sobreestimación de la distancia en el exterior}
Este segundo caso ocurre cuando la deformación aplicada hace que el rayo atraviese la superficie, como ocurre en el ejemplo \textit{\(B\)}.
Esta sobreestimación afecta considerablemente a la eficiencia del \textit{Marcher}. La solución es escalar la distancia más cercana a la superficie, obligando a que la bola deformada no pueda contener ningún punto, lo que provocará un escalado del módulo del rayo. Esto a su vez conlleva a requerir un mayor número de iteraciones.
\[d'_{n+1}=\left(\sum^{n-1}_{i=0} d_{i}\right) + f(\Vec{rayo})\cdot k \leq d_{n+1}\]
El vector \(\Vec{rayo}\) hace referencia al vector lanzado desde el ojo hasta el pixel, \(f\) es nuestra escena y \(k\in[0,1]\) es un factor de escalado para la evitar la sobreestimación, el valor es tomado de manera manual.

\begin{lstlisting}
#define FACTOR_SOBREESTIMACION k
float SphereMarching(
    vec3 ojo, 
    vec3 direccion
){
    float distancia = 0.0;
    for(int i = 0; i < PASOS; ++i){
        vec3 p = ojo + direccion * distancia;
        // Escalado del la bola
        float radio = escena_sdf(p) * FACTOR_SOBREESTIMACION;
        if(abs(radio) < EPSILON){
            return distancia;
        }
        distancia += radio;
        if(distancia >= MAXIMO) break;
    }
    return MAXIMO;
}
\end{lstlisting}
Veamos como afecta el factor \(k\) al trazado de la escena por el \textit{Marcher} sin sobreestimación en el interior.

\begin{figure}[H]
  \centering
  \captionsetup{justification=centering}%,margin=2cm
  \subfloat[k=1.0]{\includegraphics[width=0.3\textwidth]{secciones/imagenes/sdf/3d/sin_sobreestimacion_interior.png}\label{fig:twistoriginal1}}
  \hfill
  \subfloat[k=0.75]{\includegraphics[width=0.3\textwidth]{secciones/imagenes/sdf/3d/sobreestimacion_75.png}\label{fig:twist75}}
  \hfill
  \subfloat[k=0.50]{\includegraphics[width=0.3\textwidth]{secciones/imagenes/sdf/3d/sobreestimacion_50.png}\label{fig:twist50}}
  \caption{\(k\in\{1, 0.75, 0.5\}\) respectivamente sobre el \textit{Marcher} anterior}
\end{figure}

Vemos que la figura es \enquote{exacta} con \(k=0.5\) si observamos la esquina inferior izquierda de la tapadera. Este factor, o reducción de la esfera a la mitad, va a provocar que se requiera el doble de iteraciones para trazar una escena.

\subsection{Subestimación de la distancia}

Esto ocurre cuando el \textit{Marcher} ha acabado y la distancia recorrida por el rayo es inferior al plano trasero, es decir, sigue en la escena. Esto puede ocurrir principalmente cuando el rayo pasa perpendicularmente muy cerca a una superficie \(f(\Vec{rayo}) \ge \epsilon\). La siguiente imagen ilustra este problema:

\begin{figure}[H]
  \centering
  \captionsetup{justification=centering}%,margin=2cm
  \includegraphics[width=0.8\textwidth]{secciones/imagenes/estimation/subestimacion.png}\label{fig:subestimacion}
  \caption{Ejemplo de subestimación de una superficie}
\end{figure}

Estos puntos serán tratados como \textit{fallos} y por tanto, como píxeles de fondo, aunque la escena no lo fuera. Utilizar un factor de sobreestimación \(k > 1\) no sería la solución ya que crearía \textit{artefactos}, la solución propuesta es el incremento del número de iteraciones del algoritmo, en código, incrementar la definición de \textit{PASOS}, provocando un mayor gasto computacional.
    
    % Materiales
    \chapter{Materiales\label{ch:materiales}}
En este último capítulo, vamos a ver como dar color y texturas a los elementos de nuestra escena. Identificaremos cada elemento asignando un entero positivo \(id \in \mathbb{N}\) que será devuelto junto con la distancia a este objeto, es decir, vamos a devolver un \textit{vec2} cuya componente \enquote{x} será la distancia y cuya componente \enquote{y}, el identificador \(id\). Asignaremos la constante \(id=-1\) cuando no se ha trazado ningún objeto.\\\\
En primer luegar, vamos a modificar el \textit{Marcher} para que este pueda devolver ambos valores:

\begin{lstlisting}
// Devolvemos dos elementos, distancia e id.
vec2 SphereMarching(vec3 ojo, vec3 direccion){
    float distancia = 0.0;
    // Realizamos PASOS iteraciones de marching.
    for(int i = 0; i < PASOS; ++i){
        vec3 p = ojo + direccion * distancia;
        // La escena devuelve el radio de la bola y el id del elemento
        vec2 info = escena_sdf(p);
        // info.x contiene la distancia
        if(info.x < EPSILON){
            // info.y contiene el id de un elemento de la escena.
            // Devolvemos la distancia acumulada (o distancia del ojo a la superficie) y el id.
            return vec2(distancia, info.y);
        }
        // incrementamos la distancia
        distancia += info.x;
        if(distancia >= MAXIMO) break;
    }
    // Devolvemos un id desconocido.
    return vec2(MAXIMO, -1);
}
\end{lstlisting}

El vector devuelto con nombre \textit{info}, toma los valores directamente de la escena, por lo que vamos a modificar el esquema de nuestra función \enquote{\textit{escena\_sdf}}, este ahora devolverá la distancia más cercana a un objeto y su identificador. El esquema será el siguiente:

\begin{lstlisting}
vec2 escena_sdf(vec3 p){
    // Identificador inicial y la distancia máxima.
    float id = -1.0;
    float min_dist = MAXIMO;
    
    // El esquema es el siguiente para cada figura de nuestra escena.
    // 1.Creamos nuestra primera figura.
    float sdf_0 = ....;
    // 2.Comprobamos que esta figura es la más cercana encontrada hasta el momento.
    if(sdf_0 < min_dist){
        // 2.1 En caso afirmativo, actualizamos los valores.
        // Asignamos el id de esta figura.
        id = 0.;
        // Actualizamos la distancia mínima como la distancia a esa figura.
        min_dist = sdf_0;
    }
    
    // Repetimos este esquema para cada elemento de la escena,
    float sdf_1 = ...;
    if(sdf_1 < min_dist){
        id = 1.;
        min_dist = sdf_1;
    }
    ...
    // Finalmente, devolvemos la distancia mínima y el objeto que la devuelve.
    return vec2(min_dist, id);
}
\end{lstlisting}

Al devolver ahora dos componentes, debemos modificar todas las funciones que hacían uso de esta función, donde encontramos \enquote{Normal} y \enquote{ModeloIluminacion}. Vamos a utilizar el valor devuelto, asignar un material, para ello, vamos a crear una nueva función \enquote{obtenerMaterial} Pongamos como ejemplo una sección de un toro y una esfera, el código y apliquemos el esquema utilizado para \enquote{\textit{escena\_sdf}}. El resultado:

\begin{lstlisting}
vec2 escena_sdf(vec3 p){
    // Identificador inicial y distancia máxima.
    float id = -1.0;
    float min_dist = MAXIMO;
    // Toro de radio interno 0.3 y radio externo 0.05.
    // Seccionado por un plano n = -z.
    // Rotamos el toro 
    vec3 pr = rotYZ(p, PI / 4.);
    float sdf_0 = max(
        SDFToro(pr, 0.3, 0.05),
        SDFPlano(p, vec3(0., 0., -1.))
    );
    // Comprobamos que sea la mas cercana.
    if(sdf_0 < min_dist){
        // Identificador del toro
        id = 0.;
        min_dist = sdf_0;
    }
    // Esfera de radio 0.2
    float sdf_1 = SDFEsfera(p, 0.2);
    // Comprobamos que sea la mas cercana.
    if(sdf_1 < min_dist){
        // identificador de la esfera.
        id = 1.;
        min_dist = sdf_1;
    }
    // Finalmente, devolvemos la distancia mínima y el objeto que la devuelve.
    return vec2(min_dist, id);
}
\end{lstlisting}


\begin{figure}[H]
  \centering
  \captionsetup{justification=centering}%,margin=2cm
  \includegraphics[width=1.0\textwidth]{secciones/imagenes/material/materiales.png}\label{fig:material}
  \caption{Materiales asignados a las distintas figuras.}
\end{figure}

Enlace del ejemplo:\url{https://www.shadertoy.com/view/wlBBRR}
    
    % Conclusion
    \section{Conclusiones}

\SectionPage

\begin{frame}{Conclusiones}
    \begin{itemize}
        \item Se trata de una técnica novedosa y con un ámplio campo de estudio, que requiere de un elevado conocimiento matemático.
        \item El modelo de iluminación es esencial para la creación de escenas tridimensionales.
        \item Utilizar funciones de distancia con signo exactas, que ayudan a la convergencia del algoritmo.
        \item La sub/sobreestimación, requiere de un mayor ejercicio computacional.
        \item Los materiales dan una riqueza visual al ejercicio, en caso de texturización, utilizaremos proyecciones sobre las coordenadas \((u,v)\).
    \end{itemize}
\end{frame}

\end{document}